
\begin{abstract}

Modern storage systems employing quorum replication are often
configured to use partial, or non-strict quorums.  These systems wait
only for a subset of their replicas to respond to a request before
returning an answer without guaranteeing that read and write sets
overlap.  While these real-world partial quorum mechanisms provide
only basic eventual consistency guarantees, these configurations are
frequently ``good enough'' for practitioners given their latency
benefits. In this work, we discuss why partial quorums are often
acceptable in practice by analyzing the staleness of data they return.
Extending prior work on strongly consistent probabilistic quorums and
using the models of anti-entropic Dynamo-style processes, we introduce
Probabilistically Bounded Staleness (PBS) consistency, which provides
expectations of bounds on staleness across both versions and wall
clock time.  We derive a closed-form solution for versioned staleness
and model real-time staleness for representative Dynamo-style systems
under internet-scale production workloads.  We quantitatively
demonstrate why, in practice, systems employing non-strict quorums
rarely serve inconsistent data.

\end{abstract}
