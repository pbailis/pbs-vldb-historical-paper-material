
\begin{abstract}

Data store replication leads to a fundamental trade-off between
operation latency and data consistency: stronger consistency typically
requires more coordination, resulting in increased
latency. Accordingly, systems employing quorum replication are often
configured to use partial, non-strict quorums, waiting only for a
subset of replicas to respond to a request before returning data and
without guaranteeing that read and write replica sets intersect.
Partial quorums provide only basic eventual consistency guarantees,
with no limit to the recency of data returned, yet, anecdotally, they
are often ``good enough'' for practitioners given their latency
benefits. In this work, we explain why partial quorums are regularly
acceptable in practice by analyzing both the staleness of data they
return and the latency benefits they offer. We introduce
Probabilistically Bounded Staleness (PBS) consistency, which provides
expected bounds on staleness with respect to both versions and wall
clock time.  We derive a closed-form solution for versioned staleness
and model real-time staleness for representative Dynamo-style systems
under internet-scale production workloads.  We quantitatively
demonstrate why, in practice, eventually consistent systems using
partial quorums frequently serve consistent data.

\end{abstract}
