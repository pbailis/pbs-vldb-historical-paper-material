
\begin{abstract}

\begin{comment}
Distributed storage systems replicate data items to provide better
performance and availability. Many of these systems use overlapping
read and write sets, or quorums, to provide to provide consistent
reads and writes.  However, to lower the variance of user-visible
latency and handle intermittent failures, storage systems designed for
large clusters opt for ``eventual consistency,'' using \textit{non-strict}, or non-overlapping, read and write quorums. Further, operators are
required to choose a fixed number of read/write replicas to maintain
operation target latencies.
\end{comment}

Eventually consistent data stores naturally provide a spectrum of
bounded-staleness consistency properties.  We demonstrate how to
choose amongst them.  We analyze partial quorum replica management
policies to balance the trade-off between consistency and operation
performance while maintaining high availability.  By extending prior
work on strongly consistent probabilistic quorums and using the models
of anti-entropic Dynamo-style processes, we introduce
Probabilistically Bounded Staleness (PBS) consistency, which provides
expected bounds on staleness across multiple writes and wall clock
time for non-strict quorums.  Furthermore, we can optimize the quorum
size to meet latency targets chosen by the operator and dynamically
adjust the read and write sets. We subsequently deploy our policies on
Cassandra, an open source distributed database used in production.
Through a combination of benchmarking and analysis, we show that even
under high variance in operation latency, partial quorums rarely
result in stale reads.

\begin{comment}
and our
synthetic benchmarks show that our predicted staleness is XX\% within
the theoretical bound.  Finally, we present a case study of how a
social networking application can configure its replica management
scheme to meet its latency targets and our prototype implementation is
XX\% faster than using strict quorums.
\end{comment}

\end{abstract}
