
\begin{abstract}

Data store replication results in a fundamental trade-off between
operation latency and data consistency. In this paper, we examine this
trade-off in the context of quorum-replicated data stores. Under
partial, or non-strict quorum replication, a data store waits for
responses from a subset of replicas before answering a query, without
guaranteeing that read and write replica sets intersect. As deployed
in practice, these configurations provide only basic eventual
consistency guarantees, with no limit to the recency of data
returned. However, anecdotally, partial quorums are often ``good
enough'' for practitioners given their latency benefits. In this work,
we explain why partial quorums are regularly acceptable in practice,
analyzing both the staleness of data they return and the latency
benefits they offer. We introduce Probabilistically Bounded Staleness
(PBS) consistency, which provides expected bounds on staleness with
respect to both versions and wall clock time.  We derive a closed-form
solution for versioned staleness as well as model real-time staleness
for representative Dynamo-style systems under internet-scale
production workloads. Using PBS, we measure the
latency-consistency trade-off for partial quorum systems. We quantitatively
demonstrate how eventually consistent systems frequently return
consistent data within tens of milliseconds while offering significant
latency benefits.

\end{abstract}
