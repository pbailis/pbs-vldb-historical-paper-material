
\begin{abstract}

\begin{comment}
Distributed storage systems replicate data items to provide better
performance and availability. Many of these systems use overlapping
read and write sets, or quorums, to provide to provide consistent
reads and writes.  However, to lower the variance of user-visible
latency and handle intermittent failures, storage systems designed for
large clusters opt for ``eventual consistency,'' using \textit{non-strict}, or non-overlapping, read and write quorums. Further, operators are
required to choose a fixed number of read/write replicas to maintain
operation target latencies.
\end{comment}

Modern quorum-replicated storage systems are often configured to use
non-strict, or non-overlapping, quorums, waiting only for a subset of their
replicas to respond to a request before returning an answer.  While
these real-world partial quorums provide only the barest of eventually
consistent guarantees, these configurations are frequently ``good enough'' for
practitioners given their latency benefits. In this work, we discuss why partial quorums are often
acceptable in practice by analyzing the staleness of data they return.
Extending prior work on strongly consistent probabilistic quorums
and using the models of anti-entropic Dynamo-style processes, we
introduce Probabilistically Bounded Staleness (PBS) consistency, which
provides expected bounds on staleness across multiple writes and wall
clock time for non-strict quorums.  We derive a closed-form solution
for versioned staleness and model real-time staleness for
representative Dynamo-style systems under internet-scale production
workloads.  We quantitatively demonstrate why, in practice, non-strict
quorums are frequently strongly consistent.

\begin{comment}
and our
synthetic benchmarks show that our predicted staleness is XX\% within
the theoretical bound.  Finally, we present a case study of how a
social networking application can configure its replica management
scheme to meet its latency targets and our prototype implementation is
XX\% faster than using strict quorums.
\end{comment}

\end{abstract}
