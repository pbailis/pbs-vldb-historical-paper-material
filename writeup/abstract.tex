
\begin{abstract}

Modern storage systems employing quorum replication are often
configured to use partial, non-strict quorums.  These systems wait
only for a subset of their replicas to respond to a request before
returning an answer, without guaranteeing that read and write replica
sets overlap.  While these partial quorum mechanisms provide only
basic eventual consistency guarantees, with no limit to the recency of
data returned, these configurations are frequently ``good enough'' for
practitioners given their latency benefits. In this work, we discuss
why partial quorums are often acceptable in practice by analyzing the
staleness of data they return.  Extending prior work on strongly
consistent probabilistic quorums and using models of Dynamo-style
anti-entropy processes, we introduce Probabilistically Bounded
Staleness (PBS) consistency, which provides expectations of bounds on
staleness across both versions and wall clock time.  We derive a
closed-form solution for versioned staleness and model real-time
staleness for representative Dynamo-style systems under internet-scale
production workloads.  We quantitatively demonstrate why, in practice,
systems employing partial quorums often serve consistent data.

\end{abstract}
