\documentclass{vldb}
\usepackage{graphicx, subfigure, multirow, times, balance}
\usepackage{balance, url, amsfonts, verbatim, mathtools}

\renewcommand{\ttdefault}{cmtt}

\newcommand{\sectionskip}{-0em}
\newcommand{\subsectionskip}{-0em}
\usepackage{mdwlist}

\newenvironment{myitemize}
{
   \vspace{0mm}
    \begin{list}{$\bullet$ }{}
        \setlength{\topsep}{0em}
        \setlength{\parskip}{0pt}
        \setlength{\partopsep}{0pt}
        \setlength{\parsep}{0pt}         
        \setlength{\itemsep}{1mm} 
}
{
    \end{list} 
    \vspace{-1em}
}

\title{Probabilistically Bounded Staleness\\ for Practical Partial Quorums}

\author{Peter Bailis, Shivaram Venkataraman, Michael J. Franklin, Joseph M. Hellerstein, Ion Stoica\\
\affaddr{University of California, Berkeley}\\
\affaddr{\{pbailis, shivaram, franklin, hellerstein, istoica\}@cs.berkeley.edu}}

\newdef{definition}{Definition}

\begin{document}

\interfootnotelinepenalty=10000
\hyphenation{prob-a-bil-is-tic-ally}

\maketitle

\begin{quote}
\textit{All good ideas arrive by chance.}---Max Ernst
\end{quote}


\begin{abstract}

Modern storage systems employing quorum replication are often
configured to use partial, or non-strict quorums.  These systems wait
only for a subset of their replicas to respond to a request before
returning an answer without guaranteeing that read and write sets
overlap.  While these real-world partial quorum mechanisms provide
only basic eventual consistency guarantees, these configurations are
frequently ``good enough'' for practitioners given their latency
benefits. In this work, we discuss why partial quorums are often
acceptable in practice by analyzing the staleness of data they return.
Extending prior work on strongly consistent probabilistic quorums and
using the models of anti-entropic Dynamo-style processes, we introduce
Probabilistically Bounded Staleness (PBS) consistency, which provides
expectations of bounds on staleness across both versions and wall
clock time.  We derive a closed-form solution for versioned staleness
and model real-time staleness for representative Dynamo-style systems
under internet-scale production workloads.  We quantitatively
demonstrate why, in practice, systems employing non-strict quorums
rarely serve inconsistent data.

\end{abstract}


\vspace{\sectionskip}\section{Introduction}

Modern distributed data stores need to be scalable, highly available,
and fast.  These systems typically replicate data across different
machines and often across datacenters for two reasons: first, to
provide high availability when components fail and, second, to provide
improved performance by serving requests from multiple replicas.  In
order to provide predictably low read and write latency, systems often
eschew protocols guaranteeing consistency of reads\footnote{This
  distributed replica consistency differs from transactional
  consistency provided by ACID semantics~\cite{nosql,
    urbanmyths}.} and instead opt for eventually
consistent protocols~\cite{cassandradefault, abadilatconsist, dynamo,
  feinbergpc, reddit, riaktalkone, outbrain}.  However, eventually
consistent systems make no guarantees on the staleness (recency in
terms of versions written) of data items returned except that the
system will ``eventually'' return the most recent version in the
absence of new writes~\cite{vogels-defs}.

This latency-consistency trade-off inherent in distributed data stores
has significant consequences for application
design~\cite{abadilatconsist}. Low latency is critical for a large
class of applications~\cite{perf-impact}. For example, at Amazon, 100
ms of additional latency resulted in a 1\% drop in
sales~\cite{amazon-latency}, while 500 ms of additional latency in
Google's search resulted in a corresponding 20\% decrease in
traffic~\cite{google-talk}.  At scale, increased latencies correspond
to large amounts of lost revenue, but lowering latency has a
consistency cost: contacting fewer replicas for each request typically
weakens the guarantees on returned data. Programs can often tolerate
weak consistency by employing careful design patterns such as
compensation (e.g., memories, guesses, and apologies)~\cite{helland}
and by using associative and commutative operations (e.g., timelines,
logs, and notifications)~\cite{calm}.  However, potentially
\textit{unbounded} staleness (as in eventual consistency) poses
significant challenges and is undesirable in practice.

\subsection{Practical Partial Quorums}

In this work, we examine the latency-consistency trade-off in the context of
quorum-replicated data stores. Quorum systems ensure
strong consistency across reads and writes to replicas by ensuring
that read and write replica sets overlap. However, employing
\textit{partial} (or non-strict) quorums can lower latency by
requiring fewer replicas to respond.  With partial quorums, sets of
replicas written to and read from need not overlap: given $N$ replicas
and read and write quorum sizes $R$ and $W,$ partial quorums imply
$R$$+$$W$$\leq$$N$.

Quorum-replicated data stores such as Dynamo~\cite{dynamo} and its
open source descendants Apache Cassandra~\cite{cassandra-sigmod},
Basho Riak~\cite{riak}, and Project Voldemort~\cite{voldemortpub}
offer a choice between two modes of operation: strict quorums with
strong consistency or partial quorums with eventual
consistency. Despite eventual consistency's weak guarantees, operators
frequently employ partial quorums~\cite{cassandra-docs,
  cassandradefault,feinbergpc,reddit, outbrain, maxperfblog}---a
controversial decision~\cite{hamilton-cap, cops, walter, urbanmyths}.
Given their performance benefits, which are especially important as
latencies grow~\cite{abadilatconsist, feinbergpc, hamilton-cap,
  helland}, partial quorums are often considered acceptable.  The proliferation
of partial quorum deployments suggests that applications can often
tolerate occasional cases of staleness and that data tends to be
``fresh enough'' in most cases.

While common practice suggests that eventual consistency is often a viable
solution for operators, to date, this observation has been
anecdotal. In this work, we quantify the degree to which eventual
consistency is both eventual and consistent and explain why. Under
worst-case conditions, eventual consistency results in an unbounded
degree of data staleness, but, as we will show, the average case is
often different.  Eventually consistent data stores cannot promise
immediate and perfect consistency but, for varying degrees of
certainty, can offer staleness bounds with respect to time (``how
eventual'') and version history (``how consistent'').

There is little prior work describing how to make these consistency
and staleness predictions under practical conditions.  The current
state of the art requires that users make rough guesses or perform
online profiling to determine the consistency provided by their data
stores~\cite{measure-consistency, podc-hpl, consistency-cidr}. Users
have little to no guidance on how to chose an appropriate
replication configuration or how to predict the behavior of partial
quorums in production environments.

\subsection{PBS Predictions and Contributions}

To predict consistency, we need to know when and why eventually
consistent systems return stale data and how to quantify the staleness
of the data they return.  In this paper, we answer these questions by
expanding prior work on \textit{probabilistic
  quorums}~\cite{prob-quorum, quorum-overview} to account for
multi-version staleness and message dissemination protocols as used in
today's systems. More precisely, we present algorithms and models for
predicting the staleness of partial quorums, called Probabilistically
Bounded Staleness (PBS). There are two common metrics for measuring
staleness in the literature: wall clock time~\cite{podc-hpl,
  vahdat-article, vahdat-bounded} and versions~\cite{podc-hpl, aqua,
  frac}.  PBS describes both measures, providing the probability of
reading a write $t$ seconds after it returns ($t$-visibility, or
``how eventual is eventual consistency?''), of reading one of the last
$k$ versions of a data item ($k$-staleness, or ``how consistent is
eventual consistency?''), and of experiencing a combination of the two
($\langle k, t \rangle$-staleness). PBS does not propose new
mechanisms to enforce deterministic staleness bounds~\cite{ aqua,
  trapp,vahdat-article, vahdat-bounded, frac}; instead, our goal is to
provide a lens for analyzing, improving, and predicting the behavior
of \textit{existing}, widely-deployed systems.

We provide closed-form solutions for PBS $k$-staleness and use Monte
Carlo methods to explore the trade-off between latency and
$t$-visibility.  We present a detailed study of Dynamo-style PBS
$t$-visibility using production latency distributions. We show how
long-tailed one-way write latency distributions affect the
time required for a high probability of consistent reads.  For
example, in one production environment, switching from spinning disks
to solid-state drives dramatically improved staleness (e.g., $1.85$ms
versus $45.5$ms wait time for a $99.9$\% probability of consistent
reads) due to decreased write latency mean and variance.  We also make
quantitative observations of the latency-consistency trade-offs
offered by partial quorums.  For example, in another production
environment, we observe an $81.1\%$ combined read and write latency
improvement at the $99.9$th percentile ($230$ to $43.3$ms) for a
$202$ms window of inconsistency ($99.9\%$ probability consistent
reads). This analysis demonstrates the performance benefits that lead
operators to choose eventual consistency.

We make the following contributions in this paper:

\begin{myitemize}

\item We develop the theory of Probabilistically Bounded Staleness
  (PBS) for partial quorums. PBS describes the probability of
  staleness across both versions ($k$-staleness) and time
  ($t$-visibility) as well as the probability of session-based
  monotonic reads consistency.

\item We provide a closed-form analysis of $k$-staleness demonstrating
  how the probability of receiving data $k$ versions old is
  exponentially reduced by $k$.  As a corollary, $k$-staleness tolerance also
  exponentially lowers quorum system \textit{load}.

\item We describe the \textit{WARS} model for $t$-visibility in
  Dynamo-style partial quorum systems and show how message reordering
  leads to staleness.  We evaluate the $t$-visibility of Dynamo-style
  systems using a combination of synthetic and production latency
  models.

\end{myitemize}

\vspace{\sectionskip}\section{Background}
\label{sec:background}

In this section, we provide background on quorum systems both
in the theoretical academic literature and in practice.  We begin by
introducing prior work on traditional and probabilistic quorum
systems.  We next discuss Dynamo-style quorums, currently the most
widely deployed protocol for data stores employing quorum replication.
Finally, we survey reports of practitioner usage of partial quorums
for three Dynamo-style data stores.

\vspace{\subsectionskip}\subsection{Quorum Foundations: Theory}

Systems designers have long proposed quorum systems as a replication
strategy for distributed data~\cite{quorums-start}.  Under quorum
replication, a data store writes a data item by sending it to a set of
replicas, called a write quorum.  To serve reads, the data store
fetches the data from a possibly different set of replicas, called a
read quorum.  For reads, the data store compares the set of values
returned by the replicas, and, given a total ordering of
versions,\footnote{We can easily achieve a total ordering using
  globally synchronized clocks or using a causal ordering provided by
  mechanisms such as vector clocks~\cite{vectorclock} with commutative
  merge functions~\cite{cops}} can return the most recent value (or
all values received, if desired).  For each operation, the data store
chooses read and write quorums from a set of sets of replicas, known
as a \textit{quorum system}, with one system per data item.  There are
many kinds of quorum systems, but one simple configuration is to use
read and write quorums of fixed sizes, which we will denote $R$ and
$W$, for a set of nodes of size $N$.  To reiterate, a quorum
replicated data store uses one quorum system per data item.  Across
data items, quorum systems need not be identical

Informally, a strict quorum system is a quorum system with the
property that any two quorums (sets) in the quorum system overlap
(have non-empty intersection). This ensures consistency.  The minimum
sized quorum defines the system's fault tolerance, or availability.  A
simple example of a strict quorum system is the majority quorum
system, in which each quorum is of size $\lceil \frac{N}{2}\rceil$.
The theory literature describes alternative quorum system designs
providing varying asymptotic properties of capacity, scalability, and
fault tolerance, from tree-quorums~\cite{treequorum} to
grid-quorums~\cite{quorumsystems} and highly-available
hybrids~\cite{92-quorums}.  Jim\'{e}nez-Peris et al. provide an
overview of traditional, strict quorum
systems~\cite{quorums-alternative}.

Partial quorum systems are natural extensions of strict quorum
systems: at least two quorums in a partial quorum system do not
overlap.  There are two relevant variants of partial quorum systems in
the literature: probabilistic quorum systems and k-quorums.

\textit{Probabilistic quorum systems} provide probabilistic guarantees
of quorum intersection.  By scaling the number of replicas, one can
achieve an arbitrarily high probability of
consistency~\cite{prob-quorum}.  Intuitively, this is a consequence of
the Birthday Paradox: as the number of replicas increases, the
probability of non-intersection between any two quorums decreases.
Probabilistic quorums are typically used to predict the probability of
strong consistency but not (multi-version) bounded staleness.
Merideth and Reiter provide an overview of these
systems~\cite{quorum-overview}.

As an example of a probabilistic quorum system, consider $N$ replicas
with randomly chosen read and write quorums of sizes $R$ and $W$. We
can calculate the probability that the read quorum does not contain
the last written version. This probability is the number of quorums of
size $R$ composed of nodes that were not written to in the write
quorum divided by the number of possible read quorums:
\begin{equation}
\label{eq:prob-strict}
p_{s}=\frac{{N-W \choose R}}{{N \choose R}}
\end{equation}
The probability of inconsistency is high except for large $N$.  With
$N=100$, $R=W=30$, $p_{s} = 1.88 \times
10^{-6}$~\cite{non-strict}.  However, with $N=3$, $R=W=1$, $p_{s}
= 0.\overline{6}$.  The asymptotics of these systems are
excellent---but only asymptotically.

\textit{$k$-quorum systems} provide \textit{deterministic} guarantees
that a partial quorum system will return values that are within $k$
versions of the most recent write~\cite{non-strict}.  In a single
writer scenario, sending each write to to $\lceil\frac{N}{k}\rceil$
replicas with round-robin write scheduling ensures that any replica is
no more than $k$ versions out-of-date.  However, with multiple
writers, we lose the global ordering properties that the single-writer
was able to control, and the best known algorithm for the pathological
case results in a lower bound of $(2N-1)(k-1)+N$ versions
staleness~\cite{multi-k-quorum}.

This prior work makes two important assumptions. First, it typically
models quorum sizes as fixed, where the set of nodes with a version
does not grow over time.  Prior work examined ``dynamic systems'',
considering quorum membership churn~\cite{prob-quorum-dynamic},
network-aware quorum placement~\cite{delay-quorum, quorum-placement},
and network partitions~\cite{partitionedquorum} but not write
propagation. Second, it frequently assumes Byzantine failure.  We
revisit these assumptions in the next section.

\vspace{\subsectionskip}\subsection{Quorum Foundations: Practice}
\label{sec:practice}

In practice, many distributed data management systems use quorums as a
replication mechanism. Amazon's Dynamo~\cite{dynamo} is the progenitor
of a class of eventually-consistent key-value stores that includes
Apache Cassandra~\cite{cassandra-sigmod}, Basho Riak~\cite{riak}, and LinkedIn's Project
Voldemort~\cite{voldemortpub}.  All use the same variant of
quorum-style replication and we are not aware of any widely adopted
data store using a vastly different quorum replication protocol.
However, with some work, we believe that other styles of replication
can adopt our methodology.  We describe key-value stores here, but any
replicated data store can use quorums, including full RDBMS systems.

Dynamo-style quorum systems employ one quorum system per key,
typically maintaining the mapping of keys to quorum systems using a
consistent-hashing scheme or a centralized membership protocol. Each
node stores multiple keys.  As shown in
Figure~\ref{fig:dynamo-quorum}, clients send read and write requests
to a node in the system cluster, which forwards the request to
\textit{all} nodes assigned to that key as replicas.  This
coordinating node considers an operation complete when it has received
responses from a pre-determined number of replicas (typically set
per-operation).  Accordingly, without message loss, all replicas
eventually receive all writes.  This means that the write and read
quorums chosen for a request depend on which nodes respond to the
request first.  Dynamo denotes $N$ as the the replication factor of a
key, $R$ as the number of replica responses required for a successful
read, and $W$ as the number of replica acknowledgments required for a
successful write. Under normal operation, Dynamo-style systems
guarantee consistency when $R$$+$$W$$>$$N$.  Setting
$W$$>$$\lceil$$N/2$$\rceil$ ensures consistency in the presence of
concurrent writes.

\begin{figure}
\centering
\includegraphics[width=.85\columnwidth]{figs/dynamo-quorum.pdf}
\vspace{-8pt}
\caption{Diagram of control flow for client write to Dynamo-style
  quorum ($N=3$, $W=2$).  A coordinator node handles the client write
  and sends it to all $N$ replicas. The write call returns after the
  coordinator receives $W$ acknowledgments.}
\vspace{-12pt}
\label{fig:dynamo-quorum}
\end{figure}

There are significant differences between quorum theory and data
systems used in practice.  First, replication factors for data stores
are low, typically between one and
three~\cite{cassandradefault, feinbergpc, codapc}.  Second, (in the
absence of failure), in Dynamo-style partial quorums, the write quorum
size increases even after the operation returns, growing via
anti-entropy~\cite{antientropy}.  Coordinators send all requests to
all replicas but consider only the first $R$ ($W$) responses.  As a
matter of nomenclature (and to disambiguate against ``dynamic'' quorum
membership protocols), we will refer to these systems as
\textit{expanding partial quorum systems}. (We discuss additional
anti-entropy in Section~\ref{sec:anti-entropy}.) Third, as in much of
the applied literature, practitioners focus on fail-stop instead of
Byzantine failure modes~\cite{birman-byzantine}.  Following standard
practice, we do not consider Byzantine failure.

%\vspace{2em}

\vspace{\subsectionskip}\subsection{Typical Quorum Configurations}

For improved latency, operators often set $R+W \leq N$.  Here, we
survey quorum configurations according to practitioner accounts.
Operators frequently use partial quorum configurations, citing
performance benefits and high availability. Most of these accounts did
not discuss the possibility or occurrence of staleness resulting from
partial quorum configurations.

Cassandra defaults to $N$$=$$3$,
$R$$=$$W$$=$$1$~\cite{cassandradefault}. The Apache Cassandra 1.0
documentation claims that ``a majority of users do writes at
consistency level [$W$$=$$1$]'', while the Cassandra Query Language
defaults to $R$$=$$W$$=$$1$ as well~\cite{cassandra-docs}.  Production
Cassandra users report using $R$$=$$W$$=$$1$ in the ``general case''
because it provides ``maximum performance''~\cite{maxperfblog}, which
appears to be a commonly held belief~\cite{reddit, outbrain}.
Cassandra has a ``minor'' patch~\cite{cassandra-session} for session
guarantees~\cite{sessionguarantees} that is not currently
used~\cite{cassandra-session-revert}; according to our
discussions with developers, this is due to lack of interest.

Riak defaults to $N$$=$$3$, $R$$=$$W$$=$$2$~\cite{riakdefault-n,
  riakdefault-rw}. Users suggest using $R$$=$$W$$=$$1$, $N$$=$$2$ for
``low value'' data (and strict quorum variants for ``web,''
``mission critical,'' and ``financial'' data)~\cite{riaktalkone,
  riaktalktwo}.

 Finally, Voldemort does not provide sample configurations, but
 Voldemort's authors (and operators) at LinkedIn~\cite{feinbergpc}
 often choose $N$$=$$c$, $R$$=$$W$$=$$ \lceil c/2 \rceil$ for odd $c$.
 For applications requiring ``very low latency and high
 availability,'' LinkedIn deploys Voldemort with $N$$=$$3$,
 $R$$=$$W$$=$$1$.  For other applications, LinkedIn deployments
 Voldemort with $N$$=$$2$, $R$$=$$W$$=$$1$, providing ``some
 consistency,'' particularly when three-way replication is not
 required.  Unlike Dynamo, Voldemort sends read requests to $R$ of $N$
 replicas (not $N$ of $N$)~\cite{voldemortpub}; this decreases load
 per replica and network traffic at the expense of read latency and
 potential availability.  Provided staleness probabilities are
 independent across requests, this does not affect staleness: even
 when sending reads to $N$ replicas, coordinators only wait for $R$
 responses.

\vspace{\sectionskip}\section{Probabilistically Bounded\\Staleness}
\label{sec:pbs}

In this section, we introduce Probabilistically Bounded Staleness,
which describes the consistency provided by existing eventually
consistent data stores.  We present PBS $k$-staleness, which
probabilistically bounds the staleness of versions returned by read
quorums, PBS $t$-visibility, which probabilistically bounds the time
before a committed version appears to readers, and PBS $\langle k, t
\rangle$-staleness, a combination of the two prior models.

We introduce $k$-staleness first because it is self-contained, with a
simple closed-form solution.  In comparison, $t$-visibility is more
difficult, involving additional variables.  Accordingly, this
section proceeds in order of increasing difficulty, and the remainder
of the paper addresses the complexities of $t$-visibility.

Practical concerns guide the following theoretical contributions.  We
begin by considering a model without quorum expansion or other
anti-entropy.  For the purposes of a running example, as in
Equation~\ref{eq:prob-strict}, we assume that $W$ ($R$) of $N$
replicas are randomly selected for each write (read) operation.
Similarly, we consider fixed $W$, $R$ and $N$ across multiple
operations. Next, we expand our model to consider write propagation
and time-varying $W$ sizes in expanding partial quorums.  In this
section, we discuss anti-entropy in general, however we model
Dynamo-style quorums in Section \ref{sec:dynamo}. We discuss further
refinements to these assumptions in Section \ref{sec:discussion}.

\vspace{\subsectionskip}\subsection{PBS $k$-staleness}
\label{sec:kstale}

Probabilistic quorums allow us to determine the probability of
returning the most recent value written to the database, but do not
describe what happens when the most recent value is not returned.
Here, we determine the probability of returning a value within a
bounded number of versions.  In the following formulation, we consider
traditional, non-expanding write quorums (no anti-entropy):
\begin{definition}
A quorum system obeys \textit{PBS $k$-staleness consistency} if, with
probability $1-p_{sk}$, at least one value in any read quorum has been
committed within $k$ versions of the latest committed version when the
read begins.
\end{definition}
Reads may return versions whose writes that are not yet committed
(in-flight) (see Figure \ref{fig:timelines}A).  The $k$-quorum
literature defines these as $k$-regular semantics~\cite{non-strict}.

\begin{figure}
\centering
\includegraphics[width=.95\columnwidth]{figs/timelines.pdf}
\vspace{-8pt}
\caption{Versions returnable by read operations under PBS
  $k$-staleness (A) and PBS monotonic reads (B). In $k$-staleness, the
  read operation will return a version no later than $k$ versions
  older than the last committed value when it started.  In monotonic
  reads consistency, acceptable staleness depends on the number of
  versions committed since the client's last read.}
\vspace{-12pt}
\label{fig:timelines}
\end{figure}

The probability of returning a version of a key within the last $k$
versions committed is equivalent to intersecting one of $k$
independent write quorums.  Given the probability of a single quorum
non-intersection $p$, the probability of non-intersection with one of
the last $k$ independent quorums is $p^k$.  In our running example, the probability of non-intersection is Equation
\ref{eq:prob-strict} exponentiated by $k$:
\begin{equation}
\label{eq:k-consistency}
p_{sk} = \left(\frac{{N-W \choose R}}{{N \choose R}}\right)^k
\end{equation}

When $N$$=$$3$, $R$$=$$W$$=$$1$, this means that the probability of
returning a version within $2$ versions is $0.\overline{5}$, within $3$
versions, $0.\overline{703}$, $5$ versions, $> 0.868$, and $10$
versions, $>0.98$.  When $N$$=$$3$, $R$$=$$1$, $W$$=$$2$ (or, equivalently,
$R$$=$$2$, $W$$=$$1$), these probabilities increase: $k$$=$$1
\rightarrow 0.\overline{6}$, $k$$=$$2 \rightarrow 0.\overline{8}$, and
$k$$=$$5 \rightarrow > 0.995$.

This closed form solution holds for quorums that do not change size
over time.  For expanding partial quorum systems, this solution is an
upper bound on the probability of staleness.

\vspace{1em}

\vspace{\subsectionskip}\subsection{PBS Monotonic Reads}

PBS $k$-staleness can predict whether a client will ever read older
data than it has previously read, a well-known session guarantee
called \textit{monotonic reads} consistency~\cite{sessionguarantees}.
This is particularly useful when clients do not need to see the most
recent version of a data item but still require a notion of ``forward
progress'' through versions, as in timelines or streaming changelogs.

\begin{definition}
\label{def:prob-mr}
A quorum system obeys \textit{PBS monotonic reads consistency} if,
with probability at least $1-p_{sMR}$, at least one value in any
read quorum returned to a client is the same version or a newer
version than the last version that the client previously read.
\end{definition}

To guarantee that a client sees monotonically increasing versions, it
can continue to contact the same replica~\cite{vogels-defs} (provided
the ``sticky'' replica does not fail).  However, this is insufficient
for strict monotonic reads (where the client reads strictly newer data
if it exists in the system).  We can adapt
Definition~\ref{def:prob-mr} to accommodate strict monotonic reads by
requiring that the data store returns a more recent data version if it exists.

PBS monotonic reads consistency is a special case of PBS $k$-staleness
(see Figure~\ref{fig:timelines}B), where $k$ is determined by a
client's rate of reads from a data item ($\gamma_{cr}$) and the
global, system-wide rate of writes to the same data item
($\gamma_{gw}$).  If we know these rates, the number of
versions written between client reads is
$\frac{\gamma_{gw}}{\gamma_{cr}}$, as shown in Figure
\ref{fig:timelines}B.  We can calculate the probability of
probabilistic monotonic reads as a special case of $k$-staleness where
$k=1+\frac{\gamma_{gw}}{\gamma_{cr}}$.  Again extending our
running example, from Equation \ref{eq:k-consistency}:
\begin{equation}
\label{eq:prob-mr}
p_{sMR} = \left(\frac{{N-W \choose R}}{{N \choose R}}\right)^{1+\gamma_{gw}/\gamma_{cr}}
\end{equation}
For strict monotonic reads, where we cannot read the version we have
previously read (assuming there are newer versions in the database), we
exponentiate with $k=\frac{\gamma_{gw}}{\gamma_{cr}}$.

In practice, we may not know these exact rates, but, by measuring
their distribution, we can calculate an expected value.  By performing
appropriate admission control, operators can control these rates to
achieve monotonic reads consistency with high probability.

\subsection{Load Improvements}

Theory literature defines the \textit{load} of a quorum system as a
metric for the frequency of accessing the busiest quorum
member~\cite[Definition 3.2]{quorumsystems}.  Intuitively, the busiest
quorum member limits the number of requests that a given quorum system
can sustain, called its \textit{capacity}~\cite[Corollary
  3.9]{quorumsystems}.

Prior work determined that probabilistic quorum systems did not offer
significant benefits to load (providing a constant factor improvement
compared to strict quorum systems)~\cite{prob-quorum}.  Here, we show
that quorums tolerating PBS $k$-staleness have asymptotically lower
load than traditional probabilistic quorum systems (and, transitively,
than strict quorum systems).

The probabilistic quorum literature defines an
$\varepsilon$-intersecting quorum system as a quorum system that
provides a $1-\varepsilon$ probability of returning consistent
data~\cite[Definition 3.1]{prob-quorum}.  A $\varepsilon$-intersecting
quorum system has load of at least 
$\frac{1-\sqrt{\varepsilon}}{\sqrt{N}}$~\cite[Corollary
  3.12]{prob-quorum}.

In considering $k$ versions of staleness, we consider the intersection
of $k$ $\varepsilon$-intersecting quorum systems.  For a given
probability $p$ of inconsistency, if we are willing to tolerate $k$
versions of staleness, we need only require that $\varepsilon =
\sqrt[k]{p}$.  This implies that our PBS $k$-staleness system
construction has load of at least
$\frac{(1-p)^{\frac{1}{2k}}}{\sqrt{N}}$, an improved lower bound
compared to traditional probabilistic quorum systems.  PBS monotonic
reads consistency results in a lower bound on load of
$\frac{(1-p)^{\frac{1}{2C}}}{\sqrt{N}}$, where
$C=1+\frac{\gamma_{gw}}{\gamma_{cr}}$.

These results are intuitive: if we are willing to tolerate multiple
versions of staleness, we need to contact fewer replicas.  Staleness
tolerance lowers the load of a quorum system, subsequently increasing
its capacity.

\vspace{\subsectionskip}\subsection{PBS $t$-visibility}
\label{sec:tvis}

Until now, we have considered only quorums that do not grow over time.
However, as we discussed in Section \ref{sec:practice}, real-world
quorum systems expand by asynchronously propagating writes to quorum
system members over time.  This process is commonly known as
anti-entropy~\cite{antientropy}.  For generality, in this section, we
will discuss generic anti-entropy. However, we explicitly model the
Dynamo-style anti-entropy mechanisms in Section \ref{sec:dynamo}.

PBS $t$-visibility models the probability of inconsistency for
expanding quorums.  $t$-visibility is the probability that a read
operation, starting $t$ seconds after a write commits, will observe
the latest value of a data item. This $t$ captures the expected length
of the ``window of inconsistency.''  Recall that we consider in-flight
writes---which are more recent than the last committed version---as
non-stale.

\begin{definition}
A quorum system obeys \textit{PBS $t$-visibility consistency} if, with
probability $1-p_{st}$, any read quorum started at least $t$ units
of time after a write commits returns at least one value
that is at least as recent as that write.
\end{definition}

Overwriting data items effectively resets $t$-visibility; the time
between writes bounds $t$-visibility. If we space two writes to a key
$m$ milliseconds apart, then the $t$-visibility of the first write for
$t > m$ milliseconds is undefined; after $m$ milliseconds, there will
be a newer version.

We denote the cumulative density function describing the number of
replicas $\mathcal{W}_r$ that have received a particular version $v$
exactly $t$ seconds after $v$ commits as $P_w(\mathcal{W}_r, t)$.

By definition, for expanding quorums, $\forall c \in [0, W], P_w(c,0)
= 1$; at commit time, $W$ replicas will have received the value with
certainty.  We can model the probability of PBS $t$-visibility for given $t$ by summing the conditional probabilities of each possible
$\mathcal{W}_r$:
\begin{equation}
\label{eq:tv-instantreads}
p_{st} = \frac{{N-W \choose N}}{{N \choose R}}+\sum_{c\in(W, N]} \frac{{N-c \choose N}}{{N \choose R}}\cdot [P_w(c+1, t)-P_w(c,t)]
\end{equation}
However, the above equation assumes reads occur instantaneously and
writes commit immediately after $W$ replicas have the version (i.e.,
there is no delay acknowledging the write to the coordinating node).
In the real world, coordinators wait for write acknowledgments and
read requests take time to arrive at remote replicas, increasing $t$.
Accordingly, Equation~\ref{eq:tv-instantreads} is a conservative upper
bound on $p_{st}$.

In practice, $P_w$ depends on the anti-entropy mechanisms in use and
the expected latency of operations but we can approximate it (Section
\ref{sec:dynamo}) or measure it online.  For this reason, the load of
a PBS $t$-visible quorum system depends on write propagation and is
difficult to analytically determine for general-purpose expanding
quorums.  Additionally, one can model both transient and permanent
failures by increasing the tail probabilities of $P_w$
(Section~\ref{sec:discussion}).

\vspace{3em}

\vspace{\subsectionskip}\subsection{PBS $\langle k, t \rangle$-staleness}

We can combine the previous models to combine both versioned and
real-time staleness metrics to determine the probability that a read
will return a value no older than $k$ versions stale if the last write
committed at least $t$ seconds ago:
\begin{definition}
A quorum system obeys \textit{PBS $\langle k, t \rangle$-staleness
  consistency} if, with probability $1-p_{skt}$, at least one value in
any read quorum will be within $k$ versions of the latest committed
version when the read begins, provided the read begins $t$ units of
time after the previous $k$ versions commit.
\end{definition}
The definition of $p_{skt}$ follows from the prior definitions:
\begin{equation}
p_{skt} = (\frac{{N-W \choose R}}{{N \choose R}}+\sum_{c\in[W, N)} \frac{{N-c \choose R}}{{N \choose R}} \cdot [P_w(c+1, t)-P_w(c,t)])^k
\end{equation}
In this equation, in addition to (again) assuming instantaneous reads,
we also assume the pathological case where the last $k$ writes all
occurred at the same time.  If we can determine the time since commit
for the last $k$ writes, we can improve this bound by considering each
quorum's $p_{skt}$ separately (individual $t$).  However, predicting
(and enforcing) write arrival rates is challenging and may introduce
inaccuracy, so this equation is a conservative upper bound on
$p_{skt}$.

Note that PBS $\langle k, t \rangle$-staleness consistency
encapsulates the prior definitions of consistency. Probabilistic
$k$-quorum consistency is simply PBS $\langle k, 0 \rangle$-staleness
consistency, PBS monotonic reads consistency is $\langle
1+\frac{\gamma_{gw}}{\gamma_{cr}}, 0 \rangle$-staleness consistency,
and PBS $t$-visibility is $\langle 1, t \rangle$-staleness
consistency.

In practice, we believe it is easier to reason about staleness of
versions or staleness of time but not both together.  Accordingly,
having derived a closed-form model for $k$-staleness, in the remainder
of this paper, we focus mainly on deriving more specific models for
$t$-visibility. A conservative rule-of-thumb going forward is to
exponentiate the probability of inconsistency in $t$-visibility by $k$
when up to $k$ versions of staleness are tolerable.

\vspace{\sectionskip}\section{Dynamo-style $t$-visibility}
\label{sec:dynamo}

We have a closed-form model for $k$-staleness, but
$t$-visibility is dependent on both the quorum replication algorithm
and the anti-entropy processes employed by a given system.  In this
section, we discuss PBS $t$-visibility in the context of Dynamo-style
data stores and describe how to asynchronously detect
staleness.

\vspace{\subsectionskip}\subsection{Inconsistency in Dynamo: {\large \textit{WARS}} Model}
\label{sec:wars}

Dynamo-style quorum systems are inconsistent as a result of read and
write message reordering, a product of message delays.  To illustrate
this phenomenon, we introduce a model of message latency in Dynamo
operation which, for convenience, we call \textit{WARS}.

In Figure~\ref{fig:dynamo-diagram}, we illustrate \textit{WARS} using
a space-time diagram for messages between a coordinator and a single
replica for a write followed by a read $t$ seconds after the write
commits.  This $t$ corresponds to the $t$ in PBS $t$-visibility. In
brief, reads are stale when all of the first $R$ responses to the read
request arrived at their replicas before the last
(committed) write request.

\begin{figure}
\centering
\includegraphics[width=.8\columnwidth]{figs/dynamostale.pdf}
\vspace{-8pt}
\caption{The \textit{WARS} model for in Dynamo describes the message
  latencies between a coordinator and a single replica for a write
  followed by a read $t$ seconds after commit.  In an $N$ replica
  system, this messaging occurs $N$ times.}
\vspace{-12pt}
\label{fig:dynamo-diagram}
\end{figure}

For a write, the coordinator sends $N$ messages, one to each
replica. The message from the coordinator to replica containing the
write is delayed by a value drawn from distribution \texttt{W}.  The
coordinator waits for $W$ responses from the replicas before it can
consider the version committed.  Each response acknowledging the write
is delayed by a value drawn from the distribution \texttt{A}.

For a read, the coordinator (possibly different than the write
coordinator) sends $N$ messages, one to each replica.  The message
from coordinator to replica containing the read request is delayed by
a value drawn from distribution \texttt{R}.  The coordinator waits
for $R$ responses from the replicas before returning the most recent
value it receives.  The read response from each replica is delayed by
a value drawn from the distribution \texttt{S}.

The read coordinator will return stale data if the first $R$ responses
received reached their replicas before the replicas
received the latest version (delayed by \texttt{W}).  When
$R$$+$$W$$>$$N$, this is impossible.  However, under partial quorums,
the frequency of this occurrence depends on the latency distributions.
If we denote the commit time (when the coordinator has received $W$
acknowledgments) as $w_t$, a single replica's response is stale if
$r'+w_t+t< w'$ for $r'$ drawn from \texttt{R} and $w'$ drawn from
\texttt{W}.  Writes have time to propagate to additional replicas both
while the coordinator waits for all required acknowledgments
(\texttt{A}) and as replicas wait for read requests (\texttt{R}).  Read
responses are further delayed in transit (\texttt{S}) back to the read
coordinator, inducing further possibility of reordering.
Qualitatively, longer write tails (\texttt{W}) and faster reads
increase the chance of staleness due to reordering.

\textit{WARS} considers the effect of message sending, delays, and
reception, but this represents a daunting analytical formulation.  The
commit time is an order statistic of $W$ and $N$ dependent on both
\texttt{W} and \texttt{A}.  Furthermore, the probability that the
$i$th returned read message observes reordering is another order
statistic of $R$ and $N$ dependent on
\texttt{W},\texttt{A},\texttt{R}, and \texttt{S}.  Moreover, across
responses, the probabilities are dependent. These dependencies make
calculating the probability of staleness rather difficult.  Dynamo is
straightforward to reason about and program but is difficult to
analyze in a simple closed form.  As we discuss in
Section~\ref{sec:mcsim}, we instead explore \textit{WARS} using Monte
Carlo methods, which are straightforward to understand and implement.

\vspace{\subsectionskip}\subsection{{\large \textit{WARS}} Scope}
\label{sec:anti-entropy}

\textbf{Proxying operations.} Depending on which coordinator a client
contacts, coordinators may serve reads and writes locally.  In this
case, subject to local query processing delays, a read or write to $R$
or $W$ nodes behaves like a read or write to $R-1$ or $W-1$ nodes.
Although we do not do so, one could adopt \textit{WARS} to handle local
reads and writes.  The decision to proxy requests (and, if not, which
replicas serve which requests) is data store and deployment-specific.
Dynamo forwards write requests to a designated coordinator solely for
the purpose of establishing a version ordering~\cite[Section
  6.4]{dynamo} (easily achievable through other
mechanisms~\cite{zookeeper}).  Dynamo's authors observed a latency
improvement by proxying all operations and having clients act as
coordinators---Voldemort adopts this
architecture~\cite{voldemortclient}.

\textbf{Client-side delays.} End-users will likely incur additional time
between their reads and writes due to latency required to
contact the service.  Individuals making requests to web services
through their browsers will likely space sequential requests by tens
or hundreds of milliseconds due to client-to-server latency.  Although
we do not consider this delay here, it is important to remember for
practical scenarios because the delay between reads and writes ($t$)
may be large.

\textbf{Additional anti-entropy.} As we discussed in
Section~\ref{sec:practice}, anti-entropy decreases the probability of
staleness by propagating writes between replicas.  Dynamo-style
systems also support additional anti-entropy processes~\cite{nosql}.
\textit{Read repair} is a commonly used process: when a read coordinator
receives multiple versions of a data item from different replicas in
response to a read request, it will attempt to (asynchronously) update
the out-of-date replicas with the most recent version~\cite[Section
  5]{dynamo}.  Read repair acts like an additional write for every
read, except old values are re-written.  Additionally, Dynamo used
Merkle trees to summarize and exchange data contents between
replicas~\cite[Section 4.7]{dynamo}.  However, not all Dynamo-style
data stores actively employ similar gossip-based anti-entropy.  For
example, Cassandra uses Merkle tree anti-entropy only when manually
requested (e.g., \texttt{nodetool repair}), instead relying primarily
on quorum expansion and read repair~\cite{cassandra-merkle}.

These processes are rate-dependent: read repair's efficiency depends
on the rate of reads, and Merkle tree exchange's efficiency (and, more
generally, most anti-entropy efficiency) depends on the rate of
exchange.  A conservative assumption for read repair and Merkle tree
exchange is that they never occur. For example, assuming a particular
read repair rate implies a given rate of reads from each key in the
system.

In contrast, \textit{WARS} captures expanding quorum behavior
independent of read rate and with conservative write rate
assumptions. \textit{WARS} considers a single read and a single
write. Aside from load considerations, concurrent reads do not affect
staleness. If multiple writes overlap (that is, have overlapping
periods where they are in-flight but are not committed) the
probability of inconsistency decreases.  This is because overlapping
writes result in an increased chance that a client reads
as-yet-uncommitted data.  As a result, with \textit{WARS}, data may be
fresher than predicted.

\vspace{\subsectionskip}\subsection{Asynchronous Staleness Detection}

Even if a system provides a low probability of inconsistency,
applications may need notification when data returned is
inconsistent or staler than expected.  Here, as a side note, we
discuss how the Dynamo protocol is naturally equipped for staleness
detection.  We focus on PBS $t$-visibility in the following discussion
but it is easily extended to PBS $k$-staleness and $\langle k, t
\rangle$-staleness.

Knowing whether a response is stale at read time requires strong
consistency.  Intuitively, by checking all possible values in the domain against a
hypothetical staleness detector, we could determine the (strongly) consistent
value to return.  While we cannot do so synchronously, we \textit{can}
determine staleness asynchronously.  Asynchronous staleness detection
allows speculative execution~\cite{nsdispeculation} if a program
contains appropriate compensation logic.

We first consider a staleness detector providing false positives.
Recall that, in a Dynamo-style system, we wait for $R$ of $N$ replies
before returning a value.  The remaining $N-R$ replicas will still
reply to the read coordinator.  Instead of dropping these messages,
the coordinator can compare them to the version it returned.  If there
is a mismatch, then either the coordinator returned stale data, there
are in-flight writes in the system, or additional versions committed
after the read. The latter two cases, relating to data committed after
the response initiation, lead to false positives.  In these cases,
the read did not return ``stale'' data even though there were newer
but uncommitted versions in the system.  Notifying clients about newer
but uncommitted versions of a data item is not necessarily bad but may
be unnecessary and violates our staleness semantics.  This detector
does not require modifications to the Dynamo protocol and is similar
to the read-repair process.

To eliminate these uncommitted-but-newer false positives (cases two
and three), we need to determine the total, system-wide commit
ordering of writes. Recall that replicas are unaware of the commit
time for each version. The timestamps stored by replicas are not
updated after commit, and commits occur after $W$ replicas
respond. Thankfully, establishing a total ordering among distributed
agents is a well-known problem that a Dynamo-style system can solve by
using a centralized service~\cite{zookeeper} or using distributed
consensus~\cite{paxos}. This requires modifications but is feasible.



\vspace{\sectionskip}\section{Evaluating Dynamo {\large $t$}-visibility}
\label{sec:dynamoeval}

As discussed in Section~\ref{sec:tvis}, PBS $t$-visibility depends on
the propagation of reads and writes throughout a system.  We
introduced the \textit{WARS} model as a means of reasoning about
inconsistency in Dynamo-style quorum systems, but quantitative metrics
such as staleness observed in practice depend on each of
\textit{WARS}'s latency distributions.  In this section, we perform an
analysis of Dynamo-style $t$-visibility to better understand how
frequently ``eventually consistent'' means ``consistent'' and, more
importantly, why.

PBS $k$-staleness is easily captured in closed form
(Section~\ref{sec:kstale}).  It does not depend on write latency or
any environmental variables.  Indeed, in practice, without expanding
quorums or anti-entropy, we observe that our derived equations hold
true experimentally.

$t$-visibility depends on anti-entropy, which is more
complicated.  In this section, we focus on deriving experimental
expectations for PBS $t$-visibility.  While we could improve the
staleness results by considering additional anti-entropy processes
(Section~\ref{sec:anti-entropy}), we make the bare minimum of
assumptions required by the \textit{WARS} model.  Conservative
analysis decreases the number of experimental variables (supported by
empirical observations from practitioners) and increases the
applicability of our results.

\vspace{\subsectionskip}\subsection{Monte Carlo Simulation}
\label{sec:mcsim}

In light of the complicated analytical formulation discussed in
Section~\ref{sec:wars}, we implemented \textit{WARS} in an
event-driven simulator for use in Monte Carlo methods.  Calculating
$t$-visibility for a given value of $t$ is straightforward. Denoting the
$i$th sample drawn from distribution \texttt{D} as $\texttt{D}[i]$:
draw $N$ samples from \texttt{W}, \texttt{A}, \texttt{R}, and
\texttt{S} at time $t$, compute $w_t$, the $W$th smallest value of $\{\texttt{W}[i]+\texttt{A}[i], i \in [0, N)\}$, and check whether the first
$R$ samples of \texttt{R}, ordered by $\texttt{R}[i]+\texttt{S}[i]$
obey $w_t+\texttt{R}[i] + t\leq \texttt{W}[i]$.  This requires only a
few lines of code.  Extending this formulation to analyze $\langle k,
t \rangle$-staleness given a distribution of write arrival times
requires accounting for multiple writes across time but is not
difficult.

\subsection{Experimental Validation}

To validate \textit{WARS}, our simulator, and our subsequent analyses,
we compared our predicted $t$-visibility and latency with measured
values observed in a commercially available, open source Dynamo-style
data store.  We modified Cassandra to profile \textit{WARS} latencies,
disabled read repair (as it is external to \textit{WARS}), and, for
reads, only considered the first $R$ responses (often, more than $R$
messages would arrive by the processing stage, decreasing staleness).
We ran Cassandra on three servers with 2.2GHz AMD Opteron 2214
dual-core SMT processors and 4GB of 667MHz DDR2 memory, serving
in-memory data.  To measure staleness, we inserted increasing versions of a key while concurrently issuing read requests.

Our observations matched the \textit{WARS} predictions. We injected
each combination of exponentially distributed $\texttt{W}=\lambda \in
\{0.05,$ $0.1,$ $0.2\}$ (means $20$ms, $10$ms and $5$ms)
and $\texttt{A}$$=$$\texttt{R}$$=$$\texttt{S}=\lambda \in \{0.1,$
$0.2,$ $0.5\}$ (means $10$ms, $5$ms and $2$ms) across
50,000 writes.  After empirically measuring the \textit{WARS}
distributions, consistency, and latency for each partial quorum
configuration, we predicted the $t$-visibility and latency. Our
average $t$-visibility prediction RMSE was $0.28\%$
(std. dev. $0.05\%$, max. $0.53\%$) for each
$t\in$$\{1,$$\dots,$$199\}$ ms. Our predicted latency (for each of the
$\{1.0, \dots, 99.9$th$\}$ percentiles for each configuration) had an
average N-RMSE of $0.48\%$ (std. dev. $0.18\%$, max. $0.90\%$).  This
validates our Monte Carlo simulator.


\vspace{\subsectionskip}\subsection{Write Latency Distribution Effects}
\label{sec:synthetic}

As discussed in Section~\ref{sec:wars}, the \textit{WARS} model of
Dynamo-style systems dictates that high one-way write variance
(\texttt{W}) increases staleness.  To quantify these effects, we swept
a range of exponentially distributed write distributions (changing
parameter $\lambda$, which dictates the mean and tail of the
distribution) while fixing \texttt{A}=\texttt{R}=\texttt{S}.

Our results, shown in Figure~\ref{fig:varydelay}, confirm this
relationship.  When the variance of \texttt{W} is $0.0625$ms
($\lambda=4$, mean $.25$ms, one-fourth the mean of
\texttt{A}=\texttt{R}=\texttt{S}), we observe a $94\%$ chance of
consistency immediately after the write and $99.9\%$ chance after 1ms.
However, when the variance of \texttt{W} is $100$ms ($\lambda=.1$,
mean $10$ms, ten times the mean of \texttt{A}=\texttt{R}=\texttt{S}),
we observe a $41\%$ chance of consistency immediately after write and
a $99.9\%$ chance of consistency only after $65$ms.  As the variance
and mean increase, so does the probability of inconsistency.  Under
distributions with fixed means and variable variances (uniform,
normal), we observe that the mean of \texttt{W} is less important than
its variance if \texttt{W} is strictly greater than
\texttt{A}=\texttt{R}=\texttt{S}.

Decreasing the mean and variance of \texttt{W} improves the
probability of consistent reads.  This means that, as we will see,
techniques that lower one-way write latency result in lower
$t$-visibility.  Instead of increasing read and write quorum sizes,
operators could chose to lower (relative) \texttt{W} latencies through
hardware configuration or by delaying reads.  This latter option is
potentially detrimental to performance for read-dominated workloads and
may introduce undesirable queuing effects.

\begin{figure}
\centering
\includegraphics[width=.85\columnwidth]{figs/rwratio.pdf}
\vspace{-14pt}
\caption{$t$-visibility with
  exponential latency distributions for 
  \texttt{W} and \texttt{A}=\texttt{R}=\texttt{S}. Mean latency is
  $1/\lambda$. $N$$=$$3$, $R$$=$$W$$=$$1$. }
\vspace{-12pt}
\label{fig:varydelay}
\end{figure}

\vspace{\subsectionskip}\subsection{Production Latency Distributions}
\label{sec:latencies}

To study \textit{WARS} in greater detail, we obtained production latency
statistics from two internet-scale companies.

LinkedIn\footnote{LinkedIn. \url{www.linkedin.com}} is an online
professional social network with over 135 million members as of
November 2011. To provide highly available, low latency data storage,
engineers at LinkedIn built Voldemort.  Alex Feinberg, a lead engineer
on Voldemort, graciously provided us with latency distributions for a
single node under peak traffic for a user-facing service at
LinkedIn, representing 60\% read and 40\% read-modify-write
traffic~\cite{feinbergpc} (Table~\ref{table:linkedin}).  Feinberg
reports that, using spinning disks, Voldemort is ``largely IO bound
and latency is largely determined by the kind of disks we're using,
[the] data to memory ratio and request distribution.''  With solid-
state drives (SSDs), Voldemort is ``CPU and/or network bound
(depending on value size).''  As an aside, Feinberg also
noted that ``maximum latency is generally determined by [garbage
  collection] activity (rare, but happens occasionally) and is within
hundreds of milliseconds.''

\begin{table}
\centering
\begin{tabular}{|c|c|}
\hline
\%ile & Latency (ms) \\
\hline
\multicolumn{2}{|c|}{ 15,000 RPM SAS Disk}\\
\hline
Average & 4.85\\
95 & 15\\
99 & 25\\
\hline
\multicolumn{2}{|c|}{ Commodity SSD }\\
\hline
Average & 0.58 \\
95 & 1\\
99 & 2\\
\hline
\end{tabular}
\vspace{-6pt}
\caption{LinkedIn Voldemort single-node production latencies.}
\vspace{-4pt}
\label{table:linkedin}
\end{table}

Yammer\footnote{Yammer. \url{www.yammer.com}} provides private
social networking to over 100,000 companies as of December 2011 and
uses Basho's Riak for some client data~\cite{riak}.  Coda Hale, an
infrastructure architect, and Ryan Kennedy, also of Yammer, previously
presented in-depth performance and configuration details for their
Riak deployment in March 2011~\cite{riakyammer}.  Hale provided us
with more detailed performance statistics for their
application~\cite{codapc} (Table~\ref{table:yammer}).  Hale mentioned
that ``reads and writes have radically different expected latencies,
especially for Riak.''  Riak delays writes ``until the fsync returns,
so while reads are often $<$ 1ms, writes rarely are.''  Also, although
we do not model this explicitly, Hale also noted that the size of
values is important, claiming ``a big performance improvement by
adding LZF compression to values.''

\begin{table}
\centering
\begin{tabular}{|c|c|c|}
\hline
\%ile & Read Latency (ms) & Write Latency (ms)\\
\hline
Min & 1.55 & 1.68\\
50 & 3.75 & 5.73 \\
75 & 4.17 & 6.50\\
95 & 5.2 & 8.48\\
98 & 6.045 & 10.36 \\
99 & 6.59 & 131.73\\
99.9 & 32.89 & 435.83\\
Max & 2979.85 &  4465.28 \\
\hline
Mean & 9.23 & 8.62 \\
Std. Dev. & 83.93 & 26.10\\
\hline
Mean Rate & 718.18 gets/s & 45.65 puts/s\\
\hline
\end{tabular}
\vspace{-4pt}
\caption{Yammer Riak $N$$=$$3$, $R$$=$$2$, $W$$=$$2$ production latencies.}
\vspace{-12pt}
\label{table:yammer}
\end{table}

\subsection{Latency Model Fitting}

While the provided production latency distributions are invaluable,
they are under-specified for \textit{WARS}.  First, the data are
summary statistics, but \textit{WARS} requires distributions.  More
importantly, the provided latencies are round-trip times, while
\textit{WARS} requires the constituent one-way latencies for both
reads and writes.  As our validation demonstrated, these latency
distributions are easily collected, but, because they are not
currently collected in production, we must fill in the
gaps. Accordingly, to fit \texttt{W}, \texttt{A}, \texttt{R}, and
\texttt{S} for each configuration, we made a series of assumptions.
Without additional data on the latency required to read multiple
replicas, we assume that each latency distribution is independently,
identically distributed (IID).  We fit each configuration using a
mixture model with two distributions, one for the body and the other
for the tail.

LinkedIn provided two latency distributions, whose fits we denote
\texttt{LNKD-SSD} and \texttt{LNKD-DISK} for the SSD and spinning disk
data.  As previously discussed, when running on SSDs, Voldemort is
network and CPU bound.  Accordingly, for \texttt{LNKD-SSD}, we assumed
that read and write operations took equivalent amounts of time and, to
allocate the remaining time, we focused on the network-bound case and
assumed that one-way messages were symmetric
(\texttt{W}=\texttt{A}=\texttt{R}=\texttt{S}). Feinberg reported that
Voldemort performs at least one read before every write (average of 1
seek, between 1-3 seeks), and writes to the BerkeleyDB Java Edition
backend flush to durable storage either every 30 seconds or 20
megabytes---whichever comes first~\cite{feinbergpc}.  Accordingly, for
\texttt{LNKD-DISK}, we used the same \texttt{A}=\texttt{R}=\texttt{S}
as \texttt{LNKD-SSD} but fit \texttt{W} separately.

Yammer provided distributions for a single configuration, denoted
\texttt{YMMR}, but separated read and write latencies.  Under our IID
assumptions, we fit single-node latency distributions to the provided
data, again assuming symmetric \texttt{A}, \texttt{R}, and \texttt{S}.
The data again fit a Pareto distribution with a long exponential tail.
At the $98$th percentile, the write distribution takes a sharp turn.
Fitting the data closely resulted in a long tail, with $99.99+$th
percentile writes requiring tens of seconds---much higher than Yammer
specified.  Accordingly, we fit the $98$th percentile knee
conservatively; without the $98$th percentile, the write fit N-RMSE is
.104\%.

We also considered a wide-area network replication scenario, denoted
\texttt{WAN}.  Reads and writes originate in a random datacenter, and,
accordingly, one replica command completes quickly and the coordinator
routes the others remotely.  We delay remote messages by 75ms and
apply \texttt{LNKD-DISK} delays once the command reaches a remote data
center, reflecting multi-continent WAN delay~\cite{dean-keynote}.

We show the parameters for each distribution in Table~\ref{table:fits}
and plot each fitted distribution in Figure~\ref{fig:latencies}.  Note
that for $R$, $W$ of one, \texttt{LNKD-DISK} is not equivalent to
\texttt{WAN}. In \texttt{LNKD-DISK}, we only have to wait for one of
$N$ local reads (writes) to return, whereas, in \texttt{WAN}, there is
only one local read (write) and the network delays all other read
(write) requests by at least 150ms.


\begin{table}
\centering
\begin{tabular}{|c|r|}
\hline
\multirow{4}{*}{\texttt{LNKD-SSD}} & \multicolumn{1}{|l|}{$\texttt{W} = \texttt{A}= \texttt{R} = \texttt{S}:$} \\
& 91.22\%: Pareto, $x_m=.235, \alpha=10$\\
& 8.78\%: Exponential, $\lambda = 1.66$ \\
& N-RMSE: .55\%\\\hline
\multirow{4}{*}{\texttt{LNKD-DISK}} & 
 \multicolumn{1}{|l|}{\texttt{W}:}\\
& 38\%: Pareto, $x_m=1.05, \alpha=1.51$\\
& \hfill 62\%: Exponential, $\lambda = .183$ \\
& N-RMSE: .26\%\\\cline{2-2}
& \multicolumn{1}{|l|}{$\texttt{A}= \texttt{R} = \texttt{S}: \texttt{LNKD-SSD}$}\\
\hline
\multirow{8}{*}{\texttt{YMMR}} & \multicolumn{1}{|l|}{\texttt{W}:} \\
& 93.9\%: Pareto, $x_m=3, \alpha=3.35$\\
& 6.1\%: Exponential, $\lambda = .0028$ \\
& N-RMSE: 1.84\%\\\cline{2-2}
& \multicolumn{1}{|l|}{$\texttt{A}= \texttt{R} = \texttt{S}:$}\\
& 98.2\%: Pareto, $x_m=1.5, \alpha=3.8$\\
& 1.8\%: Exponential, $\lambda=.0217$\\
& N-RMSE: .06\%\\
\hline
\end{tabular}
\vspace{-6pt}
\caption{Distribution fits for production latency distributions from LinkedIn (\texttt{LNKD-*}) and Yammer (\texttt{YMMR}).}
\vspace{-12pt}
\label{table:fits}
\end{table}



\begin{figure*}[t!]
\centering
\subfigure{\includegraphics[width=\columnwidth]{figs/latlegend.pdf}}\\[-1mm]
\subfigure{\includegraphics[width=.6\columnwidth]{figs/readlats-1.pdf}}
\subfigure{\includegraphics[width=.6\columnwidth]{figs/readlats-2.pdf}}
\subfigure{\includegraphics[width=.6\columnwidth]{figs/readlats-3.pdf}}
\subfigure{\includegraphics[width=.6\columnwidth]{figs/writelats-1.pdf}}
\subfigure{\includegraphics[width=.6\columnwidth]{figs/writelats-2.pdf}}
\subfigure{\includegraphics[width=.6\columnwidth]{figs/writelats-3.pdf}}
\vspace{-8pt}
\caption{Read and write operation latency for production fits for $N$$=$$3$. For reads, \texttt{LNKD-SSD} is equivalent to \texttt{LNKD-DISK}.}
\vspace{-8pt}
\label{fig:latencies}
\end{figure*}

\vspace{\subsectionskip}\subsection{Observed {\large$t$}-visibility}

We measured the $t$-visibility for each distribution
(Figure~\ref{fig:tvis}). As we observed under synthetic distributions
in Section~\ref{sec:synthetic}, the $t$-visibility depended on both
the relative mean and variance of \texttt{W}.

\texttt{LNKD-SSD} and \texttt{LNKD-DISK} demonstrate the importance of
write latency in practice.  Immediately after write commit,
\texttt{LNKD-SSD} had a $97.4\%$ probability of consistent reads,
reaching over a $99.999\%$ probability of consistent reads after five
milliseconds. \texttt{LNKD-SSD}'s reads briefly raced its writes
immediately after commit.  However, within a few milliseconds after
the write, the chance of a read arriving before the last write was
nearly eliminated. The distribution's read and write operation
latencies were small (median $.489$ms), and writes completed quickly
across all replicas due to the distribution's short tail ($99.9$th
percentile $.657$ms).  In contrast, under \texttt{LNKD-DISK}, writes
take much longer (median $1.50$ms) and have a longer tail ($99.9$th
percentile $10.47$ ms).  \texttt{LNKD-DISK}'s $t$-visibility reflects
this defference: immediately after write commit, \texttt{LNKD-DISK}
had only a $43.9\%$ probability of consistent reads and, ten ms later,
only a $92.5\%$ probability.  This suggests that SSDs may greatly
improve consistency due to reduced write variance.

We experienced similar effects with the other distributions.
Immediately after commit, \texttt{YMMR} had a $89.3\%$ chance of
consistency.  However, \texttt{YMMR}'s long tail hampered its
$t$-visibility increase and reached a $99.9\%$ probability of
consistency 1364 ms after commit.  As expected, \texttt{WAN} observed
poor chances of consistency until after the 75 milliseconds passed
($33\%$ chance immediately after commit); the client had to wait longer to
observe the most recent write unless it originated from the reading
client's data.

\begin{figure*}[t!]
\centering
\subfigure{\includegraphics[width=.85\columnwidth]{figs/stalelegend.pdf}}\\[-1mm]
\subfigure{\includegraphics[width=3.2mm]{figs/staley.pdf}}
\subfigure{\includegraphics[width=.48\columnwidth]{figs/tstales-LNKD-SSD.pdf}}
\subfigure{\includegraphics[width=.48\columnwidth]{figs/tstales-LNKD-DISK.pdf}}
\subfigure{\includegraphics[width=.48\columnwidth]{figs/tstales-WAN.pdf}}
\subfigure{\includegraphics[width=.48\columnwidth]{figs/tstales-YMMR.pdf}}\\[-1mm]
\subfigure{\includegraphics[width=17mm]{figs/stalex.pdf}}
\vspace{-10pt}
\caption{$t$-visibility for production operation latencies.}
\vspace{-2pt}

\label{fig:tvis}
\end{figure*}

\vspace{\subsectionskip}\subsection{Quorum Sizing}

In addition to $N$$=$$3$, we consider how varying the number of
replicas (N) affects $t$-visibility while maintaining
$R$$=$$W$$=$$1$. The results, depicted in Figure~\ref{fig:varyn}, show
that the probability of consistency immediately after write commit
decreases as $N$ increases.  With 2 replicas, \texttt{LNKD-DISK} has a
$57.5\%$ probability of consistent reads immediately after commit but
only a $21.1\%$ probability with 10 replicas.  However, at high
probabilities of consistency, the wait time required for increased
replica sizes is close.  For \texttt{LNKD-DISK}, the $t$-visibility at
$99.9\%$ probability of consistency ranges from $45.3$ms for 2
replicas to $53.7$ms for 10 replicas.

These results imply that maintaining a large number of replicas for
availability or better performance, results in a potentially large
impact on consistency immediately after writing. However, the
$t$-visibility staleness will still converge quickly.

\begin{figure*}[t!]
\centering
\subfigure{\includegraphics[width=\columnwidth]{figs/nlegend.pdf}}\\[-1mm]
\subfigure{\includegraphics[width=.65\columnwidth]{figs/sweepn-LNKD-DISK.pdf}}
\subfigure{\includegraphics[width=.65\columnwidth]{figs/sweepn-LNKD-SSD.pdf}}
\subfigure{\includegraphics[width=.65\columnwidth]{figs/sweepn-WAN.pdf}}
\vspace{-14pt}
\caption{$t$-visibility for production operating latencies for variable $N$ and $R$$=$$W$$=$$1$.}
\vspace{-6pt}
\label{fig:varyn}
\end{figure*}

\vspace{\subsectionskip}\subsection{Latency vs. {\large $t$}-visibility}

Choosing a value for $R$ and $W$ is a trade-off between operation
latency and $t$-visibility. To measure the obtainable latency gains,
we compared $t$-visibility required for a $99.9\%$ probability of
consistent reads to the $99.9$th percentile read and write latencies.

Partial quorums often exhibit favorable latency-consistency trade-offs
(Table~\ref{table:lat-stale}).  For \texttt{YMMR}, $R$$=$$W$$=$$1$
results in low latency reads and writes ($16.4$ms) but high
$t$-visibility ($1364$ms). However, setting $R$$=$$2$ and $W$$=$$1$
reduces $t$-visibility to $202$ms and the combined read and write
latencies are $81.1\%$ ($186.7$ms) lower than the fastest strict
quorum ($W$$=$$1$, $R$$=$$3$).  A $99.9\%$ consistent $t$-visibility of
$13.6$ms reduces \texttt{LNKD-DISK} read and write latencies by
$16.5\%$ ($2.48$ms).  For \texttt{LNKD-SSD}, across $10M$ writes
(``seven nines''), we did not observe staleness with $R$$=$$2$,
$W$$=$$1$.  $R$$=$$W$$=$$1$ reduced latency by $59.5\%$ ($1.94$ms)
with a corresponding $t$-visibility of $1.85$ms.  Under \texttt{WAN},
$R > 1$ or $W > 1$ results in a large latency increase because this
requires WAN messages. In summary, lowering values of $R$ and $W$ can
greatly improve operation latency and that $t$-visibility can be low
even when we require a high probability of consistent reads.

\begin{table*}
\centering
\begin{tabular}{c|c|c|c|c|c|c|c|c|c|c|c|c|}
\cline{2-13}
 & \multicolumn{3}{|c|}{\texttt{LNKD-SSD}} & \multicolumn{3}{|c|}{\texttt{LNKD-DISK}} & \multicolumn{3}{|c|}{\texttt{YMMR}} & \multicolumn{3}{|c|}{\texttt{WAN}}\\
&\multicolumn{1}{|c}{$L_r$}  & \multicolumn{1}{c}{$L_w$} & \multicolumn{1}{c|}{$t$} &  \multicolumn{1}{|c}{$L_r$} & \multicolumn{1}{c}{$L_w$} & \multicolumn{1}{c|}{$t$} &  \multicolumn{1}{|c}{$L_r$} & \multicolumn{1}{c}{$L_w$} & \multicolumn{1}{c|}{$t$} &  \multicolumn{1}{|c}{$L_r$} & \multicolumn{1}{c}{$L_w$} & \multicolumn{1}{c|}{$t$} \\\hline
\multicolumn{1}{|c|}{$R$$=$$1$, $W$$=$$1$}
&  \textbf{0.66} & \textbf{0.66} & \textbf{1.85} & 0.66 & 10.99 & 45.5 & 5.58 & 10.83 & 1364.0 & \textbf{3.4} & \textbf{55.12} & \textbf{113.0} \\
\multicolumn{1}{|c|}{$R$$=$$1$, $W$$=$$2$}
&  0.66 & 1.63 & 1.79 & 0.65 & 20.97 & 43.3 & 5.61 & 427.12 & 1352.0 & 3.4 & 167.64 & 0 \\
\multicolumn{1}{|c|}{$R$$=$$2$, $W$$=$$1$}
& \textbf{1.63} & \textbf{0.65} & \textbf{0} & \textbf{1.63} & \textbf{10.9} & \textbf{13.6}& \textbf{32.6} & \textbf{10.73} & \textbf{202.0} & 151.3 & 56.36 & 30.2 \\
\multicolumn{1}{|c|}{$R$$=$$2$, $W$$=$$2$}
&  \textbf{1.62} & \textbf{1.64} & \textbf{0} & 1.64 & 20.96 & 0& 33.18 & 428.11 & 0 & 151.31 & 167.72 & 0 \\
\multicolumn{1}{|c|}{$R$$=$$3$, $W$$=$$1$}
&  4.14 & 0.65 & 0 & \textbf{4.12} & \textbf{10.89} & \textbf{0} & \textbf{219.27} & \textbf{10.79} & \textbf{0} & \textbf{153.86} & \textbf{55.19} & \textbf{0} \\
\multicolumn{1}{|c|}{$R$$=$$1$, $W$$=$$3$}
& 0.65 & 4.09 & 0 & 0.65 & 112.65 & 0& 5.63 & 1870.86 & 0 & 3.44 & 241.55 & 0 \\
\hline
\end{tabular}
\vspace{-4pt}
\caption{$t$-visibility for $p_{st} = .001$ ($99.9\%$ probability
  of consistency for $50,000$ reads and writes) and $99.9$th percentile read
  ($L_r$) and write latencies ($L_w$) across $R$ and $W$, $N$$=$$3$
  ($1M$ reads and writes). Significant latency-staleness trade-offs are in bold.}
\vspace{-12pt}
\label{table:lat-stale}
\end{table*}

\vspace{\sectionskip}\section{Discussion and Future Work}
\label{sec:discussion}

In this section, we discuss enhancements to partial quorum
systems that PBS enables along with future work for PBS.

\textbf{Latency/Staleness SLAs.} With PBS, we can automatically
configure replication parameters by optimizing operation latency given
constraints on staleness and minimum durability.  Data store operators
can subsequently provide service level agreements to applications and
quantitatively describe latency/staleness trade-offs to users.
Operators can dynamically configure replication using online latency
measurements.  PBS provides a quantitative lens for analyzing
consistency guarantees that were previously unknown.  This
optimization formulation is likely non-convex, but the state space for
configurations is small ($O(N^2)$).  This optimization also allows
disentanglement of replication for reasons of durability from
replication for reasons of low latency and higher capacity.  For
example, operators can specify a minimum replication factor for
durability and availability but can also automatically increase
$N$, decreasing tail latency for fixed $R$ and $W$.

\textbf{Variable configurations.} We have assumed the use of a single
replica configuration ($N$, $R$, and $W$) across all operations.
However, one could vary these operations over time and across keys.
By specifying a target latency, one could periodically modify $R$ and
$W$ to more efficiently guarantee a desired bound on staleness, or
vice versa. These time-varying configurations require additional
refinements and revisit prior work on fluid
replication~\cite{fluidreplication}.

\textbf{Stronger guarantees.} We have focused on bounded staleness
analysis, but there are other (often stronger) forms of
consistency (such as causal consistency)~\cite{vogels-defs}.
Predicting the probability of attaining more complex consistency
semantics requires additional modeling of application access patterns.
This is possible, but we suspect that modeling the \textit{worst-case}
semantics of these operations will result in unfavorably low
probabilities of consistent operations.  We can see this in Aiyer et
al.'s analysis of Byzantine $k$-quorums~\cite{multi-k-quorum}: in a
worst-case deployment, with an adversarial scheduler, the lower bound
on guaranteed recency is high.  We conjecture that the bound
would be even higher had the authors performed an analysis of stronger
consistency models.

\textbf{Alternative architectures.} Dynamo is conceptually easy to
understand and implement (\textit{WARS}) but is painful to
analytically analyze.  Is there a design that finds a better middle
ground between operational elegance and simplicity of analysis within
the eventually consistent design space?  Prior work on deterministic
bounded staleness (Section~\ref{sec:relatedwork}) provides guidance
but often sacrifices availability and may be more complex to reason
about.

\textbf{Multi-key operations.} We have considered single-key
operations, however the ability to perform multi-key operations is
potentially attractive.  For read-only transactions, if the key
distribution is random and each quorum is independent, we can multiply
the staleness probabilities of each key to determine multi-key
staleness probabilities. Achieving atomicity of writes to multiple
keys requires more complicated coordination mechanisms such as
two-phase commit, increasing operation latency.  Transactions are
feasible but require considerable care in implementation, complicating
what is otherwise a simple replication scheme.

\textbf{Failure modes.} In our evaluation of $t$-visibility, we
focused on normal, steady-state operating conditions. Unless failures
are common-case, they affect tail staleness probabilities (which
appear as latency spikes in \textit{WARS}).  For example, if, as Jeff
Dean of Google suggests~\cite{dean-keynote}, servers crash at least
twice per year, given a ten hour downtime per failure, this results in
.23\% downtime per machine per year.  If failures are correlated, this
may be a problem. If they are independent, a
replica set of $N$ nodes with $F$ failed nodes behaves like an $N-F$
replica set.  The probability of all $N$ nodes failing is $(.23)^N$\%
(``five nines'' reliability for $N$$=$$3$) and the probability tail
will hide these failures.  Quantifying these effects requires
information about failure rates and their impact on latency
distributions but would be beneficial. Modeling recovery semantics
such as hinted handoff~\cite[Section 4.6]{dynamo} would also be
useful.

\vspace{\sectionskip}\section{Related Work}
\label{sec:relatedwork}

We surveyed quorum replication techniques~\cite{prob-quorum-dynamic,
  92-quorums, treequorum, non-strict, multi-k-quorum, quorums-start,
  quorum-placement, partitionedquorum, quorums-alternative,
  prob-quorum, quorum-overview, quorumsystems} in
Section~\ref{sec:background}.  In this work, we specifically draw
inspiration from probabilistic quorums~\cite{prob-quorum} and
deterministic $k$-quorums~\cite{ non-strict, multi-k-quorum} in
analyzing expanding quorum systems and their consistency.  We believe
that revisiting probabilistic quorum systems---including non-majority
quorum systems such as tree quorums---in the context of write
propagation, anti-entropy, and Dynamo is a promising area for
theoretical work.

Data consistency is a long-studied problem in distributed
systems~\cite{consistency-partitioning} and concurrent
programming~\cite{linearizability}.  Given the CAP Theorem and the
inability to maintain all three of consistency, availability, and
partition tolerance~\cite{cap-proof}, data stores have turned to
``eventually consistent'' semantics to provide availability in the
face of partitions~\cite{consistency-partitioning, vogels-defs}.
Real-time causal consistency is the strongest consistency model
achievable in an available, one-way convergent (eventually consistent)
system~\cite{rtc-proof}. However, there is a plethora of alternative
consistency models offering different performance trade-offs, from
session guarantees~\cite{sessionguarantees} to causal+
consistency~\cite{cops} and parallel snapshot isolation~\cite{walter}.
Instead of proposing a new consistency model and building a system
implementing new semantics, we have examined what consistency
existing, widely deployed quorum-replicated systems actually provide.

Prior research examined how to provide deterministic staleness bounds.
FRACS~\cite{frac} allows replicas to buffer updates up to a given
staleness threshold under multiple replication schemes, including
master-drive and group gossip.  AQuA~\cite{aqua} asynchronously
propagates updates from a designated master to replicas that in turn
serve reads with bounded staleness.  AQuA actively selects which
replicas to contact depending on response time predictions and a
guaranteed staleness bound.  TRAPP~\cite{trapp} provides trade-offs
between precision and performance for continuously evolving numerical
data.  TACT~\cite{vahdat-article, vahdat-bounded} models consistency
along three axes: numerical error, order error, and staleness.  TACT
bounds staleness by ensuring that each replica (transitively) contacts
all other replicas in the system within a given time window.  Finally,
PIQL~\cite{piql} bounds the number of operations performed per query,
trading operation latency at scale with the amount of data a
particular query can access, impacting accuracy. These
deterministically bounded staleness systems represent the
deterministic dual of PBS.

Finally, recent research has focused on measuring and verifying the
consistency of eventually consistent systems both theoretically~\cite{podc-hpl} and
experimentally~\cite{measure-consistency, consistency-cidr}. This is useful for validating
consistency predictions and understanding staleness violations.

\vspace{\sectionskip}\section{Conclusion}
\label{sec:conclusion}

In this paper, we introduced Probabilistically Bounded Staleness,
which models the expected staleness of data returned by eventually
consistent quorum-replicated data stores.  PBS offers an alternative
to the all-or-nothing consistency guarantees of today's systems by
offering SLA-style consistency predictions. By extending prior theory
on probabilistic quorum systems, we derived an analytical solution for
the $k$-staleness of a partial quorum system, representing the
expected staleness of a read operation in terms of versions.  We also
analyzed $t$-visibility, or expected staleness of a read in terms of
real time, under Dynamo-style quorum replication.  To do so, we
developed the \textit{WARS} latency model to explain how message
reordering leads to staleness under Dynamo.  To examine the effect of
latency on $t$-staleness in practice, we used real-world traces from
internet companies to drive a Monte Carlo analysis.  We find that
eventually consistent quorum configurations are often consistent after
tens of milliseconds due in large part to the resilience of
Dynamo-style protocols.  We conclude that ``eventually consistent''
partial quorum replication schemes frequently deliver consistent data
while offering significant latency benefits.

\vspace{\sectionskip}\section*{Interactive Demonstration} An
interactive demonstration of Dynamo-style PBS is available at
\url{http://pbs.cs.berkeley.edu/#demo}.

\section*{Acknowledgments}

The authors would like to thank Alex Feinberg and Coda Hale for their
cooperation in providing real-world distributions for experiments and
for exemplifying positive industrial-academic relations through their
conduct and feedback.

The authors would also like to thank the following individuals whose
discussions and feedback improved this work: Marcos Aguilera, Peter
Alvaro, Eric Brewer, Neil Conway, Greg Durrett, Jonathan Ellis, Andy
Gross, Hariyadi Gunawi, Sam Madden, Bill Marczak, Kay Ousterhout, Vern
Paxson, Mark Phillips, Christopher R\'e, Justin Sheehy, Scott Shenker,
Sriram Srinivasan, Doug Terry, Greg Valiant, and Patrick Wendell.  We
would especially like to thank Bryan Kate for his extensive comments
and Ali Ghodsi, who, in addition to providing feedback, originally
piqued our interest in theoretical quorum systems.

This work was supported by gifts from Google, SAP, Amazon Web
Services, Blue Goji, Cloudera, Ericsson, General Electric, Hewlett
Packard, Huawei, IBM, Intel, MarkLogic, Microsoft, NEC Labs, NetApp,
NTT Multimedia Communications Laboratories, Oracle, Quanta, Splunk,
and VMware.  This material is based upon work supported by the
National Science Foundation Graduate Research Fellowship under Grant
DGE 1106400, National Science Foundation Grants IIS-0713661,
CNS-0722077 and IIS-0803690, the Air Force Office of Scientific
Research Grant FA95500810352, and by DARPA contract FA865011C7136.

\balance


{\small
\bibliographystyle{abbrv}
\bibliography{ernst}
}


\end{document}

