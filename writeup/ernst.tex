\documentclass{vldb}
\usepackage{graphicx}
\usepackage{balance, url, amsfonts, verbatim}  % for  \balance command ON LAST PAGE  (only there!)

\title{Probabilistic Consistency and\\ Practical Non-strict Quorum Systems}

\newdef{definition}{Definition}

\begin{document}

\maketitle

\noindent\textit{``All good ideas arrive by chance.''--Max Ernst}

\begin{abstract}


To lower the latency of read and write operations, operators of highly
available distributed data systems lower the required number of replicas
to be contacted per operation. This often results in non-overlapping
read and write replica sets, meaning the data store may return stale
data. However, modern eventually consistent quorum-based data stores
provide no guarantees on the staleness of data returned by these
so-called “non-strict” quorums.

In this work, we further develop the theory of probabilistic quorums
to provide meaningful bounds on the consistency of data items returned
by non-strict quorums of realistically-sized numbers of replicas. We
consider the staleness of data items across multiple versions and
across wall-clock time (taking anti-entropy processes into account),
including monotonic reads session guarantees. With this theory, data
system operators can determine the likelihood of various gradations of
staleness of returned data given a quorum system and then tune their
quorum selection to meet their desired application requirements.  We
formulate an optimization of operation latency in terms of quorum size
given a staleness service level agreement and system load.  Operators
can subsequently trade latency for staleness along two dimensions. We
implement these algorithms in an open source, production-use
distributed data system, Cassandra. Using this implementation, we
demonstrate how this theory can be practically applied through both
synthetic benchmarking and a sample deployment of a social networking
service running on our data system.

\end{abstract}

\section{Introduction}


Many of today's widely-deployed data storage solutions eschew ACID
guarantees in favor of BASE semantics, a large component of which is
the use of so-called eventual consistency of replicas.  In this model,
replicas are only guaranteed to eventually agree on the value of a
particular data item, and reads may return arbitrarily stale data.
The client-side semantics of this eventual consistency depend on
client access patterns~\cite{vogels-defs}, but, unless particular care
is taken, such as always contacting the same replica, the only
mechanism for ensuring strong consistency is to ensure overlapping
read and write quorums~\cite{dynamo}.

However, end-users today have come to expect undefined behavior
offering little to no insight into response consistency from their
eventually consistent data stores.  To achieve favorable performance,
users often employ \textit{non-strict or partial quorums}, in which
read and write sets are not guaranteed to overlap (given $n$ replicas
and read and write quorum sizes $r$ and $w$, $r+w \leq n$).  Modern
quorum-based scalable data systems such as Dynamo~\cite{dynamo} (and
its open source descendants Cassandra~\cite{cassandra},
Riak~\cite{riak}, and Voldemort~\cite{voldemort}) provide no
guarantees on the staleness or consistency returned by non-strict
quorums other than that the system will ``eventually'' provide the
right answer in the absence of new writes.


\begin{figure}
\includegraphics[width=.8\columnwidth]{figs/latency-stale.pdf}
\caption{In this work, we demonstrate the tradeoffs between latency
  and staleness (in versions and real-time) in quorum-based
  distributed databases.  This plot shows both modeled and empirical
  results along these axes for a quorum with $n=5$ replicas, $95\%$
  confidence, and equal rates of reads and writes. Moving along the
  curves corresponds to varying the read and write quorum
  sizes. Additional detail is given in Section~\ref{sec:eval}}
\label{fig:latency-staleness}
\end{figure}

In this paper, we prove that we can do better. Quorum-based data systems
can provide guarantees on the staleness of the data they provide. The
theory community has briefly explored the use of ``probabilistic
quorums'' to provide arbitrarily high probability of strong
consistency~\cite{prob-quorum}; in theory, these systems provide
excellent asymptotic behavior but are limited in practical
applicability due to their reliance on high replication factors.
Furthermore, in this theory, there is no guarantee on the recency of
data returned that is not the most recent version; with small $n$,
the probability of this happening is large---in the tens of percent.

In this work, we provide meaningful metrics for eventually consistent
data stores in the form of probabilistic bounded staleness.  By
relaxing the consistency guarantees from strong consistency to a
bounded-staleness model, we provide exponential improvements in the
probability of returning staler-than-promised data, resulting in
techniques that are useful at replication levels as seen in practice
($n<10$, and easily $n\leq3$).  One can easily predict the staleness
of his or her data (e.g., within 1 version with $70\%$ probability,
with 2 versions with $95\%$ probability, and so on).  From a
service-level agreement perspective, the system can minimize the
probability of returning staler-than-promised data to the order of
fractions of a percent.  Subsequently, operators using non-strict
quorums can adjust their desired trade-offs between latency and
consistency, as shown in Figure~\ref{fig:latency-staleness}.

Specifically, we provide algorithms and models for the staleness of a
data item across multiple versions and across real time and replica
propagation delay.  We can predict the probability of a request
falling within $k$ versions of the latest committed version after $t$
seconds have elapsed, for any $k$ and $t$, offering strongly
exponential improvements in the probability of returning
staler-than-promised data over the traditional literature.  We also
use these consistency measures to provide probabilistic guarantees on
monotonic reads, a form of session guarantees where reads are
guaranteed to return data items no older than what has been previously
read.

Using this theory, we can accurately provide read and write quorum
sizes ($w$,$r$,$n$) that optimize the performance of an operation on a
quorum-based data system given the desired staleness of a particular
partial quorum configuration.  Our models depend only on a model for
the latency of operations, which can be derived online and
continuously tuned.  We implement these algorithms on top of
Cassandra, a widely-deployed open source quorum-based distributed
data system, and demonstrating how to apply partial quorum theory in
practice.  We validate our new theory using several microbenchmarks
and synthetic workloads, then demonstrate the utility of staleness
measures in a Twitter clone with several thousand clients.

We make the following contributions in this paper:

\begin{itemize}

\item We further develop the theory of partial quorums to describe
  staleness probability metrics across both versions and time as well
  as probabilistic monotonic reads consistency.  This leads to
  exponentially lowered load per node and probability of
  inconsistency (Section~\ref{sec:theory}).

\item We employ this theory in optimizing read and write quorum sizes
  for minimum overall operational latency subject to a given
  service level agreement on the staleness of the data returned and
  the probability of further inconsistency (Section~\ref{sec:optimize}).

\item We implement this optimization layer and our algorithms in a
  modern distributed key-value store and demonstrate their utility in
  practice under both synthetic benchmarking and a web application
  workload (Section~\ref{sec:eval}).

\end{itemize}

\section{Background}

Write some stuff about quorum systems, replication, and so on.
SHIVARAM IS HANDLING MOST OF THIS.


There are two main reasons to replicate data: durability and
scalability.  First, in the event of server failure, having stored the
data on multiple replicas allows end-users to continue to access the
data.  The replication factor in this case depends largely on the
relative ``importance'', or cost of losing the data.  Secondly, each
server has a maximum capacity, or number of requests that it can serve within a
given time period.  All else equal, replicating the data and
performing appropriate load-balancing lowers the load on each
individual server storing the data.

However, coordinating replicas has a cost; ensuring that all replicas
are up to date is expensive, but, as previously mentioned, giving up
ACID semantics lessens this cost.  By themselves, BASE semantics are
not immediately problematic: one cannot necessarily have all of
consistency, availability, and partition tolerance~\cite{cap-proof},
so, to achieve availability and partition tolerance, BASE systems give
up consistency.


\section{Probability and Partial Quorums}
\label{sec:theory}

SUMMARY OF THIS SECTION

\subsection{Quorum Foundations: Theory}

Quorum systems have long been proposed as a replication strategy for
distributed data storage.  Informally, a strict quorum system defines
a set of sets of nodes in a distributed system with the property that
any two sets in the quorum system overlap (have non-empty
intersection).  When considering distributed get/put operations,
reading and writing to sets of nodes in a strict quorum system ensures
strong consistency in the absence of failures; the minimum sized
quorum defines its fault tolerance.  A simple example of a strict
quorum system is the majority quorum system, in which each quorum is
of size $\frac{N}{2}+1$.  However, the theory literature contains
numerous alternative quorum systems providing varying asymptotic
properties of capacity, scalability, and fault-tolerance, from
tree-quorums to grid-quorums.  Jim\'{e}nex-Peris et. al provide a
useful overview of these traditional, strict quorum
systems~\cite{quorums-alternative}.

Non-strict quorum systems are a natural extension of strict quorum
systems: at least two sets in a non-strict quorum system do not
overlap.  There are two relevant variants of non-strict quorum systems in
the literature: probabilistic quorum systems and k-quorums.

\textit{Probabilistic quorum systems} provide probabilistic guarantees
on the consistency of data returned by non-strict quorums.
Probabilistic quorums provide optimal (expected) load and fault
tolerance with an arbitrarily small probability of
inconsistency~\cite{prob-quorum}.  Intuitively, this is a consequence
of the Birthday Paradox: as the number of replicas increases, the
probability of non-overlap between any two subsets is quite low.
Given $n$ replicas and randomly chosen read and write quorums of sizes
$r$ and $w$, we can calculate the probability of the read quorum not
containing the value written by the write quorum.  The probability of
staleness is the number of quorums of size $r$ composed of nodes that
were not written to in the write quorum divided by the number of
possible quorums of size $r$:
\begin{equation}
\label{eq:prob-strict}
p_{stale}=\frac{{n-w \choose r}}{{n \choose r}}
\end{equation}
It is readily apparent that the probability of staleness is quite high
except for large values of $n$.  With $n=100$, $r=w=30$, $p_{stale} =
1.88 \times 10^{-6}$~\cite{nonstrict-availability}.  However, with
$n=3$, $r=w=1$, $p_{stale} = .\overline{6}$.  The asymptotics of these
systems are excellent---but only asymptotically.  To the best of our
knowledge, probabilistic quorums have only been used to study the
probability of strong consistency only.

\textit{$k$-quorum systems} provide strong (non-probabilistic)
guarantees that the partial quorum system will return a value that was
written within $k$ versions of the most recent
write~\cite{nonstrict-availability}.  In the single-writer scenario,
one can imagine a round-robin write scheduling scheme where each write
is sent to $\frac{n}{K}$ replicas such that each replica is no more
than $K$ versions out-of-date.  However, with multiple writers, one
loses the ordering properties that the single-writer was able to
control, and the best known algorithm for the pathological case
results in a lower bound of $(2n-1)(k-1)+n$ versions staleness~\cite{k-quorum-lb}.
Again, this prior work on $k$-quorums focused on purely deterministic
verion staleness.

This prior work has two properties with important implications for
practitioners.  First, existing theory treats quorums as static; a
write quorum is chosen and no other replicas in the system
subsequently learn about the value unless it is written again.  The
theory does not model anti-entropy processes.  Second, much of this
prior work assumes Byzantine failure.  If the description of prior
work seemed simplistic, it is largely because most of the literature
content addresses problems such as adversarial quorum selection and
scheduling.  In this work, we disregard both of these assumptions.  We
elaborate further in the next section.

\subsection{Quorum Foundations: Practice}
\label{sec:practice}

In practice, many distributed data management systems employ variable
quorum sizes. Amazon's Dynamo~\cite{dynamo} is the progenitor of a
class of eventually-consistent key-value stores that provide
quorum-style replication that includes Apache Cassandra, Basho's Riak,
and LinkedIn's Voldemort\footnote{Other BASE-style systems may employ
  master-slave replication, as in Apache HBase~\cite{hbase}.}.  In
this paper, we discuss Dynamo-style quorum systems, particularly
because we are not aware of any significally different production
quorum data systems.  However, with some work, we believe that other
systems can adopt our methodology.  Similarly, we focus on key-value
stores as the aforementioned systems provide some variant of key-value
architecture and do not provide full RDBMS semantics.  Quorum systems
may be employed in RDBMS replication, but, for simplicity, we describe
key-value stores here.

Dynamo-style quorum systems employ one quorum system per key,
typically maintaining the mapping of values to a set of nodes using a
consistent-hashing scheme or a centralized membership protocol. Each
node acts as a replica for multiple keys.  Client read and write
requests are sent to a proxy node in the key-value store and are
subsequently forwarded by the receiving node to all other nodes
assigned to that key as replicas.  The proxy node acknowledges
operation success when it has heard from a pre-defined number of
replicas.  Dynamo-style systems return guaranteed strongly consistent
data when $R+W > N$.  However, for improved latency, operators often
set $R+W \leq N$, typically $R=W=1$, as is the default for these
database systems (see Table TBD).

\begin{figure}
\centering
\includegraphics[width=.8\columnwidth]{figs/dynamo-quorum.pdf}
\caption{Diagram of control flow for client write to Dynamo-style
  quorum.  Here, $N=3$, $W=2$. Client writes are proxied and sent to
  all replicas. The write succeeds when $W$ replicas respond.  Note
  that the proxy is likely a replica as well, invoking a local write.}
\label{fig:dynamo-quorum}
\end{figure}

There are significant differences between data systems and the theory
describing quorum operation.  First, replication factors for
distributed data systems are relatively low.  Typical replication
factors are between one and three, however the literature has proposed
replication up to 10 (CITE CHAIN-REPLICATION).  Second, (in the
absence of failure), in Dynamo-style partial quorums, the number of
replicas that receive a write increases even after the operation
returns.  This is a simple version of anti-entropy.  Third, these
systems are often deployed in a trusted or semi-trusted computing
environment; there may be adversarial threats against data integrity
and denial-of-service attacks, but the underlying computing hardware
is non-adversarial. Within a controlled data center, the failure modes
are certainly reduced from the Byzantine case, and, despite its many
risks, the adversarial failures covered by prior theory appear
unlikely in emerging ``cloud computing'' environments.

\subsection{Assumptions}

These practical concerns guide the following theoretical
contributions.  Accordingly, we assume a quorum model where $w$ ($r$)
of $n$ replicas are randomly selected for each write (read) operation.
Unless otherwise noted, we consider fixed $w$ across multiple
operations.  We begin by considering a model without entropic
processes ($w$ is constant), then expand our model to consider
time-varying $w$. We consider a write ``committed'' once it has
reached $w$ replicas. We discuss further refinements to these
assumptions in Section \ref{sec:discussion}.

\subsection{Probabilistic $k$-quorums}

Probabilistic quorums allow us to determine the probability of
returning the most recent value written to the database, but it is
also useful to know the probability of returning a value within a
bounded number of versions.  Deterministic $k$-quorums provide
theoretical approaches to achieving this, however we can do better in
the presence of multiple writers by extending this theory to consider
probabilistic $k$-quorums.  In this model, we consider static write
quorums (no anti-entropy), but compose multiple write quorums to model the probable overlap of $k$ independent write sets.
\begin{definition}
A quorum system obeys \textit{probabilistic $k$-quorum consistency} if, with
probability $1-p_{staler}$, at least one value in any read quorum will
have been committed no later than $k$ versions after the latest committed
version when the read begins.
\end{definition}
Versions whose writes that are not yet committed (in-flight) may be
returned by a read in this formulation of probabilistic $k$-quorums
(see Figure \ref{fig:timelines}A).  The $k$-quorum literature defines these as $k$-regular semantics~\cite{nonstrict-availability}.

\begin{figure}
\centering
\includegraphics[width=\columnwidth]{figs/timelines.pdf}
\caption{Possible versions returned by read operations under
  probabilistic $k$-quorums (A) and monotonic reads (B). In
  $k$-consistency, the read operation will return a version no later
  than $k$ versions older than the last committed value when it
  started; more versions may be committed during the read and may be
  returned.  In monotonic reads consistency, the staleness depends on
  the number of versions committed since the time the client last
  completed a read.  This is determined by the proportion of client's
  reads to the number of writes committed to the object.}
\label{fig:timelines}
\end{figure}

The probability of returning a version of a key within $k$ versions is
equivalent to intersecting one of $k$ independent write
quorums\footnote{In the language of probabilistic quorums, we have
  constructed a probabilistic $k$-quorum from $k$ $\varepsilon$-quorum
  systems where $\varepsilon \leq \sqrt[k]{p_{staler}}$. This system
  has load $\geq \frac{1-\varepsilon^{\frac{1}{2k}}}{\sqrt{n}}$, an
  exponentially lower bound than a strict probabilistic quorum.  This
  follows immediately from~\cite[Corollary 3.12]{prob-quorum}}.
Quorums are chosen at random, so the probability of non-intersection
is simply Equation \ref{eq:prob-strict} exponentiated by $k$:
\begin{equation}
\label{eq:k-consistency}
p_{staler} = \left(\frac{{n-w \choose r}}{{n \choose r}}\right)^k
\end{equation}

For the $n=3, r=w=1$ case, this means that the probability of
returning a version within $2$ versions is $.\overline{5}$, within $3$
versions $.\overline{703}$, and within $5$ versions $> .868$, and $10$
versions $>.98$.  With $n=3, r=1, w=2$ (alternatively, $r=2, w=1$),
these probabilities increase: $k=1 \rightarrow
.\overline{6}$, $k=2 \rightarrow .\overline{8}, k=5 \rightarrow >
.995$.

\subsection{Probabilistic Monotonic Reads}

With additional information, we can use probabilistic $k$-quorums to
predict whether a client will ever read stale data.  This property,
known as \textit{monotonic reads} consistency is a well-known session
guarantee~\cite{sessionguarantees}.

\begin{definition}
\label{def:prob-mr}
A quorum system obeys \textit{probabilistic monotonic reads consistency} if, with probability at least $1-p_{staler}$, at
least one value in any read quorum the same version or a newer version
than the client's previously read value, where versions are defined
over the global commit ordering.
\end{definition}

To guarantee that a client sees monotonically increasing versions, it
can continue to contact the same replica~\cite{vogels-defs}, however
this is insufficient to guarantee strict monotonic reads (where the
client reads strictly newer data).  Definition~\ref{def:prob-mr} can
be adapted to acommodate strict monotonic reads (omitted for brevity).

We observe that monotonic reads is a special case of probabilistic
$k$-quorums where $k$ is determined by a client's rate of reads from a
data item ($\gamma_{cr}$) and the global, system-wide rate of writes
the same data item ($\gamma_{gw}$)\footnote{When constructed from $k$
  $\varepsilon$-consistent systems (as we have here), this consistency
  model has load $\geq
  \frac{(1-p_{staler}^{\frac{1}{2C}})}{\sqrt{n}}$, where
  $C=1+\frac{\gamma_{gw}}{\gamma_{cr}}$.}.  If we know these rates
exactly, the number of versions between client reads is
$\frac{\gamma_{gw}}{\gamma_{cr}}$, as shown in Figure
\ref{fig:timelines}B.  We can calculate the probability of
probabilistic monotonic reads as follows, effectively using Equation
\ref{eq:k-consistency} where $k=1+\frac{\gamma_{gw}}{\gamma_{cr}}$:

\begin{equation}
\label{eq:prob-mr}
p_{staler} = \left(\frac{{n-w \choose r}}{{n \choose r}}\right)^{1+\gamma_{gw}/\gamma_{cr}}
\end{equation}
For strict monotonic reads, exponentiate where $k=1+\frac{\gamma_{gw}}{\gamma_{cr}}$.

In practice, we may not know these exact rates, but, by measuring
their distribution, we can calculate an expected ratio that we can
integrate into these calculations.  However, by performing appropriate
admission control, operators can control these rates to achieve a
particular staleness target.

\subsection{Probabilistic $RT$-quorums}

Until now, we have considered only static write quorums.  However, as
we discussed in Section \ref{sec:practice}, modern quorum systems
propagate writes to all members of the quorum system for a given key.
This is commonly known as anti-entropy.  For generality, in this
section, we will discuss general anti-entropy models. However, we
explicitly model the Dynamo-style anti-entropy mechanism in Section
\ref{sec:dynamo-prop}.  Probabilistic $RT$-quorums model the
propagation of writes across wall-clock or real-world time such that
the set of replicas with a given version of the data (or later) grows
over time\footnote{Node failures shrink this set and can be
  incorporated into this model, but we do not consider them here.}.
We denote the cumulative density function describing the number of
replicas $\mathcal{W}$ that have a particular version $v$ (or a
version committed after $v$\footnote{We assume writes obey a total
  ordering. This can be accomplished using synchronized clocks (last
  writer wins) or with causal ordering and commutative merge
  functions~\cite{cops}.}) $t$ seconds after committing as
$P_w(\mathcal{W}, t)$.

\begin{definition}
A quorum system obeys \textit{probabilistic $RT$-quorum consistency}
if, with probability $1-p_{staler}$, at least one value in any read
quorum will either have been committed no greater than $t$ units of
time before the read began or will belong to an in-flight write.
\end{definition}

By definition, $P_w(c,0) = 1$ $\forall c \in [0, w]$.  Intuitively, at
commit time, $w$ replicas will have the value, so the probability that
zero to $w$ replicas have the value immediately after commit is
exactly $1$.  We can model the probability of probabilistic $RT$-consistency for an interval $t$ by summing the conditional probabilities of each possible $\mathcal{W}$:

\begin{equation}
p_{staler} = P_l(w, t)+\sum_{c\in(w, n]} \frac{{n-c \choose r}}{{n \choose r}}\cdot [P_w(c, t)-P_w(c-1,t)]
\end{equation}

In practice, $P_l$ depends on the expected latency of operations and can be
approximated analytically (Section \ref{sec:dynamo-prop} or measured
online\footnote{For this reason, it is difficult to analytically
  determine the load of any probabilistically consistent system
  dependent on real-time operations.}.

\subsection{Probabilistic $\langle k, t
  \rangle$-quorum Consistency}

We can combine the previous models to combine both versioned and
real-time staleness metrics to answer questions of the following form:
what is the probability that a read will return a value no later than
$k$ versions old if the write committed no sooner than $t$ seconds
ago?
\begin{definition}
A quorum system obeys \textit{probabilistic $\langle k, t
  \rangle$-quorum consistency} if, with probability $1-p_{staler}$, at
least one value in any read quorum will have been committed no later
than $k$ versions after the latest committed version when the read
begins, provided the read begins $t$ units of time after the previous
$k$ versions commit.
\end{definition}
The definition of $p_{staler}$ follows from the prior definitions:
\begin{equation}
p_{staler} = \left(p_l(n, t)+\sum_{c\in[w, n)} \frac{{n-c \choose r}}{{n \choose r}} \cdot p_l(c, t)\right)^k
\end{equation}
In this equation, we assume the pathological case where the last $k$
writes all occurred at the same time.  This is not likely in practice,
so if we can determine the time since commit for the last $k$ writes,
we can improve this staleness bound by considering each quorum's $p_{stale}$ at $RT=t$ separately.

Note that the prior definitions of consistency are encapsulated by
probabilistic $\langle k, t \rangle$-quorum consistency. probabilistic
$k$-quorum consistency is simply probabilistic $\langle k, 0
\rangle$-quorum consistency, probabilistic monotonic reads consistency
is $\langle 1+\frac{\gamma_{gw}}{\gamma_{cr}}, 0 \rangle$-quorum
consistency, and $RT$-consistency is $\langle 0, RT \rangle$-quorum
consistency.

\section{Applied Theory and Refinements}
\label{sec:optimize}

\subsection{Dynamo-style Operation Latency}

\label{sec:dynamo-prop}

Until now, we have assumed that the expected latency of operations
reaching a given number of replicas by a particular time
($p_w(\mathcal{W}, t)$ for writes) in the case of writes is known.  We
can measure this distribution empirically (Section
\ref{sec:real-latency}), but we can also model this latency
analytically, given a few assumptions.

To begin, we consider write latency.  With Dynamo-style queries, we
want to determine the probability that $w$ of the replicas in the
write quorum $\mathcal{Q}$ respond within time $t$.  We can determine
this in at least two ways.  One way is to find use order statistic
theory to determine the probability that the $w$th replica reasponds
by time $t$.  However, this is equivalent to taking the minimum
response time over the maximum of each possible set of $w$ replicas
from $\mathcal{Q}$, which works out more cleanly in this formulation.
Given the write latency cumulative density functions for a set of
nodes $S$, $L_{ws}(S, t)$, the probability density function $p_w$ of
write latency of a set of nodes is:
\begin{equation}
p_w(\mathcal{W}, t) = min(\{L_{ws}(S, t) \mid S \subseteq \mathcal{Q}, |S| = \mathcal{W}\})
\end{equation}

To determine the expected latency of a write to a quorum of size
$\mathcal{W}$, enoted $WQ_l(\mathcal{W})$, we simply need to consider
the conditional probability of $\mathcal{W}$ writes completing across
all possible time values $t$.  However, this depends on the
distribution of $L_{ws}(S, t)$, which requires some
assumptions. Determining the ordinal probability of a set of randomly
distributed variables (in our case, the minimum of all $L_{ws}(S,
t)$), is possible but is also fairly complicated.  The difficult
arises in determining the distribution of a given $L_{ws}(S)$.  If we
assume that all nodes in $S$ obey independent latency distributions,
this is tractible.  If we assume that they are dependent and therefore
obey a joint distribution, then, as $w$ grows, solving for this
probability becomes much more difficult~\cite{needed}.  Approximation
algorithms can assist here~\cite{needed}, but we can also make
simplifying assumptions about the distribution of these latencies.

If we assume that latencies are independently, identically distributed
(IID), that is, if each node obeys the same latency distribution and
each nodes latency is independent of the other nodes's latency, this
equation is greatly simplified.  Under IID assumptions, the time for a
write to reach every node in the quorum obeys a single latency
CDF\footnote{Note that this distribution only captures the request
  forward, not the round-trip time before the acknowledgement.  The
  replica can serve the data before it responds to the proxy.},
$L_w(t)$ (PDF, resp. $l_w(t)$).  The probability that $w$ nodes all
respond within time $t$ is simply $(L_w(t))^\mathcal{W}$, and the
minimum over all such of these sets is the CDF $P_w$:
\begin{equation}
P_w(\mathcal{W}, t) = 1-(1-(L_w(t))^\mathcal{W})^{n \choose \mathcal{W}}
\end{equation}
This is indeed simplifying, but it makes the theory much simpler.  We
discuss the validity of this IID assumption when we measure latencies
experimentally in Section \ref{sec:real-latency}.

Under the IID assumption, we can easily determine the expected
Dynamo-style write quorum operation latency:
\begin{equation}
WQ_l(\mathcal{W}) = \int_0^{\infty} p_w(\mathcal{W}, t) dt
\end{equation}
In practice, the proxy is often a replica as well, so for a write
quorum of size $w$, we only need to consider $\mathcal{W}=w-1$ writes.

Similarly, to determine how many replicas have a particular value (even if they have not acked) after a write quorum of size $w$, we evaluate:
\begin{equation}
P_{w:have}(\mathcal{W}, t) = 1-(1-(L_w(t))^{\mathcal{W}-w})^{n-w \choose \mathcal{W}-w}
\end{equation}


Thus far, we have only considered the write operation latencies.  The
read quorum latency may be different from the write latency depending
on node-level actions such as forced logging or disk-bound operations.
However, the expected read operation latency for a quorum of size
$\mathcal{R}$, $RQ_l(\mathcal{R})$, can be calculated similarly given a
model for the latency of reads ($L_{rs}(S,t)$ and $L_r(t))$.

\subsection{Optimization Formulation}

With these equations, we can optimize the selection of a partial
quorum system to minimize overall operation latency subject to
constraints on staleness. We can guarantee bounded staleness by
ensuring that $p_{staler} = 1$, however this is only possible in
$k$-quorum consistency as the real-time components of $RT$-quorum
consistency and its variants are inherently probabilistic (unless we
can prove an absolute bound on operation latency).  We can use a
simply formulated optimization program to determine what combination
of $r$ and $w$ minimizes latency while satisfying the system
constraints and user requirements.

Given $n$, desired $p$, $w_{min}$ (minimum durability of writes),
consistency model (with necessary parameters--$k$, $t$, $\gamma_{cr}$,
etc.), and the relative weighted ``importances'' of read
and write latency, $c_r$ and $c_w$, we can determine $r$ and $w$:

\begin{equation}
 \begin{array}{rl}
    \min        & c_r\cdot RQ_l(r) +c_w \cdot WQ_l(w) \\
    \mbox{s.t.} & p \ge p_{staler} \\
                & w \ge w_{min}.
    \end{array}
\end{equation}

We implement and validate this optimization framework in
Section~\ref{sec:optimization}.

\subsection{Discussion}
\label{sec:discussion}

There are several aspects of distributed data systems that we have not yet
addressed.  Here, we briefly discuss improvements to our models.

\textbf{Staleness detection.} Each of the $p_{staler}$ bounds
describes a probability that a particular key is no staler than
specified.  In the context of probabilistic consistency SLAs, it would
be useful to determine whether a key is staler than promised by a
given SLA.  Determining this fact on-line is tantamount to achieving
strong consistency, however there are at least two strategies for
mitigating this difficulty. First, additional gossiping such as
regular negative acknowledgements can notify nodes of potential
version staleness.  This strategy has been discussed in the
literature~\cite{tocite} for non-probabilistic quorum systems.
Second, the system can provide asychronous notifications of staleness
information.  Proxies can keep a growing log of versions which, with
appropriate timestamping, can be used to determine if previously
returned values were staler than promised.  This is fairly
straightforward but requires careful garbage collection to manage
proxy-side state as the number of client requests scales.
Additionally, the possibility of proxy failure requires that prior
operations be durably logged; however, provided the proxy eventually
comes online and updates its log, the asynchronous guarantee holds.
Both of these schemes are feasible yet add significant complexity to a
probabilistically consistent quorum system.

\textbf{Multi-key transactions.} We have considered single-key operations,
however the ability to perform distributed transactions is potentially
attractive.  For read-only transactions, if the key distribution is
random, each quorum is independent, so we can simply multiple the
staleness probabilities of each key.  Achieving atomicity of writes to
multiple keys requires more complicated coordination mechanisms such
as two-phase commit.  Again, transactions are feasible but require
considerable care in implementation, complicated what is otherwise a
simple replication scheme.

\textbf{Node failures.} Fail-stop node failures can be easily
incorporated into our latency models. The probability of a node
permanently failing $p_{f-perm}$ can be represented by setting the
probability of infinitely long read and write
operations. $L_{w}(\inf)=L_{r}(\inf)=p_{f-perm}$.  Intermittent node
failures of length $t_{fail}$ can be represented by setting
$L_{w}(t_{fail})$ to the probability of the node failing.  In this
way, operation latency captures all expected failure semantics.

\textbf{Read repair and active anti-entropy.} Modern Dynamo-style
systems use a technique known as \textit{read repair} to reconcile
divergent versions at query-time.  If two nodes return different
responses to a \textit{get} request, the responses are merged and
forwarded to the determined out-of-date node.  Dynamo performs
additional anti-entropy using Merkel trees, however the open source
databases we have examined only perform read repair (CITATION
NEEDED--at least Cassandra does this). We do not model read repair or
additional anti-entropy processes here (although doing so requires
only minimal changes to our model), so we overestimate the chance of
staleness violation.

\textbf{Variable $w$.} We have assumed the use of a single $w$ across
all writes.  However, many KVSs such as Cassandra and Riak allow the
use of per-operation consistency\footnote{Voldemort specifies
  consistency requirements at the schema level. This is ostensibly an
  engineering decision that could be changed with minimal
  modifications to the wire protocol.}

\textbf{99th Percentile Latency.}

\subsection{Typical Quorum Configurations}

CASSANDA: R=1, W=1 (\url{https://svn.apache.org/repos/asf/cassandra/branches/cassandra-1.0/interface/cassandra.thrift})

RIAK: N=3, R,W QUORUM \url{https://github.com/basho/riak_kv/blob/1.0/src/riak_kv_app.erl}
\url{http://wiki.basho.com/Riak-Glossary.html#Quorum}

Voldemort: does not provide production configs besides testing (N=2)

\section{Experimental Evaluation}
\label{sec:eval}

\subsection{implementation}

\subsection{Operation Latency}
\label{sec:real-latency}

In our measurements on EC2, the correllation of writes was MADE UP,
meaning the operations were effectively IID.  However, we can
approximate an environment with a higher correlation by modifying $w$.
In an environment where half of the nodes are co-located on a rack, we can treat $w$ as $w/2$.  We show the tradeoffs between correlation and latency in Figure \ref{fig:correlation}.

\begin{figure}
\centering
\includegraphics[width=.8\columnwidth]{figs/correlation.pdf}
\caption{Expected latency of write quorums of $w$ replicas in Dynamo-style operation versus quorum element correlation.}
\label{fig:correlation}
\end{figure}

\subsection{Optimization and SLAs}
\label{sec:optimization}

\begin{figure}
\centering
\includegraphics[width=.8\columnwidth]{figs/bothdimensions.pdf}
\caption{Time and version staleness required to ensure varying SLA constraints on the probability of returning staler-than-promised data in simple microbenchmarking.}
\label{fig:prob-staler}
\end{figure}

\begin{figure}
\centering
\includegraphics[width=.8\columnwidth]{figs/inaccuracy.pdf}
\caption{Accuracy of model compared to experimental microbenchmarking.
  The size of each point is linearly proportional to the inaccuracy of
  the model's predictions.}
\label{fig:prob-staler}
\end{figure}




\subsection{Twitter Clone}

\begin{figure}
\centering
\includegraphics[width=.8\columnwidth]{figs/twissandra.pdf}
\caption{Time and version staleness tradeoffs as measured by end users
  of Twitter clone.}
\label{fig:twissandra}
\end{figure}

\section{Related Work}

Theory

Vadhat

\section{Conclusion}

\section*{Acknowledgements}

\balance

\bibliographystyle{abbrv}
\bibliography{ernst}

\end{document}

