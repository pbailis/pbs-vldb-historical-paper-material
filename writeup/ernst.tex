\documentclass{vldb}
\usepackage{graphicx, subfigure}
\usepackage{balance, url, amsfonts, verbatim, mathtools}  % for  \balance command ON LAST PAGE  (only there!)

\title{Probabilistically Bounded Staleness and\\ Practical Partial Quorum Systems}

\newdef{definition}{Definition}

\begin{document}

\interfootnotelinepenalty=10000
\hyphenation{prob-a-bil-is-tic-ally}

\maketitle

\noindent\textit{``All good ideas arrive by chance.''--Max Ernst}


\begin{abstract}

Modern storage systems employing quorum replication are often
configured to use partial, or non-strict quorums.  These systems wait
only for a subset of their replicas to respond to a request before
returning an answer without guaranteeing that read and write sets
overlap.  While these real-world partial quorum mechanisms provide
only basic eventual consistency guarantees, these configurations are
frequently ``good enough'' for practitioners given their latency
benefits. In this work, we discuss why partial quorums are often
acceptable in practice by analyzing the staleness of data they return.
Extending prior work on strongly consistent probabilistic quorums and
using the models of anti-entropic Dynamo-style processes, we introduce
Probabilistically Bounded Staleness (PBS) consistency, which provides
expectations of bounds on staleness across both versions and wall
clock time.  We derive a closed-form solution for versioned staleness
and model real-time staleness for representative Dynamo-style systems
under internet-scale production workloads.  We quantitatively
demonstrate why, in practice, systems employing non-strict quorums
rarely serve inconsistent data.

\end{abstract}


\section{Introduction}

Modern distributed storage systems need to be scalable, highly
available, and provide high performance for reads and writes.  These
systems typically replicate data across different machines or even
across datacenters for two reasons: first, to provide high
availability when components fail and second, to provide improved
performance by serving requests from multiple replicas.  In order to
provide predictably low read and write latency, systems often eschew
consistency across replicas and instead provide eventual
consistency. In this model, replicas are only guaranteed to eventually
agree on the value of a particular data item, and reads may return
arbitrarily stale data.

Distributed quorums are often used to ensure strong consistency across
multiple replicas of a data item by overlapping read and write
sets. However, service latency is critical. For example, at Amazon,
100 ms of additional latency resulted in 1\% drop in
sales~\cite{amazon-latency}, while 500 ms of additional latency at
Google due to increasing the number of search results computed
resulted in a corresponding 20\% decrease in search
traffic~\cite{google-talk}.  At scale, these decreases correspond to
billions of dollars of lost revenue.

One technique for lowering operation latency is employing
\textit{partial or partial quorums}, in which read and write sets
are not guaranteed to overlap (given $N$ replicas and read and write
quorum sizes $R$ and $W$, $R+W \leq N$).  Modern quorum-based scalable
data systems such as Dynamo~\cite{dynamo} (and its open source
descendants Cassandra~\cite{cassandra}, Riak~\cite{riak}, and
Voldemort~\cite{voldemort}) provide two categories of quorum
operation: overlapping quorums and strong consistency or
variable-sized partial quorums with no guarantees on the staleness or
consistency returned, other than that the system will ``eventually''
provide the most recent version in the absence of new writes.  Despite
these weak guarantees, operators frequently employ partial
quorums--they are often ``good enough'' for applications, especially
given their latency benefits.

Prior work evalued the probability of strong consistency for static
partial quorum systems, where, in the absence of failures, the read
and write sets do not change over time.  Theoretical research on
\textit{probabilistic quorums} demonstrated how partial quorums can be
employed to provide arbitrarily high probability of strong
consistency~\cite{prob-quorum}. In theory, these systems provide
excellent asymptotic behavior but are limited in practical
applicability due to their reliance on high replication factors.  More
problematic, however, is that this work predicts \textit{how often} a
response will be stale but not \textit{how stale} a stale response
will be. In the real world, quorum systems are not static: writes
propagate asynchronously across time. Similarly, as evidenced by the
widespread deployment of eventually consistent data systems,
applications can tolerate varying degrees of staleness.

In this paper, we present algorithms and models for accurately
predicting the staleness of partial quorums across multiple versions
and wall clock time, or Probabilistically Bounded Staleness (PBS). PBS
can be used to determine the probability of reading one of the last
$k$ versions of a data item ($k$-staleness), of reading a data item
$t$ seconds after it is written ($t$-visibility), and of experiencing
a combination of the two ($k$-$t$-staleness).  We also use these
consistency measures to provide probabilistic guarantees on monotonic
reads, a form of session guarantees where reads are guaranteed to
return data items no older than what has been previously read.

Observing multi-version staleness is highly unlikely.  Under static
quorum systems, the probability of returning a version $k$ versions
older than the most current is exponentially reduced by a factor of
$k$.  In the presence of multiple concurrent writes in a non-static
system, there is a similar effect, determined in large part by the
staggering of the writes.  With a modest global write rate to a key, a
client making read requests no greater than the write rate has a high
probability of observing monotonic reads consistency.  Many
applications can tolerate stale versions, particularly programs
exposing append-only data structures such as logs and streaming
timelines and, more generally, commutative updates.

Observing staleness across time depends on the write propagation rate
and the latency between coordinators and replicas.  In this paper, we
examine Dynamo-style partial quorums, which are currently the most
widely deployed quorum replication strategy.  We show how staleness in
the Dynamo model is dependent on message reordering and discuss the
effect of latency variance on expected staleness across time.
Additionally, we propose asynchronous staleness detectors within
Dynamo.  We present algorithms for both a simple detector that is
easily implementable within Dynamo but may provide false positives and
a more complex, precise detector.  Using data collected in experiments
on Amazon's EC2 and using production distributions provided by two
internet-scale companies, we analyze the likelihood of $t$-staleness
across time and replica configurations.



We make the following contributions in this paper:

\begin{itemize}

\item We develop the theory of Probabilistically Bounded Staleness
  (PBS) for partial quorums. PBS describes staleness probability
  metrics across both versions ($k$-staleness) and time
  ($t$-visibility) as well as probabilistic monotonic reads
  consistency.

\item We provide a closed-form analysis of $k$-staleness demonstrating
  how the probability of receiving data $k$ versions stale is
  exponential in $k$ for static quorums.  $k$-staleness tolerance also
  exponentially lowers formally-defined quorum system load.

\item We provide a model for $t$-visibility in
  Dynamo-style partial quorum systems, showing how
  staleness is dependent on message reordering in the system.  We
  evalute the $t$-visibility of Dynamo-style systems using a
  combination of experimentally-gathered data and production traces.

\end{itemize}

The remainder of this paper is as follows: in
Section~\ref{sec:background}, we provide an overview of quorum systems
and Dynamo-style quorums.  In Section~\ref{sec:pbs}, we introduce
Probabilistically Bounded Staleness and provide a closed-form solution
to $k$-staleness under traditional static quorums.  In
Section~\ref{sec:dynamo}, we model Dynamo-style quorum systems and
discuss when staleness occurs.  In Section~\ref{sec:dynamoeval}, we
use real-world workloads and Monte Carlo analysis to determine how often Dynamo-style quorums return stale values in practice.  In Section~\ref{sec:discussion}, we describe further improvements to dynamic partial quorums, in Section~\ref{sec:relatedwork}, we discuss related work, and, in Section~\ref{sec:conclusion}, we conclude.


\section{Background}
\label{sec:background}

\subsection{Quorum Foundations: Theory}

Quorum systems have long been proposed as a replication strategy for
distributed data storage.  Informally, a strict quorum system defines
a set of sets of nodes in a distributed system with the property that
any two sets in the quorum system overlap (have non-empty
intersection).  When considering distributed get/put operations,
reading and writing to sets of nodes in a strict quorum system ensures
strong consistency in the absence of failures; the minimum sized
quorum defines its fault tolerance.  A simple example of a strict
quorum system is the majority quorum system, in which each quorum is
of size $\frac{N}{2}+1$.  However, the theory literature contains
numerous alternative quorum systems providing varying asymptotic
properties of capacity, scalability, and fault-tolerance, from
tree-quorums to grid-quorums.  Jim\'{e}nex-Peris et. al provide a
useful overview of these traditional, strict quorum
systems~\cite{quorums-alternative}.

Partial quorum systems are a natural extension of strict quorum
systems: at least two sets in a partial quorum system do not
overlap.  There are two relevant variants of partial quorum systems in
the literature: probabilistic quorum systems and k-quorums.

\textit{Probabilistic quorum systems} provide probabilistic guarantees
on the consistency of data returned by partial quorums.
Probabilistic quorums provide optimal (expected) load and fault
tolerance with an arbitrarily small probability of
inconsistency~\cite{prob-quorum}.  Intuitively, this is a consequence
of the Birthday Paradox: as the number of replicas increases, the
probability of non-overlap between any two subsets is quite low.  To
the best of our knowledge, probabilistic quorums have only been used
to study the probability of strong consistency and have not been
employed in an eventually consistent context.

Given $n$ replicas and randomly chosen read and write quorums of sizes
$r$ and $w$, we can calculate the probability of the read quorum not
containing the value written by the write quorum.  The probability of
staleness is the number of quorums of size $r$ composed of nodes that
were not written to in the write quorum divided by the number of
possible quorums of size $r$:
\begin{equation}
\label{eq:prob-strict}
p_{stale}=\frac{{n-w \choose r}}{{n \choose r}}
\end{equation}
It is readily apparent that the probability of staleness is quite high
except for large values of $n$.  With $n=100$, $r=w=30$, $p_{stale} =
1.88 \times 10^{-6}$~\cite{nonstrict-availability}.  However, with
$n=3$, $r=w=1$, $p_{stale} = .\overline{6}$.  The asymptotics of these
systems are excellent---but only asymptotically.  

\textit{$k$-quorum systems} provide strong \textit{deterministic}
guarantees that the partial quorum system will return a value that was
written within $k$ versions of the most recent
write~\cite{nonstrict-availability}.  In the single-writer scenario,
one can imagine a round-robin write scheduling scheme where each write
is sent to $\frac{n}{K}$ replicas such that each replica is no more
than $K$ versions out-of-date.  However, with multiple writers, one
loses the ordering properties that the single-writer was able to
control, and the best known algorithm for the pathological case
results in a lower bound of $(2n-1)(k-1)+n$ versions staleness~\cite{k-quorum-lb}.

This prior work has two properties with important implications for
practitioners.  First, existing theory treats quorums as static; a
write quorum is chosen and no other replicas in the system
subsequently learn about the value unless it is written again.  The
theory does not account for propagation of versions across time and
between replicas (anti-entropic processes).  Second, much of this
prior work assumes Byzantine failure.  If the description of prior
work seemed simplistic, it is largely because most of the literature
content addresses problems such as adversarial quorum selection and
scheduling.  In this work, we re-evaluate both of these assumptions.  We
elaborate further in the next subsection.

\subsection{Quorum Foundations: Practice}
\label{sec:practice}

In practice, many distributed data management systems use quorums as a
replication strategy. Amazon's Dynamo~\cite{dynamo} is the progenitor
of a class of eventually-consistent key-value stores using a
particular variant of quorum-style replication that includes Apache
Cassandra, Basho's Riak, and LinkedIn's Voldemort\footnote{Other
  BASE-style systems may employ master-slave replication, as in Apache
  HBase~\cite{hbase}.}.  In this paper, we discuss Dynamo-style quorum
systems largely because we are not aware of any significally
different, widely adopted data systems using quorum replication.
However, with some work, we believe that other systems can adopt our
methodology.  Similarly, we focus on key-value stores as the
aforementioned systems provide some variant of key-value architecture
and do not provide full relational semantics.  Quorum systems may be
employed in RDBMS replication, but, for simplicity, we describe
key-value stores here.

Dynamo-style quorum systems employ one quorum system per key,
typically maintaining the mapping of values to a set of nodes using a
consistent-hashing scheme or a centralized membership protocol. Each
node acts as a replica for multiple keys.  Client read and write
requests are sent to a node in the system cluster, which may be
subsequently forwarded to all nodes assigned to that key as replicas.
The coordinating node considers an operation complete when it has
heard from a pre-defined number of replicas.  In Dynamo terminology,
the quorum size, or replication factor, is defined as $N$, the number
of replica responses required for a successful read is defined as $R$,
and the number of replica acknowledgements required for a successful
write is defined as $W$. Dynamo-style systems return guaranteed
strongly consistent data when $R+W > N$.  Setting $W>\lceil N/2 \rceil$ accounts for
the possibility of concurrent outstanding write requests overlapping.

For improved latency, operators often set $R+W \leq N$.  Cassandra's
default configuration is $N$$=$$3$, $R$$=$$W$$=$$1$.  Riak defaults
to $N$$=$$3$,$R$$=$$W$$=$$2$.  Voldemort does not provide sample
configurations, but Voldemort's authors at LinkedIn often choose $N$$=$$c$,
$R$$=$$W$$=$$ \lceil c/2 \rceil$ for odd $c$.  However, for applications
requiring ``very low latency and high availability'', LinkedIn uses
Voldemort with $N$$=$$3$, $R$$=$$W$$=$$1$.  For other applications, LinkedIn uses
Voldemort with $N$$=$$2$, $R$$=$$W$$=$$1$, providing ``some consistency'',
particularly when $N$$=$$3$ replication si not required.  Note that
Voldemort supports a concept of preferred reads and writes, meaning it
will block until either the preferred number of replicas respond or a
timeout occurs.  In the low latency case, preferred reads is either
two or is disabled.  In the $N$$=$$2$ case, preferred reads and preferred
writes are set to two.

\begin{figure}
\centering
\includegraphics[width=.8\columnwidth]{figs/dynamo-quorum.pdf}
\caption{Diagram of control flow for client write to Dynamo-style
  quorum.  Here, $N=3$, $W=2$. Client writes are proxied and sent to
  all replicas. The write succeeds when $W$ replicas respond.  Note
  that the proxy is likely a replica as well, invoking a local write.}
\label{fig:dynamo-quorum}
\end{figure}

Many Dynamo-style systems also support additional anti-entropic
processes.  One common feature is called ``read repair'''.  When a
read coordinator receives multiple versions of a data item from
different replicas in response to a read request, it will attempt to
(asynchronously) update the out-of-date replicas with the most recent
version.  The effect of read repair on version drift between replicas
is perhaps unsurprisingly dependent on the read rate.  The original
Dynamo paper used Merkel trees to summarize and exchange data contents
between replicas as a means of secondary anti-entropy (after the
initial write to $N$ replicas).  While open source Dynamo-style
systems use Merkel trees, not all actively employ them in gossip-based
anti-entropy.  For example, Cassandra only executes Merkel tree
anti-entropy when it is manually requested (e.g., \texttt{nodetool
  repair}).

\begin{comment}
\url{http://wiki.apache.org/cassandra/Operations#Repairing_missing_or_inconsistent_data}
\end{comment}

There are significant differences between data systems and the theory
describing quorum operation.  First, replication factors for
distributed data systems are relatively low.  Typical replication
factors are between one and three, however the systems literature has
discussed replication up to a factor of 10~\cite{chain-replication}.
Second, (in the absence of failure), in Dynamo-style partial quorums,
the number of replicas that receive a write increases even after the
operation returns.  This is a simple version of anti-entropy.
Morever, read requests are sent to all replicas, however only the
first $R$ responses are considered.  Third, these systems are often
deployed in a trusted or semi-trusted computing environment; there may
be adversarial threats against data integrity and denial-of-service
attacks, but the underlying computing hardware is
non-adversarial. Within a controlled data center, the failure modes
are certainly reduced from the Byzantine case, and, despite its many
risks, the adversarial failures covered by prior theory appear
unlikely in emerging ``cloud computing'' environments~\cite{needed}.


\section{Probabilistically Bounded\\Staleness}
\label{sec:pbs}

In this section, we introduce the theory of Probabilistically Bounded
Staleness to describe the consistency that existing eventually
consistent data stores provide.  PBS is an extension of prior work on
probabilistic quorums that accounts for staleness of both versions and
across time.  We introduce the notions of PBS $k$-quorums, which
stochastically bounds the staleness of versions returned by read
quorums, PBS $t$-visibility, which stochastically bounds the time
before a committed version appears to readers, and PBS $<k,
t>$-quorums, a combination of the two prior models.


Practical concerns guide the following theoretical contributions.  We
begin by considering a model without anti-entropic processes.
Accordingly, we assume that $w$ ($r$) of $n$ replicas are randomly
selected for each write (read) operation.  We consider a write
``committed'' once it has reached $w$ replicas and the $w$ replicas
respond.  Similarly, we consider fixed $w$, $r$ and $n$ across
multiple operations. Subsequently, we expand our model to consider
write propagation and time-varying $w$ sizes, as is typically the case
in practice.  In this section, we discuss anti-entropy in general,
however we model Dynamo-style quorums in Section
\ref{sec:dynamo-prop}. We discuss further refinements to our
assumptions in Section \ref{sec:discussion}.

\subsection{PBS $k$-quorums}

Probabilistic quorums allow us to determine the probability of
returning the most recent value written to the database, but do not
tell us what happens when the most recent value is not returned.
Here, we determine the probability of returning a value within a
bounded number of versions.  In the following model, we consider
static write quorums (no anti-entropy) and compose multiple
independent write quorums to model the probable overlap of $k$
independent write sets.
\begin{definition}
A quorum system obeys \textit{PBS $k$-quorum consistency} if, with
probability $1-p_{staler}$, at least one value in any read quorum will
have been committed no later than $k$ versions after the latest committed
version when the read begins.
\end{definition}
Versions whose writes that are not yet committed (in-flight) may be
returned by a read in this formulation of probabilistic $k$-quorums
(see Figure \ref{fig:timelines}A).  The $k$-quorum literature defines
these as $k$-regular semantics~\cite{nonstrict-availability}.

\begin{figure}
\centering
\includegraphics[width=\columnwidth]{figs/timelines.pdf}
\caption{Possible versions returned by read operations under
  PBS $k$-quorums (A) and PBS monotonic reads (B). In
  $k$-consistency, the read operation will return a version no later
  than $k$ versions older than the last committed value when it
  started; more versions may be committed during the read and may be
  returned.  In monotonic reads consistency, the staleness depends on
  the number of versions committed since the time the client last
  completed a read.  This is determined by the proportion of client's
  reads to the number of writes committed to the object.}
\label{fig:timelines}
\end{figure}

The probability of returning a version of a key within the last $k$
versions committed is equivalent to intersecting one of $k$
independent write quorums.  If we assume a random quorum choice, the
probability of non-intersection is simply Equation
\ref{eq:prob-strict} exponentiated by $k$:
\begin{equation}
\label{eq:k-consistency}
p_{staler} = \left(\frac{{n-w \choose r}}{{n \choose r}}\right)^k
\end{equation}

For the $n=3, r=w=1$ case, this means that the probability of
returning a version within $2$ versions is $.\overline{5}$, within $3$
versions $.\overline{703}$, and within $5$ versions $> .868$, and $10$
versions $>.98$.  With $n=3, r=1, w=2$ (alternatively, $r=2, w=1$),
these probabilities increase: $k=1 \rightarrow
.\overline{6}$, $k=2 \rightarrow .\overline{8}, k=5 \rightarrow >
.995$.

In the language of the probabilistic quorum literature, we have
constructed a PBS $k$-quorum from $k$ $\varepsilon$-quorum systems
where $\varepsilon \leq \sqrt[k]{p_{staler}}$. This system has load~\cite{needed}
$\geq \frac{1-\varepsilon^{\frac{1}{2k}}}{\sqrt{n}}$, an exponentially
lower bound than a strict probabilistic quorum.  This follows
immediately from~\cite[Corollary 3.12]{prob-quorum}.

This analytical closed form solution holds for static quorums.  For
dynamic quorums, this solution is an upper bound on the probability of
staleness.

\subsection{PBS Monotonic Reads}

With additional information, we can use PBS $k$-quorums to predict
whether a client will ever read staler data than it has already read.
This property, known as \textit{monotonic reads} consistency is a
well-known session guarantee~\cite{sessionguarantees}.

\begin{definition}
\label{def:prob-mr}
A quorum system obeys \textit{PBS monotonic reads consistency} if,
with probability at least $1-p_{staler}$, at least one value in any
read quorum the same version or a newer version than the client's
previously read value, where versions are defined over the global
commit ordering.
\end{definition}

To guarantee that a client sees monotonically increasing versions, it
can continue to contact the same replica~\cite{vogels-defs}, however
this is insufficient to guarantee strict monotonic reads (where the
client reads strictly newer data).  Definition~\ref{def:prob-mr} can
be adapted to acommodate strict monotonic reads (omitted for brevity).

We observe that monotonic reads is a special case of PBS
$k$-quorums where $k$ is determined by a client's rate of reads from a
data item ($\gamma_{cr}$) and the global, system-wide rate of writes
the same data item ($\gamma_{gw}$).  If we know these rates
exactly, the number of versions between client reads is
$\frac{\gamma_{gw}}{\gamma_{cr}}$, as shown in Figure
\ref{fig:timelines}B.  We can calculate the probability of
probabilistic monotonic reads as follows, effectively using Equation
\ref{eq:k-consistency} where $k=1+\frac{\gamma_{gw}}{\gamma_{cr}}$:

\begin{equation}
\label{eq:prob-mr}
p_{staler} = \left(\frac{{n-w \choose r}}{{n \choose r}}\right)^{1+\gamma_{gw}/\gamma_{cr}}
\end{equation}
For strict monotonic reads, exponentiate where
$k=\frac{\gamma_{gw}}{\gamma_{cr}}$.  When constructed from $k$
$\varepsilon$-consistent systems (as we have here), this consistency
model has load $\geq \frac{(1-p_{staler}^{\frac{1}{2C}})}{\sqrt{n}}$,
where $C=1+\frac{\gamma_{gw}}{\gamma_{cr}}$.

In practice, we may not know these exact rates, but, by measuring
their distribution, we can calculate an expected ratio that we can
integrate into these calculations.  By performing appropriate
admission control, operators can control these rates to achieve
monotonic reads guarantees.

\subsection{PBS $t$-visibility}

Until now, we have considered only static write quorums.  However, as
we discussed in Section \ref{sec:practice}, modern quorum systems in
practice are frequently dynamic, and writes propagate to quorum system
members over time.  This process is commonly known as anti-entropy.
For generality, in this section, we will discuss general anti-entropy
models. However, we explicitly model the Dynamo-style anti-entropy
mechanism in Section \ref{sec:dynamo-prop}.

PBS $t$-visibility models the probability of (strong) inconsistency,
accounting for the propagation of writes across wall-clock or
real-world time such that the set of replicas with a given version of
the data (or later) grows over time.  Node failures shrink this set
and can be incorporated into this model, but we do not consider them
here.  Intuitively, $t$-visibility captures the possibility that a
reader will observe a write after $t$ seconds after it commits.
Recall that we consider in-flight writes---which are more recent than
the last committed version---as non-stale.

\begin{definition}
A quorum system obeys \textit{PBS $t$-visibility consistency} if, with
probability $1-p_{staler}$, any read quorum started at least $t$ units
of time after the last version committed returns at least one value
that is at least as recent as the last version committed when the read
began (and may not be committed yet).
\end{definition}

We denote the cumulative density function describing the number
of replicas $\mathcal{W}$ that have a particular version $v$ (or a
version committed after $v$) $t$ seconds after committing as
$P_w(\mathcal{W}, t)$.  This assumes that writes obey a total
ordering. This can be accomplished using synchronized clocks (last
writer wins) or with causal ordering and commutative merge
functions~\cite{cops}.

By definition, $P_w(c,0) = 1$ $\forall c \in [0, w]$.  Intuitively, at
commit time, $w$ replicas will have the value, so the probability that
zero to $w$ replicas have the value immediately after commit is
exactly $1$.  We can model the probability of PBS $t$-visibility for
an interval $t$ by summing the conditional probabilities of each
possible $\mathcal{W}$:

\begin{equation}
\label{eq:tv-instantreads}
p_{staler} = P_l(w, t)+\sum_{c\in(w, n]} \frac{{n-c \choose r}}{{n \choose r}}\cdot [P_w(c+1, t)-P_w(c,t)]
\end{equation}

However, the above equation assumes reads occur instantaneously.  In
the real world, reads must be sent to replicas and therefore more time
will elapse between write commit and $t$.  Accordingly,
Equation~\ref{eq:tv-instantreads} is a conservative upper bound on
$t$-staleness.  Additionally, depending on the dependence assumptions
regarding read latencies and $P_w$ latencies, one can potentially
model both transient and permanent failures by increasing the tail
probabilities for operation latencies.

In practice, $P_l$ depends on the anti-entropy protocols and the
expected latency of operations and can be approximated analytically
(Section \ref{sec:dynamo-prop} or measured online.  For this reason,
the load of a PBS $t$-visible quorum system depends on write
propagation and is difficult to determine for general-purpose dynamic
quorums.


\subsection{PBS $\langle k, t
  \rangle$-quorum Consistency}

We can combine the previous models to combine both versioned and
real-time staleness metrics to answer questions of the following form:
what is the probability that a read will return a value no later than
$k$ versions old if the last write committed no sooner than $t$ seconds
ago?
\begin{definition}
A quorum system obeys \textit{probabilistic $\langle k, t
  \rangle$-quorum consistency} if, with probability $1-p_{staler}$, at
least one value in any read quorum will have been committed no later
than $k$ versions after the latest committed version when the read
begins, provided the read begins $t$ units of time after the previous
$k$ versions commit.
\end{definition}
The definition of $p_{staler}$ follows from the prior definitions:
\begin{equation}
p_{staler} = (P_l(w, t)+\sum_{c\in[w, n)} \frac{{n-c \choose r}}{{n \choose r}} \cdot [P_w(c+1, t)-P_w(c,t)])^k
\end{equation}
In this equation, in addition to (again) assuming instanteous reads,
we also assume the pathological case where the last $k$ writes all
occurred at the same time.  This is not likely in practice, so if we
can determine the time since commit for the last $k$ writes, we can
improve this staleness bound by considering each quorum's $p_{staler}$
at $RT=t$ separately.  However, this equation provides a conservative
upper bound on $p_{staler}$.

Note that the prior definitions of consistency are encapsulated by
probabilistic $\langle k, t \rangle$-quorum consistency. probabilistic
$k$-quorum consistency is simply probabilistic $\langle k, 0
\rangle$-quorum consistency, probabilistic monotonic reads consistency
is $\langle 1+\frac{\gamma_{gw}}{\gamma_{cr}}, 0 \rangle$-quorum
consistency, and $t$-visibility is $\langle 0, RT \rangle$-quorum
consistency.

This consistency model is conceptually difficult to reason about.
Accordingly, for the sake of practicality, other than acknowledging
its existence, we focus mainly on deriving better models for
$t$-visibility.

\section{Dynamo-style $t$-visibility}
\label{sec:dynamo}

We have formulated a closed-form analytical model for $k$-staleness,
but $t$-visibility is dependent on both the distributed quorum
execution algorithm and the anti-entropy protocols employed by a data
storage system.  In this section, we discuss PBS $t$-visibility in the
context of Dynamo-style data storage systems.  We describe how to
model the probability of staleness in these systems, how to predict
the latency of a given operation, and how to asynchronously detect
staleness.

\subsection{Inconsistency in Dynamo}

Dynamo-style quorum systems are inconsistent as a result of read and
write message reordering.  Figure~\ref{fig:dynamostale} shows a
space-time diagram for messages between a sender and a single replica.
When reads and writes are executed, they are sent to all $N$


\begin{figure}
\centering
\includegraphics[width=.8\columnwidth]{figs/dynamostale.pdf}
\caption{The $WARS$ model for message ordering in Dynamo describes the
  message flow between a coordinator and a single replica for a write
  followed by a read $t$ seconds after commit.  Note that, in an $N$
  replica system, this message flow is replicated $N$ times between
  the coordinator and $N$ replicas.  Note that the read and write may be handled by different coordinators.}
\label{fig:dynamo-diagram}
\end{figure}




\subsection{Dynamo-style Operation Latency}

\label{sec:dynamo-prop}



Until now, we have assumed that the expected latency of operations
reaching a given number of replicas by a particular time
($p_w(\mathcal{W}, t)$ for writes) in the case of writes is known.  We
can measure this distribution empirically (Section
\ref{sec:real-latency}), but we can also model this latency
analytically, given a few assumptions.

To begin, we consider write latency.  With Dynamo-style queries, we
want to determine the probability that $w$ of the replicas in the
write quorum $\mathcal{Q}$ respond within time $t$.  We can determine
this in at least two ways.  One way is to find use order statistic
theory to determine the probability that the $w$th replica reasponds
by time $t$.  However, this is equivalent to taking the minimum
response time over the maximum of each possible set of $w$ replicas
from $\mathcal{Q}$, which works out more cleanly in this formulation.
Given the write latency cumulative density functions for a set of
nodes $S$, $L_{ws}(S, t)$, the probability density function $p_w$ of
write latency of a set of nodes is:
\begin{equation}
p_w(\mathcal{W}, t) = min(\{L_{ws}(S, t) \mid S \subseteq \mathcal{Q}, |S| = \mathcal{W}\})
\end{equation}


$\mathcal{W}$, enoted $WQ_l(\mathcal{W})$, we simply need to consider
the conditional probability of $\mathcal{W}$ writes completing across
all possible time values $t$.  However, this depends on the
distribution of $L_{ws}(S, t)$, which requires some
assumptions. Determining the ordinal probability of a set of randomly
distributed variables (in our case, the minimum of all $L_{ws}(S,
t)$), is possible but is also fairly complicated.  The difficult
arises in determining the distribution of a given $L_{ws}(S)$.  If we
assume that all nodes in $S$ obey independent latency distributions,
this is tractible.  If we assume that they are dependent and therefore
obey a joint distribution, then, as $w$ grows, solving for this
probability becomes much more difficult~\cite{needed}.  Approximation
algorithms can assist here~\cite{needed}, but we can also make
simplifying assumptions about the distribution of these latencies.

If we assume that latencies are independently, identically distributed
(IID), that is, if each node obeys the same latency distribution and
each nodes latency is independent of the other nodes's latency, this
equation is greatly simplified.  Under IID assumptions, the time for a
write to reach every node in the quorum obeys a single latency
CDF\footnote{Note that this distribution only captures the request
  forward, not the round-trip time before the acknowledgement.  The
  replica can serve the data before it responds to the proxy.},
$L_w(t)$ (PDF, resp. $l_w(t)$).  The probability that $w$ nodes all
respond within time $t$ is simply $(L_w(t))^\mathcal{W}$, and the
minimum over all such of these sets is the CDF $P_w$:
\begin{equation}
P_w(\mathcal{W}, t) = 1-(1-(L_w(t))^\mathcal{W})^{n \choose \mathcal{W}}
\end{equation}
This is indeed simplifying, but it makes the theory much simpler.  We
discuss the validity of this IID assumption when we measure latencies
experimentally in Section \ref{sec:real-latency}.

Under the IID assumption, we can easily determine the expected
Dynamo-style write quorum operation latency:
\begin{equation}
WQ_l(\mathcal{W}) = \int_0^{\infty} t \cdot p_w(\mathcal{W}, t) dt
\end{equation}
In practice, the proxy is often a replica as well, so for a write
quorum of size $w$, we only need to consider $\mathcal{W}=w-1$ writes.

Similarly, to determine how many replicas have a particular value (even if they have not acked) after a write quorum of size $w$, we evaluate:
\begin{equation}
P_{w:have}(\mathcal{W}, t) = 1-(1-(L_w(t))^{\mathcal{W}-w})^{n-w \choose \mathcal{W}-w}
\end{equation}

Thus far, we have only considered the write operation latencies.  The
read quorum latency may be different from the write latency depending
on node-level actions such as forced logging or disk-bound operations.
However, the expected read operation latency for a quorum of size
$\mathcal{R}$, $RQ_l(\mathcal{R})$, can be calculated similarly given a
model for the latency of reads ($L_{rs}(S,t)$ and $L_r(t))$.

\subsection{Optimization Formulation}

With these equations, we can optimize the selection of a partial
quorum system to minimize overall operation latency subject to
constraints on staleness. We can guarantee bounded staleness by
ensuring that $p_{staler} = 1$, however this is only possible in
$k$-quorum consistency as the real-time components of $RT$-quorum
consistency and its variants are inherently probabilistic (unless we
can prove an absolute bound on operation latency).  We can use a
simply formulated optimization program to determine what combination
of $r$ and $w$ minimizes latency while satisfying the system
constraints and user requirements.

Given $n$, desired $p$, $w_{min}$ (minimum durability of writes),
consistency model (with necessary parameters--$k$, $t$, $\gamma_{cr}$,
etc.), and the relative weighted ``importances'' of read
and write latency, $c_r$ and $c_w$, we can determine $r$ and $w$:

\begin{equation}
 \begin{array}{rl}
    \min        & c_r\cdot RQ_l(r) +c_w \cdot WQ_l(w) \\
    \mbox{s.t.} & p \ge p_{staler} \\
                & w \ge w_{min}.
    \end{array}
\end{equation}

We implement and validate this optimization framework in
Section~\ref{sec:optimization}.

\subsection{Typical Quorum Configurations}

CASSANDA: R=1, W=1 

<<<<<<< HEAD
RIAK: N=3, R,W QUORUM 
=======
RIAK: N=3, R,W QUORUM \url{https://github.com/basho/riak_kv/blob/1.0/src/riak_kv_app.erl}
\url{http://wiki.basho.com/Riak-Glossary.html#Quorum}
>>>>>>> 11fb27c39c7f11f1b544d903fd16543cac89854b

Voldemort: does not provide production configs besides testing (N=2)

\section{Experimental Evaluation}
\label{sec:dynamoeval}

We evaluated Dynamo-style partial quorums through the lens of PBS.  We
first implemented PBS policies on top of an open source Dynamo clone,
Cassandra, and compared our theory to observed results in a cloud
computing environment.  We subsequently predicted and measured and
predicted the PBS $k$-quorum and $t$-visibility consisistency across
several environmental configurations.

\subsection{Implementation}

We implemented PBS policies on top of Cassandra, an open source data
management system.  Cassandra provides Dynamo-style partial quorum
semantics, offering per-request quorum sizes and a BigTable-like
schemas~\cite{needed}.  Our changes to Cassandra were minimal: we
added support for partial quorums of size greater than
three---requiring changes to the wire protocol and request handling
code---and instrumented parts of the database to provide timing
information about reads and writes.  By analyzing Cassandra logs, we
were able to reconstruct the latency distributions of write requests,
write acknowledgements, read requests, and read responses.

READ REPAIR: DEPENDENT ON READ AND WRITE

We disabled read repair in Cassandra.  When a read quorum returns
divergent versions, we return the highest-valued version and do not
attempt read repair, essentially writing the new value back to the
stale replicas.  For a read request, Cassandra sends out a read
request to one node and ``digest'' requests for hashes of data items
to $R-1$ replicas on a read request.  If there is a mismatch between
any digest and the value read, Cassandra begins read repair, incurring
another set of round trips between the read coordinator and the
replicas.  We disable this feature and send out vanilla read requests
to all replicas instead.  Accordingly, we overestimate the staleness
of versions returned.

We implemented a staleness predictor using Monte Carlo simulation.
The closed-form solution is tractable, but there are several reasons
why we preferred simulation.  Simulating the message delays is simple.
One simply has to draw from each distribution and compare latencies to
determine the degree of staleness.  The Dynamo model is easily
implemented in an event-based simulation.  The analytical model is
much more complicated.  We found that implementing the analytical
model is rather onerous, requiring a large amount of numerical
integration and difficult code.  This is potentially a barrier for
adoption for practitioners.  Finally, when operating over discrete,
experimentally-gathered distributions (without curve-fitting), Monte
Carlo simulation was significantly faster.  Accordingly, our predicted
results here are derived from repeated simulation trials.

\subsection{Model Validation}

We deployed Cassandra on Amazon EC2 and compared the observed
staleness to our predicted staleness for a number of configurations.
To provide a single point of order for the series of reads and writes
in the system, we deployed $N+1$ Cassandra nodes for each experiment
involving $N$ nodes.  After configuring the Cassandra schema, we
determined the node that did not store the data and sent all requests
through it, effectively creating a remote proxy.  Our model is more
general than the specific proxy configuration, however using a
centralized proxy allows us to avoid most issues with clock skew
between replicas and rely less heavily on globally synchronized
clocks.

We used \texttt{m1.small} instances throughout our experiments within
the \texttt{us-east-1} availability zone.  To model network and remote
processing delays, we injected delays into both the request and
response sending modules in Cassandra.  We discuss our choice of
distributions and their effects in Subsection~\ref{subsec:delay}.  One
major challenge in performing our experiments was the possibility of
clock skew across replicas; we relied on both \texttt{ntp} and
synchronization between the Xen hypervisors running our virtual
machines, however this resulted in skew of several milliseconds at
worst.  We ran a single reader and a single writer to a monotonically
increasing key in Cassandra and compare the read and write timestamps
to predict the staleness.  This provided a strong control group, but
low load.  To simulate load and network delay, we injected additional
delays into the distributions of $W$,$A$,$R$,$S$ (more in Subsection
~\ref{sec:latencies}).

Across multiple latency profiles, we were able to predict staleness
with extremely high accuracy.  Using the operation delays recorded in
Cassandra across 5000 writes, we reconstructed the latency
distributions and repeated each experiment in simulation.  On average,
our Monte Carlo simulations predicted $<k,t>$ staleness within XX\% of
the observed staleness.  As shown in Figure \ref{fig:fits}, our
simulation has no greater than YY\% inaccuracy for any given $<k,t>$
staleness prediction.  We attribute this inaccuracy to sample size.

Intuitively, these figures make sense.  The time-sensitive components
of PBS are dependent on $W$,$A$,$R$, and $S$ alone.  Provided we can
model these distributions accurately, the underlying system is
irrelevant.  Our simulator is general; it models Dynamo-style
operation delay, not a particular data storage system's
internals---these are irrelevant to PBS insofar as they are not
related to write propagation.  Accordingly, and for greater control,
we chose to evaluate the remainder of our experiments in simulation.
As we have show in Appendix A, PBS consistency is a mathematical
concept, and, by construction, is isolated from non-$WARS$ system
implementation variations---provided the system obeys the fundamental
partial quorum semantics being modeled.

\subsection{Latency Models}
\label{sec:latencies}




\begin{figure*}
\centering
\subfigure{\includegraphics[width=\columnwidth]{figs/latlegend.pdf}}\\[-1mm]
\subfigure{\includegraphics[width=.6\columnwidth]{figs/readlats-1.pdf}}
\subfigure{\includegraphics[width=.6\columnwidth]{figs/readlats-2.pdf}}
\subfigure{\includegraphics[width=.6\columnwidth]{figs/readlats-3.pdf}}
\subfigure{\includegraphics[width=.6\columnwidth]{figs/writelats-1.pdf}}
\subfigure{\includegraphics[width=.6\columnwidth]{figs/writelats-2.pdf}}
\subfigure{\includegraphics[width=.6\columnwidth]{figs/writelats-3.pdf}}
\caption{FOOBAR}
\label{fig:FOO}
\end{figure*}

\begin{figure*}
\centering
\subfigure{\includegraphics[width=\columnwidth]{figs/latlegend.pdf}}\\[-1mm]
\subfigure{\includegraphics[width=.65\columnwidth]{figs/tstales-3N1R1W.pdf}}
\subfigure{\includegraphics[width=.65\columnwidth]{figs/tstales-3N2R1W.pdf}}
\subfigure{\includegraphics[width=.65\columnwidth]{figs/tstales-3N1R2W.pdf}}
\caption{FOOBAR}
\label{fig:FOO}
\end{figure*}

\begin{figure*}
\centering
\subfigure{\includegraphics[width=.65\columnwidth]{figs/sweepn-LNKD-DISK.pdf}}
\subfigure{\includegraphics[width=.65\columnwidth]{figs/sweepn-LNKD-SSD.pdf}}
\subfigure{\includegraphics[width=.65\columnwidth]{figs/sweepn-YMMR.pdf}}
\caption{FOOBAR}
\label{fig:FOO}
\end{figure*}

\begin{figure}
\centering
\includegraphics[width=\columnwidth]{figs/rwratio.pdf}
\caption{FOOBAR}
\label{fig:FOO}
\end{figure}

\subsection{Microbenchmarks}

We issued reads and writes to a single key distributed across a
Cassandra cluster and measured the observed staleness.  While we ran
multiple configurations, to begin, we will discuss the configuration
where $N=3$, $R=1$, and $W=1$, Cassandra's default settings.

With a single reader and a single writer, under normal operation, we
observed little to no staleness.  Reads completed in 2.59 ms (99th
percentile: 30.61 ms) and writes completed in 2.34 ms (99th
percentile: 22.36 ms).  This was too narrow a window to observe the
message reordering required for any staleness.  However, our experiment is somewhat synthetic



\subsection{Worst-case Analysis}

\subsection{Latency and Staleness}

\subsection{Configuration Optimization}

\section{Discussion}
\label{sec:discussion}

There are several aspects of distributed data systems that we have not yet
addressed.  Here, we briefly discuss improvements to our models.

\subsection{Is Eventually Consistent Good Enough?}

One Cassandra committer noted that although he had a patch for session guarantees, no Cassandra users had requested the functionality or claimed to require it.

\subsection{Additional Functionality}

READ-MODIFY-WRITE

\textbf{Staleness detection.} Each of the $p_{staler}$ bounds
describes a probability that a particular key is no staler than
specified.  In the context of probabilistic consistency SLAs, it would
be useful to determine whether a key is staler than promised by a
given SLA.  Determining this fact on-line is tantamount to achieving
strong consistency, however there are at least two strategies for
mitigating this difficulty. First, additional gossiping such as
regular negative acknowledgements can notify nodes of potential
version staleness.  This strategy has been discussed in the
literature~\cite{tocite} for non-probabilistic quorum systems.
Second, the system can provide asychronous notifications of staleness
information.  Proxies can keep a growing log of versions which, with
appropriate timestamping, can be used to determine if previously
returned values were staler than promised.  This is fairly
straightforward but requires careful garbage collection to manage
proxy-side state as the number of client requests scales.
Additionally, the possibility of proxy failure requires that prior
operations be durably logged; however, provided the proxy eventually
comes online and updates its log, the asynchronous guarantee holds.
Both of these schemes are feasible yet add significant complexity to a
probabilistically consistent quorum system.

\textbf{Multi-key transactions.} We have considered single-key operations,
however the ability to perform distributed transactions is potentially
attractive.  For read-only transactions, if the key distribution is
random, each quorum is independent, so we can simply multiple the
staleness probabilities of each key.  Achieving atomicity of writes to
multiple keys requires more complicated coordination mechanisms such
as two-phase commit.  Again, transactions are feasible but require
considerable care in implementation, complicated what is otherwise a
simple replication scheme.

\textbf{Node failures.} Fail-stop node failures can be easily
incorporated into our latency models. The probability of a node
permanently failing $p_{f-perm}$ can be represented by setting the
probability of infinitely long read and write
operations. $L_{w}(\inf)=L_{r}(\inf)=p_{f-perm}$.  Intermittent node
failures of length $t_{fail}$ can be represented by setting
$L_{w}(t_{fail})$ to the probability of the node failing.  In this
way, operation latency captures all expected failure semantics.

\textbf{Read repair and active anti-entropy.} Modern Dynamo-style
systems use a technique known as \textit{read repair} to reconcile
divergent versions at query-time.  If two nodes return different
responses to a \textit{get} request, the responses are merged and
forwarded to the determined out-of-date node.  Dynamo performs
additional anti-entropy using Merkel trees, however the open source
databases we have examined only perform read repair (CITATION
needed--at least Cassandra does this). We do not model read repair or
additional anti-entropy processes here (although doing so requires
only minimal changes to our model), so we overestimate the chance of
staleness violation.

\textbf{Variable $w$.} We have assumed the use of a single $w$ across
all writes.  However, many KVSs such as Cassandra and Riak allow the
use of per-operation consistency\footnote{Voldemort specifies
  consistency requirements at the schema level. This is ostensibly an
  engineering decision that could be changed with minimal
  modifications to the wire protocol.}

\textbf{99th Percentile Latency.}


\section{Related Work}
\label{sec:relatedwork}

Theory

Vadhat

\section{Conclusion}
\label{sec:conclusion}

\begin{comment}
\section*{Acknowledgements}

Alex Feinberg
Coda Hale

Andy Gross
Justin Sheehy

Doug Terry

Sam Madden
Christopher Re
Marcos Aguilera

Eric Brewer

Greg Durrett
Greg Valiant


Ali Ghodsi

Peter Alvaro
Bill Marczak
Neil Conway
Hariyadi Gunawi

Bryan Kate

Patrick Wendell
Kay Ousterhout

\end{comment}

\balance

\bibliographystyle{abbrv}
\bibliography{ernst}

\begin{comment}
\begin{appendix}
\section{Dynamo-style Staleness}

We present an analytical closed-form solution to $p_{staler}$ for
$t$-visible partial quorums unde Dynamo-style.  What is the
probability that a read result is not strongly consistent (that is,
does not return the most recent value when the read began)? In this
section, when we refer to the ``last committed version'', we are
referring to the last committed version before the read began. We
consider the pathological case where a read occurs immediately after a
write occurs.

We assume that each writes to a replica is independently delayed by a
randomly distributed variate $W$, each write acknowledgement from a
replica to a writer is independently delayed by a randomly distributed
variate $A$, each read request to a replica is independently delayed
by a randomly distributed variate $R$, and each read response from a
replica to a reader is delayed according to a randomly distributed
variate $S$.  We make no assumptions as to the distribution of $W,$
$A,$ $R,$ and $S$, except that their domain is strictly positive (no
negative delay) and that values drawn from each are independently,
identically distributed (IID).

The probability that a read quorum does not contain the last committed
version is equivalent to the probability that each of the first $R$
read responses are all stale.  Denote the probability that the $i$th
response is stale as $stale_i$:
\begin{equation*}
p_{staler} = \prod_{i=0}^{R} \Pr(stale_i)
\end{equation*}

Denote the return time of the $i$th read response as $S_i$.  The probability that the $i$th response is stale is given by:
\begin{equation*}
\Pr(stale_i) = \int_{0}^{\infty} \Pr(S_i = t) \cdot \Pr(stale_i \mid S_i = t) dt
\end{equation*}

We can use simple order statistics to calculate the return time of the $i$th read response:
\begin{multline*}
\Pr(S_i = t) = \frac{N!}{(i-1)!(N-i)!}\cdot[\Pr(R+S< t)]^{i-1}\\\cdot\Pr(R+S = t)\cdot[\Pr(R+S > t)]^{N-i}
\end{multline*}

 Recall that under Dynamo-style messaging, read response from a
 replica is stale if the read request arrived at the replica before
 the write command for the last committed version reached the replica.
 Denote the time required to commit the last version (alternatively, the time required for $W$ of $N$ replicas to acknowledge the last write) as $w_t$.
\begin{multline*}
\Pr(stale_i \mid S_i = t) = \int_{0}^{t} \Pr(R=\tau)\cdot\Pr(S=t-\tau)\\\cdot\Pr(R+w_t < W \mid R = \tau) d\tau
\end{multline*}
\begin{multline*}
  \Pr(R+w_t < W \mid R = \tau) = \int_{\tau}^{\infty} \Pr(W-w_t = x) dx
\end{multline*}
\begin{multline*}
  \Pr(W-w_t = z) = \int_{0}^{\infty} \Pr(W = y) \cdot \Pr(-w_t = z-y) dy
\end{multline*}
Finally, to calculate $w_t$, we calculate the probabilty that
$\mathcal{W}$ of $N$ acknowledgements come in at time $t$:
\begin{multline*}
\Pr(w_t = t) = \frac{N!}{(\mathcal{W}-1)!(N-\mathcal{W})!}\cdot[\Pr(W+A< t)]^{\mathcal{W}-1}\\\cdot\Pr(W+A = t)\cdot[\Pr(W+A > t)]^{N-\mathcal{W}}
\end{multline*}
Combining these equations, we have a closed-form solution for the
staleness dependent on $W,$ $A,$ $R,$ and $S$, as desired.  To consider
multiple version staleness, exponentiate $p_s$ by the number of
versions stale.

\end{appendix}
\end{comment}

\end{document}

