\documentclass{vldb}
\usepackage{graphicx, subfigure, multirow, times}
\usepackage{balance, url, amsfonts, verbatim, mathtools, draftwatermarktop, draftwatermarkbottom}  % for  \balance command ON LAST PAGE  (only there!)

    \renewcommand{\topfraction}{1}	% max fraction of floats at top
    \renewcommand{\bottomfraction}{1}	% max fraction of floats at bottom
    %   Parameters for TEXT pages (not float pages):
    \renewcommand{\dbltopfraction}{1}	% fit big float above 2-col. text
    \renewcommand{\textfraction}{0.5}	% allow minimal text w. figs
    %   Parameters for FLOAT pages (not text pages):
    \renewcommand{\floatpagefraction}{0.7}	% require fuller float pages
	% N.B.: floatpagefraction MUST be less than topfraction !!
    \renewcommand{\dblfloatpagefraction}{0.7}	% require fuller float pages

  \renewcommand{\ttdefault}{cmtt}


\usepackage{mdwlist}

\title{Probabilistically Bounded Staleness and\\ Practical Partial Quorum Systems}

\author{Peter Bailis, Shivaram Venkataraman, Michael J. Franklin, Ion Stoica, Joseph M. Hellerstein\\
\affaddr{University of California, Berkeley}\\
\affaddr{\{pbailis, shivaram, franklin, stoica, hellerstein\}@cs.berkeley.edu}}

\newdef{definition}{Definition}

%%% This file is generated by Makefile.
%%% Do not edit this file!\n%%%
		\gdef\GITAbrHash{8fa3b67}		\gdef\GITAuthorDate{Thu Dec 22 06:20:23 2011 -0800}		\gdef\GITAuthorName{Peter Bailis}

\setlength{\dbltextfloatsep}{1em}
\setlength{\dblfloatsep}{1.5em}

\begin{document}

\interfootnotelinepenalty=10000
\hyphenation{prob-a-bil-is-tic-ally}

\maketitle


\noindent\textcolor{red}{Base revision~\GITAbrHash,~\GITAuthorDate\\\GITAuthorName.}

\begin{quote}
\textit{All good ideas arrive by chance.}---Max Ernst
\end{quote}


\begin{abstract}

Modern storage systems employing quorum replication are often
configured to use partial, or non-strict quorums.  These systems wait
only for a subset of their replicas to respond to a request before
returning an answer without guaranteeing that read and write sets
overlap.  While these real-world partial quorum mechanisms provide
only basic eventual consistency guarantees, these configurations are
frequently ``good enough'' for practitioners given their latency
benefits. In this work, we discuss why partial quorums are often
acceptable in practice by analyzing the staleness of data they return.
Extending prior work on strongly consistent probabilistic quorums and
using the models of anti-entropic Dynamo-style processes, we introduce
Probabilistically Bounded Staleness (PBS) consistency, which provides
expectations of bounds on staleness across both versions and wall
clock time.  We derive a closed-form solution for versioned staleness
and model real-time staleness for representative Dynamo-style systems
under internet-scale production workloads.  We quantitatively
demonstrate why, in practice, systems employing non-strict quorums
rarely serve inconsistent data.

\end{abstract}


\section{Introduction}

Modern distributed storage systems need to be scalable, highly
available, and fast.  These systems typically replicate data across
different machines and often across datacenters for two reasons:
first, to provide high availability when components fail and, second,
to provide improved performance by serving requests from multiple
replicas.  In order to provide predictably low read and write latency,
systems often eschew protocols guaranteeing consistency of reads, and
instead opt for eventually consistent
protocols~\cite{cassandradefault, abadilatconsist, dynamo, feinbergpc,
  reddit, riaktalkone, outbrain}.  However, weak eventually consistent
systems make no guarantees on the staleness, or recency in terms of
versions written, of data items returned, other than that the system
will ``eventually'' return the most recent version in the absence of
new writes~\cite{vogels-defs}.

Distributed quorums can be used to ensure strong consistency across
multiple replicas of a data item by overlapping read and write
sets. However, waiting for responses from the potentially large
resulting quorum sizes increases operation latency, an important
consideration for service operators. For example, at Amazon, 100 ms of
additional latency resulted in a 1\% drop in
sales~\cite{amazon-latency}, while 500 ms of additional latency in
Google's search product resulted in a corresponding 20\% decrease in
traffic~\cite{google-talk}.  At scale, these decreases correspond to
large amounts of lost revenue.

Employing \textit{partial} or non-strict quorums lowers operation
latency in quorum replication.  Under partial quorums, sets of nodes
written to and read from are not guaranteed to overlap: given $N$
replicas and read and write quorum sizes $R$ and $W$, partial quorums
imply $R+W \leq N$.  Modern quorum-based data systems such as
Dynamo~\cite{dynamo} (and its open source descendants
Cassandra~\cite{cassandra}, Riak~\cite{riak}, and
Voldemort~\cite{voldemort}) offer a choice between these two modes of
quorum replication: overlapping quorums, providing strong consistency,
and variable-sized partial quorums, providing eventual consistency.

Despite the weak guarantees eventual
consistency provides~\cite{hamilton-cap, cops, walter}, operators frequently
employ partial quorums~\cite{cassandra, cassandra-docs,
  cassandradefault,feinbergpc,reddit, outbrain, maxperfblog}---they
are often ``good enough'' for applications, given their latency
benefits, which are especially important as latencies grow (e.g., a
wide-area network scenario)~\cite{abadilatconsist, feinbergpc}.  Given
that many programs can handle staleness resulting from eventual
consistency through the use of associative and commutative operations
(e.g., timelines, logs, and notifications)~\cite{calm, helland}, these
trade-offs are often justified.  Most importantly, the proliferation
of partial quorum configurations suggests that either data store
operators are simply na\"{\i}ve and misguided or, more likely, in
practice, eventually consistent data stores provide a reasonable
degree of consistency.

Is ``weakly eventually consistent'' synonymous with ``no hope for
consistency''? The basic semantics of eventual consistency dictate
that this may be the case, but practice suggests otherwise. Under
worst-case conditions, eventual consistency results in an unbounded
degree of data staleness, but, as we will show, the average case is
frequently different.  Eventually consistent data stores cannot
promise consistency, but they can describe what consistency they are
\textit{likely} to provide for varying degrees of certainty.  However,
there is little prior work that describes how to make these
consistency and staleness predictions under practical conditions.
Accordingly, the current state-of-the-art requires data system users
to make rough guesses or perform online profiling to determine the
consistency provided by their data stores.

To do better, we need to know when and why eventually consistent
systems return stale data as well as the staleness of the data they
return.  In this work, we answer these questions in the context of
quorum replicated data stores by expanding theoretical research on
\textit{probabilistic quorums}~\cite{prob-quorum, quorum-overview} to
account for multi-version staleness, anti-entropy, and quorum
dissemination protocols as found in the wild.  More precisely, we
present algorithms and models for accurately predicting the staleness
of partial quorums across multiple versions and wall clock time,
called Probabilistically Bounded Staleness (PBS) for partial
quorums. PBS can be used to determine the probability of reading one
of the last $k$ versions of a data item ($k$-staleness), of reading a
data item $t$ seconds after it is written ($t$-visibility), and of
experiencing a combination of the two ($\langle k, t
\rangle$-staleness). PBS does not enforce deterministic staleness
bounds~\cite{ aqua, trapp,vahdat-article, vahdat-bounded, frac} but is
instead intended as a lens for analyzing and potentially improving
\textit{existing} systems.

We provide closed-form solutions for PBS $k$-staleness and use Monte
Carlo analysis to explore the trade-off between latency and
$t$-visibility.  We present a detailed study of Dynamo-style PBS
$t$-visibility using production latency distributions to explain why
real-world partial quorum systems operators frequently choose to
employ partial quorums. For example, at $99.9$th percentile confidence
and combined read and write latency, for one production workload, we
observe a $59.5\%$ latency improvement ($3.26$ to $1.32$ ms) in
exchange for a $1.58$ ms window of inconsistency using solid state
drives and a $16.5\%$ latency improvement ($15.01$ to $12.53$ ms) in
exchange for a $13.6$ ms window of inconsistency using spinning disks.
Under another workload, with a long-tailed latency distribution, we
observe a $81.1\%$ latency improvement ($230$ to $43.3$ ms) for a
$202$ ms window of inconsistency.  While the benefit of these
trade-offs is application-specific, our analysis demonstrates the
performance benefits many operators opt for over strong consistency.

We make the following contributions in this paper:

\begin{itemize*}

\item We develop the theory of Probabilistically Bounded Staleness
  (PBS) for partial quorums. PBS describes the probability of
  staleness across both versions ($k$-staleness) and time
  ($t$-visibility) as well as monotonic reads consistency.

\item We provide a closed-form analysis of $k$-staleness demonstrating
  how the probability of receiving data $k$ versions stale is
  exponential in $k$.  As a corollary, $k$-staleness tolerance also
  exponentially lowers quorum system \textit{load}.

\item We provide a model for $t$-visibility in
  Dynamo-style partial quorum systems, \textit{WARS},  showing how
  staleness is dependent on message reordering in the system.  We
  evaluate the $t$-visibility of Dynamo-style systems using a
  combination of experimentally-gathered data and production traces.

\end{itemize*}

The remainder of this paper is as follows: in
Section~\ref{sec:background}, we provide an overview of quorum systems
and Dynamo-style quorums.  In Section~\ref{sec:pbs}, we introduce
Probabilistically Bounded Staleness and provide a closed-form solution
to $k$-staleness under traditional static quorums.  In
Section~\ref{sec:dynamo}, we model Dynamo-style quorum systems and
discuss when and why staleness occurs.  In Section~\ref{sec:dynamoeval}, we
use real-world workloads and Monte Carlo analysis to determine how
often Dynamo-style quorums return stale values in practice.  In
Section~\ref{sec:discussion}, we describe further improvements to
PBS partial quorums. In Section~\ref{sec:relatedwork}, we discuss
related work, and, in Section~\ref{sec:conclusion}, we conclude.


\section{Background}
\label{sec:background}

In this section, we provide background regarding quorum systems both
in theoretical academic literature and in practice.  We begin by
overviewing work on traditional and probabilistic quorum systems.  We
next discuss Dynamo-style quorums, currently the most widely deployed
replication strategy for commercially available data storage systems
employing quorum replication.  Finally, we survey reports of
practitioner usage of partial quorums for three Dynamo-style data
systems.

\subsection{Quorum Foundations: Theory}

Quorum systems have long been proposed as a replication strategy for
distributed data storage~\cite{quorums-start}.  Under quorum
replication, the data storage system writes a data item by sending it
to a set of replicas, called a write quorum.  To serve reads, the data
system fetches the data from a possibly different set of replicas,
called a read quorum.  For reads, the storage system compares the set
of values returned by the replicas, and, given a total ordering over
versions of the data item\footnote{This total ordering can be achieved
  using globally synchronized clocks~\cite{synch-clocks} or using a
  causal ordering provided by mechanisms such as vector
  clocks~\cite{vectorclock} with commutative merge
  functions~\cite{cops}}, can return the most recent value (or all
values returned, if desired).  Read and write quorums are chosen for
each operation from a set of sets of replicas, known as a
\textit{quorum system}.  There are many ways to configure quorum
systems, but one easy solution is to use read and write quorums of
fixed sizes, which we will denote $R$ and $W$, respectively, for a set
of nodes of size $N$.  To recap, the data system uses one quorum
system per data item.  The quorum systems across data items may be
identical, but they need not be.

Informally, a strict quorum system is a quorum system with the
property that any two quorums (sets) in the quorum system overlap
(have non-empty intersection). When considering distributed get/put
operations, reading and writing to sets of nodes in a strict quorum
system ensures strong consistency in the absence of failures.  The
minimum sized quorum defines its fault tolerance.  A simple example of
a strict quorum system is the majority quorum system, in which each
quorum is of size $\lceil \frac{N}{2}\rceil$.  However, the theory
literature contains numerous alternative quorum systems providing
varying asymptotic properties of capacity, scalability, and
fault-tolerance, from tree-quorums~\cite{treequorum} to
grid-quorums~\cite{quorumsystems}.  Jim\'{e}nex-Peris et. al provide a
useful overview of these traditional, strict quorum
systems~\cite{quorums-alternative}.

Partial quorum systems are natural extensions of strict quorum
systems: at least two quorums in a partial quorum system do not
overlap.  There are two relevant variants of partial quorum systems in
the literature: probabilistic quorum systems and k-quorums.

\textit{Probabilistic quorum systems} provide probabilistic guarantees
about whether data returned by a quorum read is consistent.  As the
number of nodes in the system scales, we can achieve an arbitrarily
high probability of consistency~\cite{prob-quorum}.  Intuitively, this
is a consequence of the Birthday Paradox: as the number of replicas
increases, the probability of non-intersection between any two quorums
decreases.  Prior work has examined the effect of ``dynamic systems''
in terms of quorum member churn (e.g., joining and leaving a
distributed hash table)~\cite{prob-quorum-dynamic}, network-aware
quorum placement~\cite{delay-quorum, quorum-placement}, and network
partitions~\cite{partitionedquorum} but did not model write
propagation. Additionally, and most importantly, to the best of our
knowledge, probabilistic quorums have only been used to study the
probability of strong consistency and have not been used to study
bounded staleness, particularly in the presence of anti-entropic
processes.  Merideth and Reiter provide a useful overview of recent
advances in quorum systems~\cite{quorum-overview}.

As an example of a probabilistic quorum system, given $N$ replicas and
randomly chosen read and write quorums of sizes $R$ and $W$, we can
calculate the probability of the read quorum not containing the value
last written by the write quorum.  The probability of inconsistency is
the number of quorums of size $R$ composed of nodes that were not
written to in the write quorum divided by the number of possible
quorums of size $R$:
\begin{equation}
\label{eq:prob-strict}
p_{stale}=\frac{{N-W \choose R}}{{N \choose R}}
\end{equation}
The probability of inconsistency is quite high except for large values
of $N$.  With $N=100$, $R=W=30$, $p_{stale} = 1.88 \times
10^{-6}$~\cite{non-strict}.  However, with $N=3$, $R=W=1$, $p_{stale}
= .\overline{6}$.  The asymptotics of this class of systems are
excellent---but only asymptotically.

\textit{$k$-quorum systems} provide strong \textit{deterministic}
guarantees that the partial quorum system will return a value that was
written within $k$ versions of the most recent
write~\cite{non-strict}.  In the single-writer scenario, a round-robin
write scheduling scheme where each write is sent to $\frac{N}{K}$
replicas ensures that any replica is no more than $K$ versions
out-of-date.  However, with multiple writers, the global ordering
properties that the single-writer was able to control are lost, and
the best known algorithm for the pathological case results in a lower
bound of $(2N-1)(k-1)+N$ versions staleness~\cite{multi-k-quorum}.

This prior work makes two important assumptions: first, it treats
quorum sizes as fixed, where the set of nodes with a version does not
grow over time, and, second, it frequently assumes Byzantine failure.
We revisit these assumptions at the end of the next section.

\subsection{Quorum Foundations: Practice}
\label{sec:practice}

In practice, many distributed data management systems use quorums as a
replication mechanism. Amazon's Dynamo~\cite{dynamo} is the progenitor
of a class of eventually-consistent key-value stores using a
particular variant of quorum-style replication that includes Apache
Cassandra~\cite{cassandra, cassandra-sigmod}, Basho's
Riak~\cite{riak}, and LinkedIn's Voldemort~\cite{voldemort,
  voldemortpub}.  We are not aware of any significantly different,
widely adopted data systems using quorum replication.  However, with
some work, we believe that other styles of replication can adopt our
methodology.  We describe key-value stores here, but any replicated
data store can use quorums, including full RDBMS systems.

Dynamo-style quorum systems employ one quorum system per key,
typically maintaining the mapping of keys to quorum systems using a
consistent-hashing scheme or a centralized membership protocol. Each
node stores multiple keys.  As shown in
Figure~\ref{fig:dynamo-quorum}, client read and write requests are
sent to a node in the system cluster, which forwards the request to
\textit{all} nodes assigned to that key as replicas.  The coordinating
node considers an operation complete when it has received responses
from a pre-defined number of replicas.  Accordingly, without message
loss, all replicas eventually receive all writes.  This means that the
write and read quorums chosen for a request depend on which nodes
respond to the request first.  In Dynamo terminology, the quorum size,
or replication factor, is defined as $N$, the number of replica
responses required for a successful read is defined as $R$, and the
number of replica acknowledgements required for a successful write is
defined as $W$. Dynamo-style systems are guaranteed to be consistent
when $R+W > N$.  Setting $W>\lceil N/2 \rceil$ ensures that a majority
of replicas will will receive a write in the presence of multiple
concurrent write requests.

\begin{figure}
\centering
\includegraphics[width=.8\columnwidth]{figs/dynamo-quorum.pdf}
\caption{Diagram of control flow for client write to Dynamo-style
  quorum.  Here, $N=3$, $W=2$. The client write is handled by a
  coordinator node and sent to all replicas. The write succeeds when
  $W$ replicas respond.  Note that the coordinator is possibly a
  replica as well, invoking a local write.}
\label{fig:dynamo-quorum}
\end{figure}

Many Dynamo-style systems also support additional anti-entropy
processes~\cite{nosql}.  One common feature is called \textit{read
  repair}~\cite{dynamo}.  When a read coordinator receives multiple
versions of a data item from different replicas in response to a read
request, it will attempt to (asynchronously) update the out-of-date
replicas with the most recent version.  The effect of read repair on
version drift between replicas is dependent on the read rate.  The
original Dynamo paper described the use of Merkle trees to summarize
and exchange data contents between replicas as a means of secondary
anti-entropy (after the initial write to $N$ replicas).  However, not
all Dynamo-style data stores actively employ similar gossip-based
anti-entropy.  For example, Cassandra only executes Merkle tree
anti-entropy when it is manually requested (e.g., \texttt{nodetool
  repair}), choosing to rely more heavily on read
repair~\cite{cassandra-merkle}.

There are significant differences between data systems in the wild and
the theory describing quorum operation.  First, replication factors
for distributed data systems are relatively low.  Typical replication
factors are between one and three~\cite{cassandradefault, feinbergpc,
  codapc}.  Second, (in the absence of failure), in Dynamo-style
partial quorums, the write quorum size increases even after the
operation returns, growing via anti-entropy.  Moreover, requests
are sent to all replicas, however only the first $R$ responses are
considered.  As a matter of nomenclature (and to disambiguate against
``dynamic'' quorum membership protocols), we will refer to these systems
as \textit{expanding partial quorum systems}.  Third, as in much of
the practical literature, practitioners largely focus on fail-stop
failure modes instead of Byzantine failure~\cite{birman-byzantine}.
Following standard practice, we do not consider Byzantine failure.

\subsection{Typical Quorum Configurations}

For improved latency, operators often set $R+W \leq N$.  Here, we
survey partial quorum configurations drawn from practitioner accounts.
In summary, many operators appear to use partial quorum
configurations, frequently citing performance benefits and high
availability.

 Cassandra's default operation configuration is $N$$=$$3$,
 $R$$=$$W$$=$$1$~\cite{cassandradefault}. The Apache Cassandra 1.0
 documentation claims that ``a majority of users do writes at
 consistency level [$W$$=$$1$]'', while the Cassandra Query Language
 defaults to $R$$=$$W$$=$$1$ as well~\cite{cassandra-docs}.
 Production Cassandra users report using $R$$=$$W$$=$$1$ in the
 ``general case'' because it provides ``maximum
 performance''~\cite{maxperfblog}, which appears to be a commonly held
 belief~\cite{reddit, outbrain}.  Cassandra has a ``minor'' patch for
 session guarantees~\cite{sessionguarantees}, but this has not yet
 been integrated into the mainline source~\cite{cassandra-session} as
 of version 1.0; according to our discussions with developers, this is
 due to lack of interest.

Riak defaults to $N$$=$$3$, $R$$=$$W$$=$$2$~\cite{riakdefault-n,
  riakdefault-rw}, however users suggest using $R$$=$$W$$=$$1$,
$N$$=$$2$ for ``low value'' data (and variants of overlapping quorums
for ``web,'' ``mission critical,'' and ``financial''
data)~\cite{riaktalkone, riaktalktwo}.

 Finally, Voldemort does not provide sample configurations, but
 Voldemort's authors (and operators) at LinkedIn~\cite{feinbergpc}
 often choose $N$$=$$c$, $R$$=$$W$$=$$ \lceil c/2 \rceil$ for odd $c$.
 For applications requiring ``very low latency and high
 availability,'' LinkedIn deploys Voldemort with $N$$=$$3$,
 $R$$=$$W$$=$$1$.  For other applications, LinkedIn deployments
 Voldemort with $N$$=$$2$, $R$$=$$W$$=$$1$, providing ``some
 consistency,'' particularly when $N$$=$$3$ replication is not
 required.  Additionally, Voldemort supports a concept of preferred
 reads and writes, meaning it will block until either the preferred
 number of replicas respond or a timeout occurs, at which point the
 request succeeds.  In the low latency case, preferred reads is either
 two or is disabled.  In the $N$$=$$2$ case, preferred reads and
 preferred writes are set to two.  Voldemort also slightly differs
 from Dynamo in that it sends read requests to $R$ of $N$ replicas
 (not $N$ of $N$)~\cite{voldemortpub}; this decreases load per replica
 and network traffic at the expense of read latency and potential
 availability.  However, provided the per-request staleness
 probabilities are independent, this does not affect staleness
 because, even when sending reads to $N$ of $N$ replicas, coordinators
 still only wait for $R$ responses.


\section{Probabilistically Bounded\\Staleness}
\label{sec:pbs}

In this section, we introduce the theory of Probabilistically Bounded
Staleness to describe the consistency provided by existing eventually
consistent data stores.  PBS extends prior work on
probabilistic quorums by accounting for staleness of both versions and
across time.  We introduce the notions of PBS $k$-staleness, which
stochastically limit the staleness of versions returned by read
quorums, PBS $t$-visibility, which stochastically bounds the time
before a committed version appears to readers, and PBS $\langle k,
t \rangle$-staleness, a combination of the two prior models.


Practical concerns guide the following theoretical contributions.  We
begin by considering a model without anti-entropic processes.  For the
purposes of a running example, as in Equation~\ref{eq:prob-strict}, we
assume that $W$ ($R$) of $N$ replicas are randomly selected for each
write (read) operation.  Similarly, we consider fixed $W$, $R$ and $N$
across multiple operations. Next, we expand our model to consider
write propagation and time-varying $W$ sizes in expanding partial
quorums.  In this section, we discuss anti-entropy in general, however
we model Dynamo-style quorums in Section \ref{sec:dynamo}. We discuss
further refinements to our assumptions in Section
\ref{sec:discussion}.

\subsection{PBS $k$-staleness}

Probabilistic quorums allow us to determine the probability of
returning the most recent value written to the database, but do not
tell us what happens when the most recent value is not returned.
Here, we determine the probability of returning a value within a
bounded number of versions.  In the following model, we non-expanding 
write quorums (no anti-entropy) and compose multiple independent write
quorums to model the probable overlap of $k$ independent write sets.
\begin{definition}
A quorum system obeys \textit{PBS $k$-staleness consistency} if, with
probability $1-p_{s}$, at least one value in any read quorum will
have been committed within $k$ versions of the latest committed
version when the read begins.
\end{definition}
Versions whose writes that are not yet committed (in-flight) may be
returned by a read in this formulation of probabilistic $k$-quorums
(see Figure \ref{fig:timelines}A).  The $k$-quorum literature defines
these as $k$-regular semantics~\cite{non-strict}.

\begin{figure}
\centering
\includegraphics[width=\columnwidth]{figs/timelines.pdf}
\caption{Possible versions returned by read operations under
  PBS $k$-staleness (A) and PBS monotonic reads (B). In
  $k$-consistency, the read operation will return a version no later
  than $k$ versions older than the last committed value when it
  started; more versions may be committed during the read and may be
  returned.  In monotonic reads consistency, the staleness depends on
  the number of versions committed since the time the client last
  completed a read.  This is determined by the proportion of client's
  reads to the number of writes committed to the key.}
\label{fig:timelines}
\end{figure}

The probability of returning a version of a key within the last $k$
versions committed is equivalent to intersecting one of $k$
independent write quorums.  Given a probability of intersecting with a
single write quorum $p$, the probability of intersecting one of $k$
independent quorums is $p^k$.  Accordingly, in our example quorum
system, the probability of non-intersection is simply Equation
\ref{eq:prob-strict} exponentiated by $k$:
\begin{equation}
\label{eq:k-consistency}
p_{s} = \left(\frac{{N-W \choose R}}{{N \choose R}}\right)^k
\end{equation}

For the $N=3, R=W=1$ case, this means that the probability of
returning a version within $2$ versions is $.\overline{5}$, within $3$
versions $.\overline{703}$, and within $5$ versions $> .868$, and $10$
versions $>.98$.  With $N=3, R=1, W=2$ (alternatively, $R=2, W=1$),
these probabilities increase: $k=1 \rightarrow
.\overline{6}$, $k=2 \rightarrow .\overline{8}, k=5 \rightarrow >
.995$.

This closed form solution holds for quorums that do not change size
over time.  For expanding partial quourm systems, this solution is an
upper bound on the probability of staleness.  We discuss the
\textit{load} improvements offered by PBS $k$-staleness as discussed
in quorum system literature in Appendix A.

\subsection{PBS Monotonic Reads}

With additional information, we can use PBS $k$-staleness to predict
whether a client will ever read staler data than it has already read.
This property, known as \textit{monotonic reads} consistency is a
well-known session guarantee~\cite{sessionguarantees}.

\begin{definition}
\label{def:prob-mr}
A quorum system obeys \textit{PBS monotonic reads consistency} if,
with probability at least $1-p_{s}$, at least one value in any
read quorum returned to a client is the same version or a newer
version than the last version that the client previously read.
\end{definition}

To guarantee that a client sees monotonically increasing versions, it
can continue to contact the same replica~\cite{vogels-defs} (provided
the ``sticky'' replica does not fail).  However, this is insufficient to
guarantee strict monotonic reads (where the client reads strictly
newer data if it exists in the system).  Definition~\ref{def:prob-mr}
can be adapted to accommodate strict monotonic reads by requiring
that, if a more recent data version is available, it is returned.

PBS monotonic reads consistency is a special case of PBS $k$-staleness (see
Figure~\ref{fig:timelines}B), where $k$ is determined by a client's
rate of reads from a data item ($\gamma_{cr}$) and the global,
system-wide rate of writes to the same data item ($\gamma_{gw}$).  If
we know these rates exactly, the number of versions between client
reads is $\frac{\gamma_{gw}}{\gamma_{cr}}$, as shown in Figure
\ref{fig:timelines}B.  We can calculate the probability of
probabilistic monotonic reads as a special case of $k$-staleness where
$k=1+\frac{\gamma_{gw}}{\gamma_{cr}}$.  For example, extending the
example in Equation \ref{eq:k-consistency}:
\begin{equation}
\label{eq:prob-mr}
p_{s} = \left(\frac{{N-W \choose R}}{{N \choose R}}\right)^{1+\gamma_{gw}/\gamma_{cr}}
\end{equation}
For strict monotonic reads, where we cannot read the version we have
previously read (assuming there are newer versions in the database),
exponentiate where $k=\frac{\gamma_{gw}}{\gamma_{cr}}$.

In practice, we may not know these exact rates, but, by measuring
their distribution, we can calculate an expected ratio that we can
integrate into these calculations.  By performing appropriate
admission control, operators can control these rates to achieve
monotonic reads guarantees.

\subsection{PBS $t$-visibility}
\label{sec:tvis}

Until now, we have considered only quorums that do not grow over time.
However, as we discussed in Section \ref{sec:practice}, real-world quorum
systems are expanding, or asynchronously propagate writes to quorum system
members over time.  This process is commonly known as
anti-entropy~\cite{antientropy}.  For generality, in this section, we
will discuss general anti-entropy models. However, we explicitly model
the Dynamo-style anti-entropy mechanisms in Section \ref{sec:dynamo}.

PBS $t$-visibility models the probability of inconsistency for
expanding quorums.  Intuitively, PBS $t$-visibility captures the
possibility that a reader will observe a write $t$ seconds after it
commits.  Recall that we consider in-flight writes---which are more
recent than the last committed version---as non-stale.

\begin{definition}
A quorum system obeys \textit{PBS $t$-visibility consistency} if, with
probability $1-p_{s}$, any read quorum started at least $t$ units
of time after the last version committed returns at least one value
that is no older than the last version committed when the read
began (and may not be committed yet).
\end{definition}

We denote the cumulative density function describing the number of
replicas $\mathcal{W}$ that have received a particular version $v$ (or
a version newer than $v$) $t$ seconds after $v$ commits as
$P_w(\mathcal{W}, t)$.

By definition, for expanding quorums, $\forall c \in [0, w], P_w(c,0)
= 1$; at commit time, $W$ replicas will have received the value with
certainty.  We can model the probability of PBS $t$-visibility for an
interval $t$ by summing the conditional probabilities of each possible
$\mathcal{W}$:
\begin{equation}
\label{eq:tv-instantreads}
p_{s} = \frac{{N-W \choose N}}{{N \choose R}}+\sum_{c\in(W, N]} \frac{{N-c \choose N}}{{N \choose R}}\cdot [P_w(c+1, t)-P_w(c,t)]
\end{equation}
However, the above equation assumes reads occur instantaneously and
writes commit immediately after $W$ replicas have the version (i.e.,
there is no delay acknowledging the write to the coordinating node).
In the real world, writes need to be acknowledged and read requests
take time to arrive at remote replicas.  Therefore, more time will elapse
between the time $W$ of $N$ replicas have a version and $t$.
Accordingly, Equation~\ref{eq:tv-instantreads} is a conservative upper
bound on $t$-staleness.

In practice, $P_w$ depends on the anti-entropy protocols and the
expected latency of operations and can be approximated (Section
\ref{sec:dynamo}) or measured online.  For this reason, the load of a
PBS $t$-visible quorum system depends on write propagation and is
difficult to analytically determine for general-purpose expanding
quorums.  Additionally, one can model both transient and permanent
failures by increasing the tail probabilities in $P_w$.


\subsection{PBS $\langle k, t \rangle$-staleness}

We can combine the previous models to combine both versioned and
real-time staleness metrics to answer questions of the following form:
what is the probability that a read will return a value no older than
$k$ versions stale if the last write committed no sooner than $t$ seconds
ago?
\begin{definition}
A quorum system obeys \textit{PBS $\langle k, t \rangle$-staleness
  consistency} if, with probability $1-p_{s}$, at least one value in
any read quorum will be within $k$ versions of the latest committed
version when the read begins, provided the read begins $t$ units of
time after the previous $k$ versions commit.
\end{definition}
The definition of $p_{s}$ follows from the prior definitions:
\begin{equation}
p_{s} = \left(\frac{{N-W \choose R}}{{N \choose R}}+\sum_{c\in[W, N)} \frac{{N-c \choose R}}{{N \choose R}} \cdot [P_w(c+1, t)-P_w(c,t)]\right)^k
\end{equation}
In this equation, in addition to (again) assuming instantaneous reads,
we also assume the pathological case where the last $k$ writes all
occurred at the same time.  This is not likely in practice, so if we
can determine the time since commit for the last $k$ writes, we can
improve this staleness bound by considering each quorum's $p_{s}$
at $RT=t$ separately.  However, this equation provides a conservative
upper bound on $p_{s}$.

Note that the prior definitions of consistency are encapsulated by PBS
$\langle k, t \rangle$-staleness consistency. probabilistic $k$-quorum
consistency is simply PBS $\langle k, 0 \rangle$-staleness consistency,
PBS monotonic reads consistency is $\langle
1+\frac{\gamma_{gw}}{\gamma_{cr}}, 0 \rangle$-staleness consistency, and
PBS $t$-visibility is $\langle 0, t \rangle$-staleness consistency.

In practice, we believe it is easier to reason about staleness of
versions or staleness in terms of real time but not both together.
Accordingly, having derived a closed-form model for $k$-visibility, in
the remainder of this paper, we focus mainly on deriving specific models
for $t$-visibility.

\section{Dynamo-style $t$-visibility}
\label{sec:dynamo}

We have formulated a closed-form analytical model for $k$-staleness,
but $t$-visibility is dependent on both the distributed quorum
execution algorithm and the anti-entropy protocols employed by a data
storage system.  In this section, we discuss PBS $t$-visibility in the
context of Dynamo-style data storage systems.  We describe how to
model the probability of staleness in these systems and how to
asynchronously detect staleness.

\subsection{Inconsistency in Dynamo: WARS Model}

Dynamo-style quorum systems are inconsistent as a result of read and
write message reordering.  Reads and writes are sent to all quorum
members, so the staleness under normal operation results only when all
of the first $R$ responses to a read request arrived at their
respective replicas before the last committed write request.  To
demonstrate this phenomenon, we introduce a model of Dynamo operation
which, for convenience, we will call \textit{WARS} .

We illustrate \textit{WARS} in Figure~\ref{fig:dynamo-diagram}, a
space-time diagram for messages between a coordinator and a single
replica for a write followed by a read $t$ seconds after the write
commits.  This $t$ corresponds to the $t$ in PBS $t$-visibility.

\begin{figure}
\centering
\includegraphics[width=.8\columnwidth]{figs/dynamostale.pdf}
\caption{The \textit{WARS} model for message ordering in Dynamo
  describes the message flow between a coordinator and a single
  replica for a write followed by a read $t$ seconds after commit.  In
  an $N$ replica system, this message flow occurs $N$ times, once for
  each of the $N$ replicas.  The read and write may be handled by
  different coordinators.}
\label{fig:dynamo-diagram}
\end{figure}

For a write, the coordinator sends $N$ messages, one to each
replica. The message from coordinator to replica containing the
version to be written is delayed by a value chosen from distribution
\texttt{W}.  The coordinator waits for $W$ responses from the replicas
before it can consider the version committed.  Each response
acknowledging the write is delayed by a value chosen from the
distribution \texttt{A}.

For a read, the coordinator sends $N$ messages, one to each replica.
The message from coordinator to replica containing the read request is
delayed by a value chosen from distribution \texttt{R}.  The
coordinator waits for $R$ responses from the replicas before returning
the most recent value it recieves.  Each read response is delayed by a
value chosen from the distribution \texttt{S}.

In the absence of additional anti-entropic processes, the coordinator
will return stale data if all $R$ of the $N$ responses received before
the read returns reached their respective replicas before the
respective message delayed by \texttt{W} does.  When $R$$+$$W$$>$$N$,
this is impossible.  However, under partial quorums (when
$R$$+$$W$$\leq$$N$), this depends on the latency distributions.  If we
denote the commit time as $w_t$, to observe staleness, $r+w_t+t< w$ for
$r$ chosen from \texttt{R} and $w$ chosen from \texttt{W}.  Writes
have time to propagate to additional replicas both while the
coordinator waits for all required acknowledgments (\texttt{A}) and as
read requests are subsequently sent (\texttt{R}).  Similarly, read
responses are further delayed in transit (\texttt{S}) back to the read
coordinator, inducing further possibility of reordering.
Qualitatively, longer write tails and faster reads increase the chance
of staleness due to the possibility of reordering.

Depending on which coordinator a client contacts, some reads and
writes may be served locally.  In this case, subject to local query
processing delays, a read or write to $R$ or $W$ nodes is behaves like
a read or write to $R-1$ or $W-1$ nodes, respectively.  Although we do
not do so, \textit{WARS} can be adopted to handle local reads and
writes.  Determining whether requests will be proxied (and, if not,
which replicas serve which requests) is data store and
deployment-specific.  Dynamo forwards write requests to a designator
coordinator solely for the purpose of establishing a version
ordering~\cite[Section 6.4]{dynamo} (easily achievable through other
mechanisms~\cite{zookeeper}).  Dynamo's authors observed a latency
improvement by proxying all operations and having clients act as
coordinators---Voldemort adopts this architecture~\cite{voldemortclient}.

An end-user will likely incur additional time between his or her
reads and writes due to latency required to contact the datacenter.
An individual making requests to a web service using his or her
browser will likely incur tens or hundreds of milliseconds of delay
between requests.  Although we do not consider this delay here, it is
useful to to consider in practical scenarios because the delay between
reads and writes may be large.

\textit{WARS} considers the effect of message sending, delays, and
reception, but this represents a daunting analytical formulation.  The
commit time represents an order statistic dependent on both \texttt{W}
and \texttt{A}. \begin{comment}{\color{red}: Peter A
    ``confused''}\end{comment} Furthermore, message reordering is
dependent on the commit time.  The probability that the $i$th returned
read message observes reordering is both an order statistic dependent
on $W$ and $A$.  Across $i$, the probabilities are dependent.  The
expected read and write latencies can be computed using simple order
statistics if one makes independence assumptions about the
distributions in \textit{WARS} .  However, a concise, closed form for
the probability of inconsistency eludes us.  Dynamo is simple to
reason about and program but difficult to analyze in closed form.  As
we discuss in Section~\ref{sec:dynamoeval}, we explore \textit{WARS}
using Monte Carlo analysis, which suffices for our purposes in
practice and, from a practical perspective, is relatively easy to
implement.

\subsection{Additional Anti-entropy}

As we discussed in Section~\ref{sec:practice}, anti-entropic processes
decrease the probability of staleness by further propagating versions
between members.  In addition to write-all semantics, Dynamo employs
read repair and Merkle tree exhange. However, both of these processes
are rate dependent: read repair effects depend on the rate of reads,
and Merkle tree exchange effects (and, more generally, most
anti-entropy protocols) depend on the rate of exchange.  \textit{WARS}
is read rate independent but is affected by the write rate.  If
multiple writes overlap (that is, have overlapping periods where they
are in-flight but are not committed) the probability of inconsistency
decreases.  Intuitively, this is because overlapping writes results in an
increased chance that a client reads as-yet-uncommitted data.
Accordingly, because \textit{WARS} does not capture read repair or
Merkle tree exchange, it represents a lower bound for PBS
$t$-staleness for Dynamo-style expanding partial quorums. In practice,
versions may be fresher than predicted.\begin{comment} {\color{red}
    Peter A: more general}\end{comment}

In the presence of multiple concurrent writes (or even periodic
writes), $t$-visibility time is bounded by the time between writes.
Intuitively, if two writes to a key are spaced $m$ milliseconds apart,
then the $t$-visibility of the first write for $t > m$ milliseconds is
undefined; after $m$ milliseconds, there is a newer version.

\subsection{Asynchronous Staleness Detection}

Even if a system provides an extremely high probability of
consistency, it  may be useful if applications can be notified when
data returned is inconsistent, or staler than expected.  The Dynamo
model is naturally equipped for staleness detection.

Knowing whether a response is stale at read time requires strong
consistency.  By checking all possible values in the domain against a
hypothetical staleness detector, we could determine the consistent
value to return.  While we cannot do so synchronously, we \textit{can}
determine staleness asynchronously.  Asynchronous staleness detection
allows speculative execution~\cite{nsdispeculation} if a program
contains appropriate compensation logic.

We first consider a staleness detector that provides false positives.
Recall that, in a Dynamo-style system, we wait for $R$ of $N$ replies
before returning a value.  The remaining $N-R$ replicas will still
reply to the read coordinator.  Instead of dropping these messages,
the coordinator can compare them to the version it returned.  If there
is a mismatch, then either the coordinator returned stale data, there
are in-flight writes in the system, or additional versions committed
after the read. The latter two cases, relating to data committed after
the response was initiated, lead to false positives.  Because we
define staleness according to the last committed write when the read
began, versions written after the last committed version do not
technically constitute stale reads.  Notifying clients of newer
versions of a data item is not necessarily bad but may be unnecessary
and violates our staleness semantics.  Note that this detector does
not require modifications to the Dynamo protocol and is similar to the
read-repair process.

To eliminate false positives, we need to determine the total,
system-wide commit ordering of writes. Recall that replicas are
unaware of the commit time for each version; commits occur after $W$
replicas respond, and the version stamps that replicas store are not
updated after commit.  Establishing a total ordering is a well-known
distributed systems problem that could be accomplished in Dynamo using
a centralized service~\cite{zookeeper} or using distributed
consensus~\cite{paxos}.  Eliminating false positives requires
modifications to the Dynamo protocol but is definitely feasible.

While this discussion has been couched in the terms of strong
consistency, it is easily extended to PBS $k$-staleness and PBS
$\langle k, t \rangle$-staleness.

\section{Evaluating Dynamo $t$-visibility}
\label{sec:dynamoeval}

As discussed in Section~\ref{sec:tvis}, PBS $t$-visibility depends on
the propagation of reads and writes throughout a system.  We
introduced the \textit{WARS}  model as a means of reasoning about
inconsistency in Dynamo-style quorum systems, but quantitative metrics
such as staleness observed in practice depend on each of \textit{WARS}'s
latency distributions.  In this section, we perform an analysis of
Dynamo-style $t$-visibility under several different assumptions.  We
use a mix of real-world experiments and Monte Carlo analyses
under distributions from production environments to better
understand how frequently ``eventually consistent'' means
``consistent'' and, more importantly, why.

Recall that PBS $k$-staleness is easily captured in closed form for
non-expanding quorums.  In practice, without anti-entropy, we observe
that our derived equations hold true.

In this section, we focus on worst-case bounds for PBS $t$-visibility.  While
we could improve the staleness results by considering additional
anti-entropic processes, we make the bare minimum of assumptions as
dictated by the \textit{WARS}  model.  Opting for conservative analyses
decreases the number of varibles in our experiments (each of which is
supported by empirical observations from practitioners) and increases
the applicability of these results.

\subsection{Monte Carlo Simulator}

We implemented \textit{WARS} in an event-driven simulator.
Calculating $t$-visibility for a given value of $t$ is rather
straightforward: draw $N$ samples from \texttt{W}, \texttt{A},
\texttt{R}, and \texttt{S} at time $t$ (denote index $i$ as $[i]$),
compute $w_t = $ the $W$th smallest value of the respective
$\texttt{W}+\texttt{A}$ values, and check whether the first $R$
samples of \texttt{R}, ordered by $\texttt{R}[i]+\texttt{S}[i]$ obey
$w_t+\texttt{R}[i] + t\leq \texttt{W}[i]$.  This requires only a few
lines of code.  Implementing $\langle k, t \rangle$-staleness requires
remembering write ordering and history but is not difficult.

\subsection{Model Validation}

To validate \textit{WARS}  (and our subsequent analyses), we instrumented a
commercially available, open source Dynamo-style key-value store to
measure the staleness observed under partial quorum operation.
Unsurprisingly, our observations matched \textit{WARS}.

We modified Cassandra, a data store originally designed at Facebook,
which provides Dynamo-style partial quorum semantics, offering
per-request quorum sizes and a BigTable-like data
model~\cite{cassandra, cassandra-sigmod}, to provide timing
information about reads and writes.  To provide a single point of
order for the series of reads and writes in the system, we used $N+1$
Cassandra nodes for each experiment involving $N$ nodes.  After
configuring the Cassandra schema, we determined the node that did not
store the data and sent all requests through it, effectively creating
a remote proxy.  \textit{WARS} is more general than the specific proxy
configuration, however using a centralized proxy allows us to avoid
most issues with clock skew between replicas and rely less heavily on
globally synchronized clocks.  We disabled read repair because it is
external to \textit{WARS}.

We ran Cassandra on a set of machines with 2.2GHz AMD Opteron 2214
dual-core SMT processors with 4GB of 667MHz DDR2 memory serving
in-memory data.  To test \textit{WARS}, we inserted monotonically
increasing version numbers to a single key, while five concurrent
processes read values measured the probability of staleness at every
millisecond.  We injected a set of delays into both the request and
response sending modules in Cassandra (we discuss a range of
distributions in Section~\ref{sec:latencies}) and measured the
probability of staleness.  Under the empirically-measured delay
distributions, our simulation was accurate for $t$-visibility for
$t\in\{1,\cdots,199\}$ with RMSE=0.0023\%.  This validates our Monte
Carlo simulator.

\subsection{Production Latency Distributions}
\label{sec:latencies}

\textit{WARS} depends only on latency distributions, and, rather than
conjecture as to what represent ``reasonable'' scenarios, we analyzed
production distributions from two internet-scale companies, LinkedIn
and Yammer.

LinkedIn~\cite{linkedin} is an online professional social network
claiming over 135 million members as of November
2011~\cite{linkedinmembers}. To handle highly available, low latency
data storage, engineers at LinkedIn built Voldemort, a Dynamo-style
quorum replicated key value store~\cite{voldemort, voldemortpub}.
Alex Feinberg, a lead engineer on Voldemort, graciously provided us
with latency distributions for a single node replaying peak load
traffic for a user-facing service at LinkedIn, representing 60\% read
and 40\% read-modify-write traffic~\cite{feinbergpc}
(Table~\ref{table:linkedin}).  Feinberg reports that ``with spinning
disks, we're largely IO bound and latency is largely determined by the
kind of disks we're using, data to memory ratio and request
distribution.  With [solid state drives (SSDs)], we're CPU and/or
network bound (depending on value size).''  As an interesting aside,
Feinberg also notes that ``maximum latency is generally determined by
[garbage collection] activity (rare, but happens occasionally) and is
within hundreds of milliseconds.''

\begin{table}
\centering
\begin{tabular}{|c|c|}
\hline
\%ile & Latency (ms) \\
\hline
\multicolumn{2}{|c|}{ 15,000 RPM SAS Disk}\\
\hline
Average & 4.85\\
95 & 15\\
99 & 25\\
\hline
\multicolumn{2}{|c|}{ Commodity SSD }\\
\hline
Average & 0.58 \\
95 & 1\\
99 & 2\\
\hline
\end{tabular}
\caption{LinkedIn Voldemort single-node production latencies.}
\label{table:linkedin}
\end{table}

Yammer is an online social network designed to provide enterprises
with private social networking capabilities.  As of December 2011,
Yammer claimed over 100,000 companies as customers~\cite{yammer}.
Yammer uses Basho's Riak, another open source Dynamo-style quorum
replicated database for client data~\cite{riak}.  Coda Hale,
infrastructure architect, and Ryan Kennedy, also of Yammer, presented
on their use of Riak including surprisingly in-depth performance
numbers in March 2011~\cite{riakyammer}.  Yammer provided us with more
detailed performance numbers for their application~\cite{codapc}
(Table~\ref{table:yammer}).  Hale noted that ``reads and writes have
radically different expected latencies, especially for Riak. Writes
don't return until the fsync returns, so while reads are often $<$
1ms, writes rarely are.''  As we will see, this has important
consequences for \textit{WARS}.  Although we do not model this
explicitly, Hale also notes that ``value size is also interesting. We
saw a big performance improvement by adding LZF compression to
values.''

\begin{table}
\centering
\begin{tabular}{|c|c|c|}
\hline
\%ile & Read Latency (ms) & Write Latency (ms)\\
\hline
Min & 1.55 & 1.68\\
50 & 3.75 & 5.73 \\
75 & 4.17 & 6.50\\
95 & 5.2 & 8.48\\
98 & 6.045 & 10.36 \\
99 & 6.59 & 131.73\\
99.9 & 32.89 & 435.83\\
Max & 2979.85 &  4465.28 \\
\hline
Mean & 9.23 & 8.62 \\
Std. Dev. & 83.93 & 26.10\\
\hline
Mean Rate & 718.18 gets/s & 45.65 puts/s\\
\hline
\end{tabular}
\caption{Yammer Riak $N$$=$$3$, $R$$=$$2$, $W$$=$$2$ production latencies.}
\label{table:yammer}
\end{table}

\subsection{Production Latency Model Fitting}

While insights from production data are invaluable, they are
underspecified for \textit{WARS}.  First, the data we collected are
summary statistics, but \textit{WARS} requires the underlying
distributions.  More importantly, the operation latencies represent
round-trip times, while \textit{WARS} requires the constituent one-way
latencies for both reads and writes.  As we demonstrated in our
Cassandra experiments, these latency distributions are easily
collected.  However, because they are not currently collected in
practice, we must do our best to fill in the gaps. Accordingly, to
derive \texttt{W},\texttt{A},\texttt{R},\texttt{S} for each
configuration, we made a series of assumptions, which we believe are
justified given the advantage of production data over synthetic
distributions.  Without additional data on the latency required to
read multiple replicas, we assume that each latency distribution is
independently, identically distributed.

LinkedIn provided two latency distributions, which we denote
\texttt{LNKD-SSD} and \texttt{LNKD-DISK} for the SSD and spinning
disks, respectively.  As previously discussed, when running on SSDs,
Voldemort is largely network and CPU bound.  Accordingly, we assumed
that read and write operations took equivalent amounts of time and, to
split the remaining time, we focused on the network-bound case and
assumed that one-way trips were symmetric
(\texttt{W}=\texttt{A}=\texttt{R}=\texttt{S}).  This allowed us to
derive \texttt{LNKD-SSD}.  For \texttt{LNKD-DISK}, Feinberg mentioned
that Voldemort performs at least one read before every write (average
of 1 seek, between 1-3 seeks), and writes to the BerkeleyDB Java
Edition backend are flushed every 30 seconds or 20
megabytes---whichever comes first.  Accordingly, we kept the same
\texttt{A}=\texttt{R}=\texttt{S} as in \texttt{LNKD-SSD} but
calculated \texttt{W} separately.  

\begin{table}
\centering
\begin{tabular}{|c|r|}
\hline
\multirow{4}{*}{\texttt{LNKD-SSD}} & \multicolumn{1}{|l|}{$\texttt{W} = \texttt{A}= \texttt{R} = \texttt{S}:$} \\
& 91.22\%: Pareto, $x_m=.235, \alpha=10$\\
& 8.78\%: Exponential, $\lambda = 1.66$ \\
& N-RMSE: .55\%\\\hline
\multirow{4}{*}{\texttt{LNKD-DISK}} & \multicolumn{1}{|l|}{$\texttt{A}= \texttt{R} = \texttt{S}: \texttt{LNKD-SSD}$}\\\cline{2-2}
& \multicolumn{1}{|l|}{\texttt{W}:}\\
& 38\%: Pareto, $x_m=1.05, \alpha=1.51$\\
& \hfill 62\%: Exponential, $\lambda = .183$ \\
& N-RMSE: .26\%\\
\hline
\multirow{8}{*}{\texttt{YMMR}} & \multicolumn{1}{|l|}{\texttt{W}:} \\
& 93.9\%: Pareto, $x_m=3, \alpha=3.35$\\
& 6.1\%: Exponential, $\lambda = .0028$ \\
& N-RMSE: 1.84\%\\\cline{2-2}
& \multicolumn{1}{|l|}{$\texttt{A}= \texttt{R} = \texttt{S}:$}\\
& 98.2\%: Pareto, $x_m=1.5, \alpha=3.8$\\
& 1.8\%: Exponential, $\lambda=.0217$\\
& N-RMSE: .06\%\\
\hline
\end{tabular}
\caption{Distribution fits for production latency distributions.}
\label{table:fits}
\end{table}


Yammer provided distributions for a single configuration but separated
read from write latencies, which we denote \texttt{YMMR}.  Using our
IID assumptions, we fit single-node latency distributions to the
provided distributions, again assuming symmetric \texttt{A},
\texttt{R}, and \texttt{S}.  The data fit a Pareto distribution with a
long exponential tail.  At the $98$th percentile, the write
distribution takes a sharp turn.  Fitting the data closely resulted in
an extremely long tail, with $99.99+$th percentile writes requiring
tens of seconds---much higher than Yammer specified.  Accordingly, we
fit the $98$th percentile knee conservatively; without the $98$th
percentile, the write fit N-RMSE is .104\%.


\begin{figure*}[t!]
\centering
\subfigure{\includegraphics[width=\columnwidth]{figs/latlegend.pdf}}\\[-1mm]
\subfigure{\includegraphics[width=.6\columnwidth]{figs/readlats-1.pdf}}
\subfigure{\includegraphics[width=.6\columnwidth]{figs/readlats-2.pdf}}
\subfigure{\includegraphics[width=.6\columnwidth]{figs/readlats-3.pdf}}
\subfigure{\includegraphics[width=.6\columnwidth]{figs/writelats-1.pdf}}
\subfigure{\includegraphics[width=.6\columnwidth]{figs/writelats-2.pdf}}
\subfigure{\includegraphics[width=.6\columnwidth]{figs/writelats-3.pdf}}
\caption{Read and write operation latency for \textit{WARS} latencies fitted
  to production datasets for $N$$=$$3$.}
\label{fig:latencies}
\end{figure*}

Finally, we simulated a normal-case WAN replication scenario,
\texttt{WAN}.  Reads and writes are routed to random data centers,
and, accordingly, one replica operation commits quickly while the
others are routed remotely.  We delay remote operations and responses
by 75ms and apply \texttt{LNKD-DISK} delays once the operation reaches its
target data center, in accordance with a near-worst-case
multi-continent WAN network delay~\cite{dean-keynote}.

We show the parameters for each distribution in
Table~\ref{table:fits}. We plot each fitted distribution in
Figure~\ref{fig:latencies}.  Note that for $R$, $W$ of one,
\texttt{LNKD-DISK} is not equivalent to \texttt{WAN}.  This is
because, in \texttt{LNKD-DISK}, we only have to wait for the first of
$N$ local reads (writes) to return, whereas, for \texttt{WAN}, there
is only one local read (write) and all other read (write) requests
are be delayed at least 150ms.

\subsection{Observed $t$-visibility}

We measured the $t$-visibility for each distribution
(Figure~\ref{fig:tvis}).  We simulated reads
after each of 10,000 non-overlapping writes staggered by time
and calculated the probability of consistency, or 
$t$-visibility across time.

As our qualitative \textit{WARS} analysis predicted, the probability
of consistency was largely dependent on the write tail size.
Immediately after write commit, with $N$$=$$3$, $R$$=$$W$$=$$1$,
\texttt{YMMR} had a $89.3\%$ chance of consistency, \texttt{LNKD-SSD}
had a $97.5\%$ chance, and \texttt{LNKD-DISK} had a $43.9\%$ chance.
Ten milliseconds after write commit, \texttt{YMMR} had a $93.9\%$
chance of consistency, \texttt{LNKD-SSD} had a $100\%$ chance, and
\texttt{LNKD-DISK} had a $92.5\%$ chance.  \texttt{YMMR}'s tall body
but long tail limited its probability increase, and only reached
$99.9\%$ consistency 1364 ms after writing.  \texttt{LNKD-SSD}'s reads
raced its writes immediately after commit, but, one millisecond after
the write, the chance of a read round-trip time plus one millisecond
beating its write was almost eliminated.  \texttt{LNKD-DISK} had the
the longest body of these three distributions but reached $99.9\%$
confidence after 45.5 ms (having a shorter tail than \texttt{YMMR}).  As
expected, \texttt{WAN} observed poor chances of consistency until
after the 75 milliseconds passed; unless a reader read from the same
datacenter in which the last write committed, it had to wait for the
long propagation to observe the most recent value.

\begin{figure*}
\centering
\subfigure{\includegraphics[width=\columnwidth]{figs/latlegend.pdf}}\\[-1mm]
\subfigure{\includegraphics[width=.65\columnwidth]{figs/tstales-3N1R1W.pdf}}
\subfigure{\includegraphics[width=.65\columnwidth]{figs/tstales-3N2R1W.pdf}}
\subfigure{\includegraphics[width=.65\columnwidth]{figs/tstales-3N1R2W.pdf}}
\caption{$t$-visibility for production operation latencies.}
\label{fig:tvis}
\end{figure*}

\subsection{Latency Tail Effects}

%the ratio of \texttt{W} to \texttt{A}=\texttt{R}=\texttt{S} and swept
%a range of exp 
To better understand the impact of write tail latency on
$t$-staleness, we swept a range of exponential distributions, fixing
\texttt{A}=\texttt{R}=\texttt{S} and varying \texttt{W}.  The results,
shown in Figure~\ref{fig:varydelay}, indicate that probability of
consistency is largely influenced by the write latency. When the mean
of \texttt{W} ($.25$ms) is one-fourth the mean of
\texttt{A}=\texttt{R}=\texttt{S}, we observe a $94\%$ chance of
consistency immediately after the write and $99.9\%$ chance after 1ms.
However, when the mean of \texttt{W} ($10$ms) is ten times the mean of
\texttt{A}=\texttt{R}=\texttt{S}, we observe a $41\%$ chance of
consistency immediately after write and a $99.9\%$ chance of
consistency after $65$ ms. This result corroborates our earlier
observation that employing SSDs lowers the write latency and results
in lower $t$-visibility when compared to using disks.  More generally,
this result suggests that instead of increasing read and write quorum
sizes to decrease $t$-visibility, operators could chose to lower
(relative) \texttt{W} latencies.  This is achievable through hardware
configurations or by delay read latencies.  However, this latter
option is potentially problematic for read-heavy workloads and
introduces queuing effects for reads and writes.

\begin{figure}
\centering
\includegraphics[width=.85\columnwidth]{figs/rwratio.pdf}
\caption{$t$-visibility for $N$$=$$3$, $R$$=$$W$$=$$1$ varying \texttt{W} and \texttt{A}=\texttt{R}=\texttt{S} exponentially distributed delays.  Mean latency is $1/\lambda$.}
\label{fig:varydelay}
\end{figure}

\subsection{Replica Size}

We consider how changing the number of replicas (N) affects
$t$-visibility while maintaining $R$$=$$W$$=$$1$. The results shown in
Figure~\ref{fig:varyn} show that the probability of consistency
immediately after write commit decreases as $N$ increases.  However,
at the tail, the $t$-visibility for increased replica sizes is
surprisingly close.  For example, the \texttt{LNKD-DISK} latency
distribution, the $t$-visibility at $99.9\%$ probability of
consistency ranges from $45.3$ ms for 2 replicas to $53.7$ ms for 10
replicas.  This implies that even if we choose to maintain a large
number of replicas for availability or better performance, the
$t$-visibility staleness will still converge relatively quickly for large $t$
grows. We believe the difference is larger due to the longer tail in
the write latency distribution.

\begin{figure*}
\centering
\subfigure{\includegraphics[width=.65\columnwidth]{figs/sweepn-lnkd-disk.pdf}}
\subfigure{\includegraphics[width=.65\columnwidth]{figs/sweepn-LNKD-SSD.pdf}}
\subfigure{\includegraphics[width=.65\columnwidth]{figs/sweepn-WAN.pdf}}
\caption{$t$-visibility for production operating latencies for variable $N$ and $R$$=$$W$$=$$1$.}
\label{fig:varyn}
\end{figure*}

\subsection{Latency vs. $t$-visibility}

As we have discussed, choosing a value for $R$ and $W$ results in a
trade-off between operation latency and the $t$-visibility observed by
reads. To measure the obtainable latency gains, we studied
$t$-visibility required for a $99.9\%$ probability of consistent
reads, capturing the long tail of $t$-visibility.  We also measured
the $99.9$ percentile read, write latencies.
Table~\ref{table:lat-stale} shows the results for our latency
distributions and $R$, $W$ configurations.  For \texttt{YMMR}, we find
that though the best setting in terms of latency is $W$$=$$R$$=$$1$,
this configuration results in a high value for $t$-visibility: $1364$
ms. However, by setting $R$$=$$2$ and $W$$=$$1$, the $t$-visibility
reduces to $202$ ms and the combined read and write latencies decrease
by $81.1\%$ when with a compared to an overlapping quorum with
$W$$=$$1$, $R$$=$$3$.  Similarly, in the case of \texttt{LNKD-DISK} we
see that by adjusting the value of $R$ and $W$, combined read and
write latencies can reduce by $16.5\%$ with $t$-visibility of $13.6$
ms.  The write tail of \texttt{LNKD-SSD} was such that we never
observed message reordering for $R$$=$$2$, $W$$=$$1$, allowing a 30\%
latency reduction for no observable staleness (even across
$10,000,000$ writes).  $R$$=$$W$$=$$1$, reduced latency by $59.5\%$
with a corresponding $t$-visibility of $1.85$ ms.  Under \texttt{WAN},
$R$ or $W$ greater than one results in waiting for non-local
operations to complete and therefore large propagation time.  This
unsurprisingly results in staleness only when $R$$=$$W$$=$$1$.  These
results indicate that there are significant latency gains when to
employing lower values of $R$, $W$ and that $t$-visibility is can be
low even when we require a high percentage ($99.9\%$) of reads to be
consistent.

\begin{table*}
\centering
\begin{tabular}{c|c|c|c|c|c|c|c|c|c|c|c|c|}
\cline{2-13}
& \multicolumn{3}{|c|}{\texttt{YMMR}} & \multicolumn{3}{|c|}{\texttt{WAN}} & \multicolumn{3}{|c|}{\texttt{LNKD-SSD}} & \multicolumn{3}{|c|}{\texttt{LNKD-DISK}}\\
&\multicolumn{1}{|c}{$L_r$}  & \multicolumn{1}{c}{$L_w$} & \multicolumn{1}{c|}{$t$} &  \multicolumn{1}{|c}{$L_r$} & \multicolumn{1}{c}{$L_w$} & \multicolumn{1}{c|}{$t$} &  \multicolumn{1}{|c}{$L_r$} & \multicolumn{1}{c}{$L_w$} & \multicolumn{1}{c|}{$t$} &  \multicolumn{1}{|c}{$L_r$} & \multicolumn{1}{c}{$L_w$} & \multicolumn{1}{c|}{$t$} \\\hline
\multicolumn{1}{|c|}{$R$$=$$1$, $W$$=$$1$}
& 5.58 & 10.83 & 1364.0 & 3.4 & 55.12 & 113.0 & 0.66 & 0.66 & 1.85 & 0.66 & 10.99 & 45.5 \\
\multicolumn{1}{|c|}{$R$$=$$1$, $W$$=$$2$}
& 5.61 & 427.12 & 1352.0 & 3.4 & 167.64 & 0 & 0.66 & 1.63 & 1.79 & 0.65 & 20.97 & 43.3 \\
\multicolumn{1}{|c|}{$R$$=$$2$, $W$$=$$1$}
& 32.6 & 10.73 & 202.0 & 151.3 & 56.36 & 30.2 & 1.63 & 0.65 & 0 & 1.63 & 10.9 & 13.6 \\
\multicolumn{1}{|c|}{$R$$=$$2$, $W$$=$$2$}
& 33.18 & 428.11 & 0 & 151.31 & 167.72 & 0 & 1.62 & 1.64 & 0 & 1.64 & 20.96 & 0 \\
\multicolumn{1}{|c|}{$R$$=$$3$, $W$$=$$1$}
& 219.27 & 10.79 & 0 & 153.86 & 55.19 & 0 & 4.14 & 0.65 & 0 & 4.12 & 10.89 & 0 \\
\multicolumn{1}{|c|}{$R$$=$$1$, $W$$=$$3$}
& 5.63 & 1870.86 & 0 & 3.44 & 241.55 & 0 & 0.65 & 4.09 & 0 & 0.65 & 112.65 & 0 \\
\hline
\end{tabular}
\caption{$t$-visibility for $p_{s} = .001$ ($99.9\%$ probability
  of consistency for $50,000$ reads and writes) and $99.9\%$ read
  ($L_r$) and write latency ($L_w$) across $R$ and $W$, $N$$=$$3$
  ($1M$ reads and writes).}
\label{table:lat-stale}
\end{table*}


\section{Discussion and Future Work}
\label{sec:discussion}

PBS enables several enhancements to partial quorum systems that we
have not yet addressed.  Here, we briefly discuss them along with
improvements to our models.

\textbf{Latency/Staleness SLAs.} Using PBS, we can automatically
configure the configuration of replication parameters.  We can
optimize for operation latency given constraints on staleness and
minimum durability.  Data store operators can subsequently provide
service level agreements to applications and provide quantitative
trade-offs to developers, allowing them to reason about staleness
versus end-user latency.  Operators can determine configurations
analytically and revise them given online feedback.  PBS provides a
quantitative lens for analyzing consistency guarantees that were
previously unknown.  While this optimization formulation is likely
convex, the state space for configurations is rather small ($O(N^2)$).
This optimization also allows disentanglement of replication for
reasons of durability from replication for reasons of low latency and
higher capacity.  For example, operators can specify a minimum
replication factor for durability and availability but may want to
automatically increase $N$, which decreases tail latency for a fixed
$R$ and $W$.

\textbf{Variable configurations.} We have assumed the use of a single
replica configuration ($N$, $R$, and $W$) across all operations.
However, one could consider varying these operations over time and
across keys, and many KVSs such as Cassandra and Riak allow the use of
per-operation consistency settings.  By specifying an \textit{average}
operation latency, one could periodically modify $R$ and $W$ to more
efficiently guarantee a desired bound on staleness.  Similarly, under
scenarios with unexpectedly heavy load, one might consider increasing
the number of replicas and scaling $R$ and $W$ accordingly, requiring
additional refinements to our model, essentially revisiting prior work
on fluid replication~\cite{fluidreplication}.

\textbf{Stronger guarantees.} In this paper, we have focused on
probabilistic staleness analysis, but there are a range of stronger
models such as causal consistency within the spectrum of eventual
consistency~\cite{vogels-defs}.  Predicting the probability of
attaining more complex consistency semantics requires additional
modeling of application access patterns.  For example, to attain
causal session guarantees, we would need to know which replicas are
contacted and whether clients inform one another about causal
relationships out-of-band from the data store.  This is possible, but
modeling the \textit{worst-case} semantics of these operations is
likely to result in unfavorably low bounds on consistency properties.
We can see this in Aiyer et al.'s analysis of Byzantine
$k$-quorums~\cite{multi-k-quorum}.  In a worst-case deployment, with
an adversarial scheduler, the lower bound on staleness is quite high;
we conjecture that the bound would be even higher had the authors
performed an analysis of stronger consistency models.

\textbf{Alternative architectures.} We have analyzed Dynamo-style
$t$-visibility in depth.  As we discussed, Dynamo is conceptually easy
to understand and implement (\textit{WARS}) but is painful to analyze
analytically, leading us to favor Monte Carlo analysis.  Dynamo is
surprisingly resilient to inconsistency in practice, but attaining
\textit{provable} (even probabilistically provable) $t$-staleness
bounds would be desirable for applications that cannot tolerate
statistical error.  While research on deterministic bounded staleness
(Section~\ref{sec:relatedwork}) provides provable deterministic bounds
(often sacrificing availability), it is unclear whether there is a
design that finds the middle ground between operational elegance and
easy, provable analysis in the eventually consistent protocol design
space.

\textbf{Multi-key operations.} We have considered single-key
operations, however the ability to perform multi-key operations is
potentially attractive.  For read-only transactions, if the key
distribution is random and each quorum is independent, we can simply
multiple the staleness probabilities of each key.  Achieving atomicity
of writes to multiple keys requires more complicated coordination
mechanisms such as two-phase commit, increasing operation latency.
Transactions are feasible but require considerable care in
implementation, complicating what is otherwise a simple replication
scheme.

\textbf{Failure modes.} In our evaluation of $t$-visibility, we
focused on normal operating conditions. Unless failures are
common-case, they affect $p_{stale}$ at the tail.  If, as Jeff Dean of
Google suggests~\cite{dean-keynote}, servers crash at least twice per
year, assuming a worst-case ten hour downtime for machines, this
roughly represents .23\% downtime per machine.  If failures are
correlated, this small percentage may be a problem.  If they are
independent, a replica set of $N$ failed nodes with $F$ failed nodes
behaves like an $N-F$ replica set, and the probability of all $N$
nodes failing is $(.23)^N$\% (``five nines'' reliability for
$N$$=$$3$).  However, it would be beneficial to quantify this impact
more precisely given actual failure distributions.  We have asserted
that failures or latency spikes can be accommodated in \textit{WARS}
by adjusting latency distributions to match failure probabilities.
However, modeling the probability of treating replicas as inactive
(while it is just hung) and modeling recovery semantics such as hinted
handoff would be fruitful.  This requires additional information about
failure rates and additional care in gathering latency distributions.

\section{Related Work}
\label{sec:relatedwork}

We surveyed quorum replication
techniques~\cite{prob-quorum-dynamic, treequorum,non-strict,
  multi-k-quorum, quorums-start, quorum-placement, partitionedquorum, quorums-alternative, prob-quorum,
  quorum-overview, quorumsystems} in Section~\ref{sec:background}.  In
this work, we specifically draw inspiration from probabilistic
quorums~\cite{prob-quorum} and deterministic
$k$-quorums~\cite{multi-k-quorum, non-strict} in analyzing
write-propagating quorum systems and their staleness.  We believe that
revisiting probabilistic quorum systems---including non-majority
quorum systems such as tree quorums---in the context of write
propagation, dynamism, and Dynamo is a promising area for theoretical
work.

Data consistency is a long-studied problem in distributed
systems~\cite{consistency-partitioned, danger-rep} and concurrent
programming~\cite{linearizability}.  Given the CAP Theorem and the
inability to maintain all three of consistency, availability, and
partition tolerance~\cite{cap-proof}, data systems have turned to
``eventually consistent'' semantics to provide availability in the
face of partitions~\cite{consistency-partitioning, vogels-defs}.
Real-time consistency (RTC) is the strongest consistency model
available in an available, one-way convergent (eventually consistent)
system~\cite{rtc-proof}, however there is a plethora of alternative
consistency models offering various performance tradeoffs, from
session guarantees~\cite{sessionguarantees} to causal+
consistency~\cite{cops} and parallel snapshot isolation~\cite{walter}.
Instead of proposing a new consistency model and building a system
implementing new semantics, we have examined what consistency
existing, widely deployed quorum-replicated systems actually provide.

There are several systems that provide deterministic bounds on
staleness.  FRACS~\cite{frac} allows replicas to buffer updates up to
a given staleness threshold under several replication schemes,
including master-drive and group gossip.  AQuA~\cite{aqua}
asynchronously propagates updates from a designated master to replicas
that in turn serve reads depending on probabilistic models regarding
their staleness.  AQuA actively selects which replica to contact
depending on the staleness bound and replica staleness predictions.
TRAPP~\cite{trapp} provides tradeoffs between precision and
performance for continuously evolving numerical data.
TACT~\cite{vahdat-article, vahdat-bounded} models consistency along
three axes: numerical error, order error, and staleness.  TACT bounds
staleness by ensuring that each replica makes (transitive) contact
with all other replicas in the system within a given time window.
Finally, PIQL~\cite{piql} bounds the number of operations performed
per query, trading operation latency at scale with the amount of data
a particular query can access, potentially impacting accuracy.

In this work, we analyze quorum replication systems and study the
properties real-world Dynamo-style quorum systems in depth.  Many of
the aforementioned deterministic bounded staleness systems represent the
deterministic dual of PBS, and their bounding algorithms could likely
be employed in an expanding partial quorum system like Dynamo.

\section{Conclusion}
\label{sec:conclusion}

In this paper, we developed models for the staleness of values
returned from eventually consistent quorum-replicated data stores.  By
extending prior theory on probabilistic quorum systems, we derived an
analytical solution for the $k$-staleness of a partial quorum system,
representing the expected staleness of a read operation in terms of
versions.  We analyzed the $t$-visibility, or expected staleness of a
read in terms of real time, under Dynamo-style quorum replication.  To
do so, we developed the the \textit{WARS} latency model to explain how
message reordering leads to staleness under Dynamo.  To examine the
effect of latency on $t$-staleness in practice, we used real-world
traces from internet companies to drive a Monte Carlo analysis.  We
find that eventually consistent quorum-replicated data stores are
frequently consistent after only a few milliseconds, explaining the
prevalence of partial-quorum operation configurations in practice.  We
conclude that ``eventually consistent'' partial quorum replication
schemes frequently deliver consistent semantics in practice due largely to the
resilience of Dynamo-style messaging.

\section*{Interactive Demonstration}

An interactive demonstration of Dynamo-style PBS is available online at \url{http://cs.berkeley.edu/~pbailis/pbs/}.  The username is \texttt{demo} and the password is \texttt{vldborbust}.

\section*{Acknowledgements}

The authors would like to thank Alex Feinberg and Coda Hale for their
cooperation in providing real-world distributions for experiments and for
exemplifying positive industrial-academic relations through their conduct.

The authors would also like to thank the following individuals whose
discussions and feedback improved this work: Marcos Aguilera, Peter
Alvaro, Eric Brewer, Neil Conway, Greg Durrett, Hariyadi Gunawi, Bryan
Kate, Sam Madden, Bill Marczak, Kay Ousterhout, Christopher R\'e,
Scott Shenker, Sriram Srinivasan, Doug Terry, Greg Valiant, and
Patrick Wendell, \textbf{YOUR NAME HERE, BOLD READER}!  We would
especially like to thank Ali Ghodsi, who, in addition to providing
feedback, originally piqued our interest in theoretical quorum
systems.

This work was supported in part by gifts from Google, SAP,
Amazon Web Services, Cloudera, Ericsson, Huawei, IBM, Intel,
MarkLogic, Microsoft, NEC Labs, NetApp, Oracle, Splunk, and VMware and
by DARPA (contract \#FA8650-11-C-7136).  This material is based upon
work supported by the National Science Foundation Graduate Research
Fellowship under Grant No. DGE 1106400.


\begin{comment}
Andy Gross
Justin Sheehy


\end{comment}

\balance

{\small
\bibliographystyle{abbrv}
\bibliography{ernst}
}

\begin{appendix}
\section{PBS Quorum Load}
Theory literature describes the \textit{load} of a quorum system as a
metric for the frequency of accessing the busiest quorum
member~\cite[Definition 3.2]{quorumsystems}.  Intuitively, the busiest
quorum member limits the number of requests that a given quorum system
can sustain, called its \textit{capacity}~\cite[Corollary
  3.9]{quorumsystems}.

Prior work determined that probabilistic quorum systems did not offer
significant benefits to load (providing a constant factor improvement
compared to strict quorum systems)~\cite{prob-quorum}.  Here, we show
that PBS $k$-quorums provide asymptotically lower load than
traditional probabilistic quorum systems (and, transitively, than
strict quorum systems).

The probabilistic quorum literature defines an
$\varepsilon$-intersecting quorum system as a quorum system that
provides a $1-\varepsilon$ probability of returning consistent
data~\cite[Definition 3.1]{prob-quorum}.  A $\varepsilon$-intersecting
quorum system has a lower bound load of
$\frac{1-\sqrt{\varepsilon}}{\sqrt{N}}$~\cite[Corollary
  3.12]{prob-quorum}.

In considering $k$ versions of staleness, we consider the intersection
of $k$ $\varepsilon$-intersecting quorum systems.  For a given
probability $p$ of inconsistency, if we are willing to tolerate $k$
versions of staleness, we need only require that that $\varepsilon =
\sqrt[k]{p}$.  This implies that our PBS $k$-quorum system
construction has load of at least
$\frac{(1-p)^{\frac{1}{2k}}}{\sqrt{N}}$, an improved lower bound
compared to traditional probabilistic quorum systems.  PBS monotonic
reads results in a lower bound on load of
$\frac{(1-p)^{\frac{1}{2C}}}{\sqrt{N}}$, where
$C=1+\frac{\gamma_{gw}}{\gamma_{cr}}$.

These results are intuitive: if we are willing to tolerate multiple
versions of staleness, we need to contact fewer replicas.  However, we
believe this is the first formal demonstration that staleness
tolerance lowers the load requirement for a quorum system,
subsequently increasing its capacity.


\end{appendix}


\end{document}

